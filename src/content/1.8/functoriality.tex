% !TEX root = ../../ctfp-print.tex

\lettrine[lhang=0.17]{関}{手}が何であるか理解した今、小さな関手から大きな関手を構成する方法を見てみましょう。特に、圏間の対象に対応する型コンストラクタが、射に対応するマッピングも含む関手に拡張できるかどうかを見るのが興味深いです。

\section{双関手}

関手は$\Cat$ (圏の圏) の射なので、射について、特に関数についての直感が関手にも適用されます。例えば、2つの引数を取る関数があるように、2つの引数を取る関手、すなわち\newterm{双関手}が存在します。双関手は、圏$\cat{C}$の対象と圏$\cat{D}$の対象の各ペアを、圏$\cat{E}$の対象に写像します。これは、圏$\cat{C}\times{}\cat{D}$から$\cat{E}$への写像と言っているのと同じです。

\begin{figure}[H]
  \centering\includegraphics[width=0.3\textwidth]{images/bifunctor.jpg}
\end{figure}

\noindent
これは比較的わかりやすいですが、関手性により、双関手は射も写像する必要があります。ただし、今回は$\cat{C}$からの射と$\cat{D}$からの射のペアを$\cat{E}$の射に写像する必要があります。

再び、射のペアは単なる圏$\cat{C}\times{}\cat{D}$から$\cat{E}$への単一の射です。圏のデカルト積における射を定義すると、一つの対象ペアから別の対象ペアへ行く射のペアです。これらの射ペアは明らかな方法で合成されます: 
\[(f, g) \circ (f', g') = (f \circ f', g \circ g')\]
合成は結合的であり、恒等射ペア$(\id, \id)$が存在します。したがって、圏のデカルト積もまた圏です。

双関手についてもう少し簡単に考える方法は、それをそれぞれの引数に対して関手として考えることです。したがって、関手から双関手への関手則 --- 結合性と恒等性を持つこと --- を翻訳する代わりに、それぞれの引数に対して別々にそれらをチェックするだけで十分です。しかし、一般的には、個別の関手性だけでは共同関手性を証明するには不十分です。共同関手性が失敗する圏は\newterm{プレモノイダル}と呼ばれます。

Haskellで双関手を定義しましょう。この場合、すべての三つの圏は同じです: Haskell型の圏。双関手は2つの型引数を取る型コンストラクタです。ここに\code{Control.Bifunctor}ライブラリから直接取った\code{Bifunctor}型クラスの定義があります: 

\src{snippet01}

\begin{figure}[H]
  \centering\includegraphics[width=0.3\textwidth]{images/bimap.jpg}
  \caption{bimap}
\end{figure}

型変数\code{f}が双関手を表しています。すべての型シグネチャにおいて、常に2つの型引数に適用されていることがわかります。
最初の型シグネチャは\code{bimap}を定義しています: 一度に2つの関数のマッピングです。結果は持ち上げられた関数、
\code{(f a b -> f c d)}で、双関手の型コンストラクタによって生成された型に作用します。\code{bimap}の\code{Control.Bifunctor}と
\code{second}に関してのデフォルト実装があります。 (前に述べたように、これは常に機能するとは限らず、二つのマップが可換でない場合があります、つまり\code{first g . second h}が\code{second h . first g}と同じでないかもしれません。) 

\noindent
他の2つの型シグネチャ、\code{first}と\code{second}は、それぞれ最初と二番目の引数における\code{f}の関手性を証明する\code{fmap}の二つです。

\begin{figure}[H]
  \centering
  \begin{minipage}{0.45\textwidth}
    \centering
    \includegraphics[width=0.65\textwidth]{images/first.jpg} % first figure itself
    \caption{first}
  \end{minipage}\hfill
  \begin{minipage}{0.45\textwidth}
    \centering
    \includegraphics[width=0.6\textwidth]{images/second.jpg} % second figure itself
    \caption{second}
  \end{minipage}
\end{figure}

\noindent
型クラス定義は、\code{bimap}に関してのデフォルト実装の両方を提供します。

\code{Bifunctor}のインスタンスを宣言するとき、\code{bimap}を実装して\code{first}と\code{second}のデフォルトを受け入れるか、\code{first}と\code{second}の両方を実装して\code{bimap}のデフォルトを受け入れるかの選択があります (もちろん、すべての三つを実装することもできますが、その場合はそれらがこのように関連していることを自分で確認する必要があります) 。

\section{積と余積の双関手}

双関手の重要な例は、普遍構成で定義される圏の積―二つの対象の積です。任意の対象ペアに対して積が存在する場合、それらの対象から積へのマッピングは双関手的です。これは一般的にも、特にHaskellにおいても真です。ここにペアコンストラクタ―最も単純な積型のための\code{Bifunctor}インスタンスがあります: 

\src{snippet02}[b]
選択肢はほとんどありません: \code{bimap}は単に最初の関数をペアの最初のコンポーネントに適用し、二番目の関数を二番目のコンポーネントに適用します。コードは、型が与えられると自動的に書かれます: 

\src{snippet03}
この場合、双関手の作用は型のペアを作ることです。例えば: 

\begin{snip}{haskell}
(,) a b = (a, b)
\end{snip}
双対性によって、余積も、圏内の任意の対象ペアに対して定義されていれば、双関手です。Haskellでは、\code{Either}型コンストラクタが\code{Bifunctor}のインスタンスであることによって示されています: 

\src{snippet04}[b]
このコードも自動的に書かれます。

さて、モノイダル圏について話したことを覚えていますか?モノイダル圏は、対象に作用する二項演算子と、単位対象を定義します。$\Set$がデカルト積に関してモノイダル圏であり、単集合を単位としていること、また、非交和に関しても空集合を単位とするモノイダル圏であることを言及しました。言及していないのは、モノイダル圏にとって二項演算子が双関手であるという要求です。これは非常に重要な要求です - 私たちはモノイダル積が、射によって定義される圏の構造と互換性を持っていることを望みます。これで、モノイダル圏の完全な定義に一歩近づきました (自然性について学ぶ前にはそこに到達できませんが) 。

\section{関手的な代数的データ型}

私たちは、\code{fmap}を定義できるパラメータ化されたデータ型のいくつかの例を見てきました --- それらは関手でした。複雑なデータ型は、単純なデータ型から構成されます。特に、代数的データ型 (\acronym{ADT}) は、和と積を使用して作成されます。私たちはすでに和と積が関手的であることを見てきました。また、関手は合成可能です。したがって、\acronym{ADT}の基本的な構成要素が関手的であることを示すことができれば、パラメータ化された\acronym{ADT}も関手的であるとわかります。

では、パラメータ化された代数的データ型の構成要素は何でしょうか?まず、関手の型パラメーターに依存しない項目があります。例えば、\code{Maybe}の\code{Nothing}や\code{List}の\code{Nil}です。これらは\code{Const}関手に相当します。\code{Const}関手はその型パラメーターを無視します (実際には、私たちにとって興味のある\emph{二番目}の型パラメーターを無視します。最初のものは一定です) 。

次に、型パラメーター自体を単純にカプセル化する要素があります。例えば、\code{Maybe}の\code{Just}です。これらは恒等関手に相当します。以前、恒等関手を\emph{Cat}の恒等射として言及しましたが、Haskellでの定義は示しませんでした。それはこちらです: 

\src{snippet05}

\src{snippet06}
\code{Identity}を、型\code{a}の常に1つの (不変な) 値を格納する最も単純なコンテナと考えることができます。

その他のすべての代数的データ構造は、積と和を使って、この2つのプリミティブな要素から構成されます。

この新しい知識を持って、\code{Maybe}型コンストラクタを新たな視点から見てみましょう: 

\src{snippet07}
これは2つの型の和ですが、今では和が関手的であることがわかります。最初の部分、\code{Nothing}は、\code{a}に作用する\code{Const ()}として表現できます (\code{Const}の最初の型パラメーターは単位に設定されています - 後で\code{Const}のより興味深い使用法を見ていきます) 。二番目の部分は、単に恒等関手の異なる名前です。私たちは、\code{Maybe}を同型まで\code{Either}と2つの関手、\code{Const ()}と\code{Identity}の合成として定義できました: 

\src{snippet08}
したがって、\code{Maybe}は、2つの関手、\code{Const ()}と\code{Identity}で部分適用された双関手\code{Either}の合成です。 (\code{Const}は本当に双関手ですが、ここでは常に部分適用されています。) 

私たちはすでに、関手の合成が関手であることを見てきました - 双関手と2つの関手の合成が射に対してどのように機能するかを理解するだけです。2つの射を与えられたら、1つを1つの関手で、もう1つを別の関手で持ち上げます。そして、双関手で持ち上げられた射のペアを持ち上げます。

Haskellでこの合成を表現しましょう。新しいデータ型を定義しますが、それは双関手\code{bf} (2つの型引数を取る型コンストラクタである型変数) 、2つの関手\code{fu}と\code{gu} (それぞれ1つの型変数を取る型コンストラクタ) 、そして2つの通常の型\code{a}と\code{b}によってパラメータ化されています。私たちは\code{fu}を\code{a}に適用し、\code{gu}を\code{b}に適用し、その後で\code{bf}を結果の2つの型に適用します: 

\src{snippet09}
これが対象、つまり型上の合成です。Haskellでは型コンストラクタを型に適用する方法は、関数を引数に適用する方法と同じです。

もし少し迷っているなら、\code{BiComp}を\code{Either}、\code{Const ()}、\code{Identity}、\code{a}、そして\code{b}に、この順番で適用してみてください。そうすると、私たちの\code{Maybe b}の骨組みバージョンを回復することができます (\code{a}は無視されます) 。

新しいデータ型\code{BiComp}は、\code{a}と\code{b}において双関手ですが、\code{bf}自体が\code{Bifunctor}であり、\code{fu}と\code{gu}が\code{Functor}である場合に限ります。コンパイラは\code{bf}に対して\code{bimap}の定義が、そして\code{fu}と\code{gu}に対して\code{fmap}の定義が存在することを知っていなければなりません。Haskellでは、これはインスタンス宣言における前提条件として表現されます: 一連のクラス制約に続いて二重射があります: 

\src{snippet10}[b]
\code{BiComp}の\code{bimap}の実装は、\code{bf}の\code{bimap}と2つの\code{fmap}の用語で与えられます。コンパイラは自動的にすべての型を推測し、\code{BiComp}が使用されるたびに正しいオーバーロードされた関数を選択します。

\code{bimap}の定義における\code{x}は次の型を持ちます: 

\src{snippet11}
これはかなり多くのものです。外側の\code{bimap}は外側の\code{bf}層を通過し、2つの\code{fmap}はそれぞれ\code{fu}と\code{gu}の下を掘ります。もし\code{f1}と\code{f2}の型が: 

\src{snippet12}
ならば、最終結果は型\code{bf (fu a') (gu b')}になります: 

\src{snippet13}[b]
もしジグソーパズルが好きなら、このような型操作で何時間も楽しむことができるでしょう。

したがって、\code{Maybe}が関手であることを証明する必要はなかったことがわかります - これは、2つの関手的原始からの和として構成された方法から導かれました。

洞察に富んだ読者は次のような質問をするかもしれません: 代数的データ型のための\code{Functor}インスタンスの導出がこれほど機械的なら、コンパイラによって自動化され、実行されることはできないのでしょうか?実際にはでき、されています。あなたはソースファイルの先頭にこの行を含めることによって特定のHaskell拡張機能を有効にする必要があります: 

\begin{snip}{haskell}
{-# LANGUAGE DeriveFunctor #-}
\end{snip}
そして、データ構造に\code{deriving Functor}を追加します: 

\begin{snip}{haskell}
data Maybe a = Nothing | Just a deriving Functor
\end{snip}
そして対応する\code{fmap}があなたのために実装されます。

代数的データ構造の規則性は、\code{Functor}だけでなく、以前言及した\code{Eq}型クラスを含むいくつかの他の型クラスのインスタンスも導出することが可能にします。また、独自の型クラスのインスタンスをコンパイラに導出させる方法を教えるオプションもありますが、それは少し高度です。しかし、考え方は同じです: 基本的な構成要素と和と積のための動作を提供し、残りはコンパイラに任せます。

\section{C++における関手}

もしC++プログラマであれば、関手を実装することに関しては自分自身で行うことになるでしょう。しかし、C++において代数的データ構造のいくつかの型を認識することができるはずです。そのようなデータ構造がジェネリックテンプレートにされた場合、それに対して\code{fmap}を迅速に実装することができるはずです。

木データ構造を見てみましょう。これはHaskellでは再帰的な和型として定義されるでしょう: 

\src{snippet14}
前に言及したように、和型をC++で実装する一つの方法はクラス階層を通じてです。オブジェクト指向言語で自然なのは、基底クラス\code{Functor}の仮想関数として\code{fmap}を実装し、すべてのサブクラスでそれをオーバーライドすることです。残念ながら、これは\code{fmap}がテンプレートであり、それが作用するオブジェクトの型 (\code{this}ポインター) だけでなく、それに適用された関数の戻り型によってもパラメータ化されているため、不可能です。C++では、仮想関数はテンプレート化できません。私たちは\code{fmap}をジェネリックな自由関数として実装し、\code{dynamic\_cast}でパターンマッチングを置き換えます。

基底クラスは少なくとも1つの仮想関数を定義する必要があります。動的キャストをサポートするために、私たちはデストラクターを仮想にします (いずれにせよ、良いアイデアです) : 

\begin{snip}{cpp}
template<class T>
struct Tree {
    virtual ~Tree() {}
};
\end{snip}
\code{Leaf}は単なる\code{Identity}関手の変形です: 

\begin{snip}{cpp}
template<class T>
struct Leaf : public Tree<T> {
    T _label;
    Leaf(T l) : _label(l) {}
};
\end{snip}
\code{Node}は積型です: 

\begin{snip}{cpp}
template<class T>
struct Node : public Tree<T> {
    Tree<T> * _left;
    Tree<T> * _right;
    Node(Tree<T> * l, Tree<T> * r) : _left(l), _right(r) {}
};
\end{snip}
\code{fmap}を実装するときには、\code{Tree}の型に対する動的ディスパッチを利用します。\code{Leaf}ケースは\code{Identity}版の\code{fmap}を適用し、\code{Node}ケースは2つのコピーの\code{Tree}関手で合成された双関手のように扱われます。C++プログラマとして、このような用語でコードを分析することに慣れていないかもしれませんが、圏論的な思考における良い練習です。

\begin{snip}{cpp}
template<class A, class B>
Tree<B> * fmap(std::function<B(A)> f, Tree<A> * t) {
    Leaf<A> * pl = dynamic_cast <Leaf<A>*>(t);
    if (pl)
        return new Leaf<B>(f (pl->_label));
    Node<A> * pn = dynamic_cast<Node<A>*>(t);
    if (pn)
        return new Node<B>( fmap<A>(f, pn->_left)
                          , fmap<A>(f, pn->_right));
    return nullptr;
}
\end{snip}
簡単のために、私はメモリとリソース管理の問題を無視することにしましたが、実際のコードではおそらくスマートポインターを使用するでしょう (共有か非共有かはあなたの方針次第)。

Haskellの\code{fmap}の実装と比較してみてください: 

\src{snippet15}
この実装もまた、コンパイラによって自動的に導出されることができます。

\section{Writer関手}

以前に紹介した\hyperref[kleisli-categories]{Kleisli圏}に戻ることを約束しました。その圏の射は、\code{Writer}データ構造を返す「装飾された」関数として表されました。

\src{snippet16}
その装飾が何らかの形で自己関手に関連していると言いました。そして、実際に\code{Writer}型コンストラクタは\code{a}において関手的です。それに対して\code{fmap}を実装する必要すらありませんでした。なぜなら、それは単なる単純な積型だからです。

しかし、一般的には、Kleisli圏と関手の関係は何でしょうか?Kleisli圏は圏であり、合成と恒等性を定義します。合成は魚演算子によって与えられます: 

\src{snippet17}
そして、\code{return}と呼ばれる関数による恒等射があります: 

\src{snippet18}
実は、これら2つの関数の型を十分に長い間 (本当に、\emph{長い}間) 見ていれば、\code{fmap}として機能する正しい型シグネチャの関数を生成する方法を見つけることができます。このように: 

\src{snippet19}
ここでは、魚演算子は2つの関数を組み合わせています: 1つはおなじみの\code{id}、もう1つは\code{return}を\code{f}のラムダの引数に適用するラムダです。\code{id}の使用について頭を悩ませるかもしれません。魚演算子の引数は「通常の」型から装飾された型に変換する関数ではないのでしょうか?実際にはそうではありません。\code{a}の\code{a -> Writer b}は「通常の」型である必要はありません。それは型変数なので、何でもよく、特に\code{Writer b}のような装飾された型になることができます。

したがって、\code{id}は\code{Writer a}を受け取り、\code{Writer a}に変換します。魚演算子は\code{a}の値を釣り出し、ラムダに\code{x}として渡します。そこで\code{f}はそれを\code{b}に変換し、\code{return}はそれを装飾し、\code{Writer b}にします。すべてを一緒に考えると、\code{Writer a}を取り、\code{Writer b}を返す関数が得られます。これはちょうど\code{fmap}が生成すべきものです。

この議論は非常に一般的です: あなたは\code{Writer}を任意の型コンストラクタに置き換えることができます。魚演算子と\code{return}をサポートする限り、\code{fmap}を定義することができます。したがって、Kleisli圏の装飾は常に関手です。 (しかし、すべての関手がKleisli圏を生み出すわけではありません。) 

あなたは、私たちが定義した\code{fmap}が、\code{deriving Functor}でコンパイラが私たちのために導出してくれる\code{fmap}と同じものかどうか疑問に思うかもしれません。興味深いことに、それはそうです。これは、Haskellが多相関数を実装する方法によるものです。それは\newterm{パラメトリック多相性}と呼ばれ、いわゆる\newterm{無料の定理}の源です。そのような定理の1つは、与えられた型コンストラクタに対して恒等性を保つ\code{fmap}の実装がある場合、それは一意でなければならないというものです。

\section{共変関手と反変関手}

writer関手を見直したので、reader関手に戻ってみましょう。それは部分適用された関数矢印型コンストラクタに基づいていました: 

\src{snippet20}
これを型シノニムとして書き換えることができます: 

\src{snippet21}
私たちが以前見たように、そのための\code{Functor}インスタンスは次のようになります: 

\src{snippet22}
しかし、ペア型コンストラクタや\code{Either}型コンストラクタのように、関数型コンストラクタは2つの型引数を取ります。ペアと\code{Either}は両方の引数において関手的でした - それらは双関手でした。関数コンストラクタも双関手ですか?

最初の引数において関手的にすることを試みましょう。型シノニムを使いましょう。これは\code{Reader}と似ていますが、引数が反転しています: 

\src{snippet23}
この場合、戻り値の型\code{r}を固定し、引数の型\code{a}を変えます。\code{fmap}を実装しようとすると、それは以下のような型シグネチャを持つことになります: 

\src{snippet24}
\code{a}を取り、それぞれ\code{b}と\code{r}を返す2つの関数を持っているだけでは、\code{b}を取り\code{r}を返す関数を構成する方法は単純にはありません!しかし、もし最初の関数を何とか反転させることができれば、つまりそれが\code{b}を取り\code{a}を返すようにできれば、異なるものが得られるかもしれません。任意の関数を反転することはできませんが、反対の圏に行くことはできます。

簡単に復習しましょう: 圏$\cat{C}$ごとに、その双対圏$\cat{C}^\mathit{op}$があります。これは$\cat{C}$と同じ対象を持ちますが、すべての射が逆転しています。

$\cat{C}^\mathit{op}$と他の圏$\cat{D}$との間の関手を考えてみましょう: 
\[F \Colon \cat{C}^\mathit{op} \to \cat{D}\]
このような関手は、$\cat{C}^\mathit{op}$内の射$f^\mathit{op} \Colon a \to b$を$\cat{D}$内の射$F f^\mathit{op} \Colon F a \to F b$に写像します。しかし、射$f^\mathit{op}$は秘密裏に元の圏$\cat{C}$のある射$f \Colon b \to a$に対応しています。方向の逆転に注目してください。

ここで、$F$は通常の関手ですが、$F$に基づいて別の写像を定義することができます。それを$G$としましょう。それは$\cat{C}$から$\cat{D}$への写像です。それは対象を$F$と同じ方法で写像しますが、射を写像するときはそれらを逆転します。それは$\cat{C}$の射$f \Colon b \to a$を取り、それを反対の射$f^\mathit{op} \Colon a \to b$に写像し、その後それに関手$F$を適用して、$F f^\mathit{op} \Colon F a \to F b$を得ます。

$F a$が$G a$と同じであり、$F b$が$G b$と同じであると考えると、全体のプロセスは次のように記述できます: $G f \Colon (b \to a) \to (G a \to G b)$
これは「捩れのある関手」です。このように射の方向を逆転させる圏の写像は、\emph{反変関手}と呼ばれます。反変関手は、反対圏からの通常の関手に過ぎません。ちなみに、これまでに研究してきた通常の関手は、\emph{共変関手}と呼ばれます。

\begin{figure}[H]
  \centering
  \includegraphics[width=40mm]{images/contravariant.jpg}
\end{figure}

\noindent
反変関手 (実際には、反変\emph{自己}関手) を定義するHaskellの型クラスがこちらです: 

\src{snippet25}
私たちの型コンストラクタ\code{Op}はそれのインスタンスです: 

\src{snippet26}
関数\code{f}は、\code{Op}の内容の\emph{前} (つまり右) に挿入されることに注意してください。

\code{Op}の\code{contramap}の定義は、引数を反転する特別な関数である\code{flip}を使用することでさらに簡潔にすることができます: 

\src{snippet27}
それを使って、次のようになります: 

\src{snippet28}

\section{プロ関手}

関数矢印演算子は、その最初の引数において反変的であり、二番目の引数において共変的であることがわかりました。このような性質を持つものに名前はありますか?対象圏が$\Set$である場合、このようなものは\newterm{プロ関手}と呼ばれます。反変関手が反対圏からの共変関手と同等であるため、プロ関手は次のように定義されます: 
\[\cat{C}^\mathit{op} \times \cat{D} \to \Set\]
第一近似として、Haskellの型は集合であるため、2つの引数を持つ型コンストラクタ\code{p}に\code{Profunctor}という名前を適用します。それは最初の引数において反関手的であり、二番目の引数において関手的です。こちらが\code{Data.Profunctor}ライブラリからの適切な型クラスです: 

\src{snippet29}[b]
すべての3つの関数はデフォルト実装付きです。\code{Bifunctor}と同じように、\code{Profunctor}のインスタンスを宣言する際には、\code{dimap}を実装して\code{lmap}と\code{rmap}のデフォルトを受け入れるか、または\code{lmap}と\code{rmap}の両方を実装して\code{dimap}のデフォルトを受け入れる選択があります。

\begin{figure}[H]
  \centering
  \includegraphics[width=0.4\textwidth]{images/dimap.jpg}
  \caption{dimap}
\end{figure}

\noindent
これで、関数矢印演算子が\code{Profunctor}のインスタンスであると断言することができます: 

\src{snippet30}[b]
プロ関手は、Haskellのlensライブラリでの応用があります。私たちは、エンドと余エンドについて話すときに再びそれらを見るでしょう。

\section{ホム関手}

上記の例は、対象のペア$a$と$b$を取り、それらの間の射の集合、ホム集合$\cat{C}(a, b)$に割り当てる写像が関手であるというより一般的な主張の反映です。それは$\cat{C}^\mathit{op}\times{}\cat{C}$から集合の圏$\Set$への関手です。

射に対するその作用を定義しましょう。$\cat{C}^\mathit{op}\times{}\cat{C}$の射は$\cat{C}$からの射のペアです: 
\begin{gather*}
  f \Colon a' \to a \\
  g \Colon b \to b'
\end{gather*}
このペアを持ち上げることは、集合$\cat{C}(a, b)$から集合$\cat{C}(a', b')$への射 (関数) でなければなりません。$\cat{C}(a, b)$の任意の要素$h$ (それは$a$から$b$への射です) を取り、それに次のものを割り当てます: 
\[g \circ h \circ f\]
これは$\cat{C}(a', b')$の要素です。

ご覧の通り、ホム関手はプロ関手の特別なケースです。

\section{チャレンジ}

\begin{enumerate}
  \tightlist
  \item
        データ型が双関手であることを示してください: 

        \begin{snip}{haskell}
data Pair a b = Pair a b
\end{snip}

        追加のクレジットとして、\code{Bifunctor}のすべての3つの方法を実装し、これらの定義が適用可能な場合にデフォルト実装と互換性があることを等式推論を使用して示してください。
  \item
        \code{Maybe}の標準定義とこの糖衣構文との間の同型を示してください: 

        \begin{snip}{haskell}
type Maybe' a = Either (Const () a) (Identity a)
\end{snip}

        ヒント: 2つの実装間にマッピングを定義します。追加のクレジットとして、等式推論を使用してそれらが互いに逆であることを示してください。
  \item
        別のデータ構造を試してみましょう。私はそれを\code{PreList}と呼びます、なぜならそれは\code{List}の前身です。それは再帰を型パラメーター\code{b}で置き換えます。

        \begin{snip}{haskell}
data PreList a b = Nil | Cons a b
\end{snip}

        \code{List}を再帰的に\code{PreList}に適用することで、私たちの以前の\code{List}の定義を回復することができます (固定点について話すときにどのように行うかを見ます) 。

        \code{PreList}が\code{Bifunctor}のインスタンスであることを示してください。
  \item
        次のデータ型が\code{a}と\code{b}において双関手を定義することを示してください: 

        \begin{snip}{haskell}
data K2 c a b = K2 c

data Fst a b = Fst a

data Snd a b = Snd b
\end{snip}

        追加のクレジットとして、あなたのソリューションをConor McBrideの論文\urlref{http://strictlypositive.org/CJ.pdf}{Clowns to the Left of
          me, Jokers to the Right}と照合してください。
  \item
        Haskell以外の言語で双関手を定義してください。その言語で汎用ペアに対して\code{bimap}を実装してください。
  \item
        \code{std::map}は2つのテンプレート引数\code{Key}と\code{T}において双関手またはプロ関手と考えるべきでしょうか?このデータ型をどのように再設計するとそうなりますか?
\end{enumerate}
