% !TEX root = ../../ctfp-print.tex

\lettrine[lhang=0.17]{圏}{は}その対象が集合を形成するとき小さいとされます。しかし、集合より大きなものが存在することが知られています。有名な例として、全ての集合の集合は標準的な集合論 (Zermelo-Fraenkel理論、必要に応じて選択公理を付加したもの) 内では形成できません。従って、全ての集合の圏は大きなものでなければなりません。Grothendieck宇宙のような数学的トリックを使って、集合を超えたコレクションを定義することができます。これらのトリックにより、大きな圏について語ることができます。

圏が\emph{局所的に小さい}とは、任意の2つの対象間の射が集合を形成する場合を指します。もし集合を形成しない場合は、いくつかの定義を再考する必要があります。特に、射を集合から選び出すことができない場合、射の合成をどのように意味するかです。解決策として、\emph{対象}を集合圏$\Set$の対象ではなく、別の圏$\cat{V}$の対象に置き換えることによって、ホム集合をブートストラップします。一般に、対象は要素を持たないため、個々の射について話すことはもはや許されません。ホム対象全体に対して実行できる操作の観点から、\emph{豊穣}圏のすべての性質を定義する必要があります。これを行うためには、ホム対象を提供する圏には追加の構造が必要です --- それはモノイダル圏でなければなりません。このモノイダル圏を$\cat{V}$と呼ぶとすると、$\cat{V}$に豊穣化された圏$\cat{C}$について語ることができます。

大きさの理由以外に、単なる集合よりも多くの構造を持つものにホム集合を一般化することに興味があるかもしれません。例えば、伝統的な圏は対象間の距離の概念を持っていません。対象は射によって接続されているか、そうでないかのどちらかです。ある対象に接続されている全ての対象はその近隣です。現実とは異なり、圏では友達の友達の友達は自分の親友と同じくらい近いです。適切に豊穣化された圏では、対象間の距離を定義することができます。

豊穣圏についての経験を積むためのもう一つの非常に実用的な理由は、圏論的知識の非常に便利なオンラインソースである\urlref{https://ncatlab.org/}{nLab}が、主に豊穣圏の観点から書かれているためです。

\section{なぜモノイダル圏か?}

豊穣圏を構成する際には、モノイダル圏を$\Set$と置き換え、ホム対象をホム集合と置き換えたときに、通常の定義を回復できるように心に留めておく必要があります。これを達成する最善の方法は、通常の定義から始め、点を使わない方法 (つまり、集合の要素を名付けることなく) でそれらを再定式化することです。

まず合成の定義から始めましょう。通常、合成はペアの射を取ります。一つは$\cat{C}(b, c)$から、もう一つは$\cat{C}(a, b)$からで、これを$\cat{C}(a, c)$からの射にマップします。言い換えると、それは次のようなマッピングです: 
\[\cat{C}(b, c)\times{}\cat{C}(a, b) \to \cat{C}(a, c)\]
これは集合間の関数です --- そのうちの一つは2つのホム集合のデカルト積です。この公式は、デカルト積をもっと一般的なものに置き換えることで簡単に一般化できます。圏論的積が機能しますが、さらに一般的なテンソル積を使用することもできます。

次に恒等射についてです。ホム集合から個々の要素を選ぶ代わりに、単集合$\cat{1}$からの関数を使ってそれらを定義することができます: 
\[j_a \Colon \cat{1} \to \cat{C}(a, a)\]
再び、単集合を終対象に置き換えることができますが、テンソル積の単位要素$i$に置き換えることでさらに進むことができます。

ご覧のとおり、あるモノイダル圏$\cat{V}$から取られた対象はホム集合の置き換えに適しています。

\section{モノイダル圏}

モノイダル圏について以前話しましたが、定義を再度述べる価値があります。モノイダル圏はテンソル積を定義し、それは双関手です: 
\[\otimes \Colon \cat{V}\times{}\cat{V} \to \cat{V}\]
テンソル積が結合的であることを望んでいますが、自然同型まで満たせば十分です。この同型は結合子 (associator) と呼ばれ、そのコンポーネントは次のとおりです: 
\[\alpha_{a b c} \Colon (a \otimes b) \otimes c \to a \otimes (b \otimes c)\]
これは全ての3つの引数において自然でなければなりません。

モノイダル圏はまた、特別な単位対象$i$を定義しなければなりません。それはテンソル積の単位要素として機能し、再び自然同型までです。2つの同型はそれぞれ左単位子 (left uniter) と右単位子 (right uniter) と呼ばれ、それらのコンポーネントは以下の通りです: 
\begin{align*}
  \lambda_a & \Colon i \otimes a \to a \\
  \rho_a    & \Colon a \otimes i \to a
\end{align*}
結合子と単位子は整合性条件を満たさなければなりません: 

\begin{figure}[H]
  \centering
  \begin{tikzcd}[row sep=large]
    ((a \otimes b) \otimes c) \otimes d
    \arrow[d, "\alpha_{(a \otimes b)cd}"]
    \arrow[rr, "\alpha_{abc} \otimes \id_d"]
    & & (a \otimes (b \otimes c)) \otimes d
    \arrow[d, "\alpha_{a(b \otimes c)d}"] \\
    (a \otimes b) \otimes (c \otimes d)
    \arrow[rd, "\alpha_{ab(c \otimes d)}"]
    & & a \otimes ((b \otimes c) \otimes d)
    \arrow[ld, "\id_a \otimes \alpha_{bcd}"] \\
    & a \otimes (b \otimes (c \otimes d))
  \end{tikzcd}
\end{figure}

\begin{figure}[H]
  \centering
  \begin{tikzcd}[row sep=large]
    (a \otimes i) \otimes b
    \arrow[dr, "\rho_{a} \otimes \id_b"']
    \arrow[rr, "\alpha_{aib}"]
    & & a \otimes (i \otimes b)
    \arrow[dl, "\id_a \otimes \lambda_b"] \\
    & a \otimes b
  \end{tikzcd}
\end{figure}

\noindent
モノイダル圏が対称的であるとは、次のような自然同型のコンポーネントがあることを意味します: 
\[\gamma_{a b} \Colon a \otimes b \to b \otimes a\]
その「二乗が1」です: 
\[\gamma_{b a} \circ \gamma_{a b} = \idarrow[a \otimes b]\]
そしてそれがモノイダル構造と整合している必要があります。

モノイダル圏についての興味深い点は、内部ホム (関数対象) をテンソル積の右随伴として定義できるかもしれないことです。覚えておくべきことは、関数対象、または指数は、圏論的積の右随伴を通して標準的に定義されました。そのような対象が任意の対象のペアに対して存在した圏はデカルト閉と呼ばれました。ここではモノイダル圏における内部ホムの随伴を定義します: 
\[\cat{V}(a \otimes b, c) \sim \cat{V}(a, [b, c])\]
以下に従って、\urlref{http://www.tac.mta.ca/tac/reprints/articles/10/tr10.pdf}{G. M.
Kelly}の記法を使います。内部ホムに${[}b, c{]}$を使用しています。この随伴の余単位は評価射と呼ばれる自然変換のコンポーネントです: 
\[\varepsilon_{a b} \Colon ([a, b] \otimes a) \to b\]
テンソル積が対称的でない場合、次の随伴を使用して別の内部ホムを定義することができます: 
\[\cat{V}(a \otimes b, c) \sim \cat{V}(b, [[a, c]])\]
両方が定義されているモノイダル圏は双閉と呼ばれます。$\Set$における自己関手の圏は、関手合成をテンソル積として使用する、双閉ではない圏の例です。これはモナドを定義するために使用した圏です。

\section{豊穣圏}

モノイダル圏$\cat{V}$に豊穣化された圏$\cat{C}$はホム集合をホム対象に置き換えます。$\cat{C}$の任意の対象ペア$a$と$b$に対して、$\cat{V}$の対象$\cat{C}(a, b)$を関連付けます。ホム対象にも同じ記法を使用しますが、それが射を含まないことを理解しています。一方で$\cat{V}$は通常の (豊穣化されていない) 圏で、ホム集合と射を持っています。従って、集合から完全には脱却していません --- ただそれを隠しています。

$\cat{C}$の個々の射について話すことができないため、射の合成は$\cat{V}$の射の族に置き換えられます: 
\[\circ \Colon \cat{C}(b, c) \otimes \cat{C}(a, b) \to \cat{C}(a, c)\]

\begin{figure}[H]
  \centering
  \includegraphics[width=0.45\textwidth]{images/composition.jpg}
\end{figure}

\noindent
同様に、恒等射は$\cat{V}$の射の族に置き換えられます: 
\[j_a \Colon i \to \cat{C}(a, a)\]
ここで$i$は$\cat{V}$のテンソル単位です。

\begin{figure}[H]
  \centering
  \includegraphics[width=0.4\textwidth]{images/id.jpg}
\end{figure}

\noindent
合成の結合性は$\cat{V}$の結合子に関して定義されます: 

\begin{figure}[H]
  \centering
  \begin{tikzcd}[column sep=large]
    (\cat{C}(c,d) \otimes \cat{C}(b,c)) \otimes \cat{C}(a,b)
    \arrow[r, "\circ\otimes\id"]
    \arrow[dd, "\alpha"]
    & \cat{C}(b,d) \otimes \cat{C}(a,b)
    \arrow[dr, "\circ"] \\
    & & \cat{C}(a,d) \\
    \cat{C}(c,d) \otimes (\cat{C}(b,c) \otimes \cat{C}(a,b))
    \arrow[r, "\id\otimes\circ"]
    & \cat{C}(c,d) \otimes \cat{C}(a,c)
    \arrow[ur, "\circ"]
  \end{tikzcd}
\end{figure}

\noindent
単位則は同様に単位子を用いて表されます: 

\begin{figure}[H]
  \centering
  \begin{subfigure}
    \centering
    \begin{tikzcd}[row sep=large]
      \cat{C}(a,b) \otimes i
      \arrow[rr, "\id \otimes j_a"]
      \arrow[dr, "\rho"]
      & & \cat{C}(a,b) \otimes \cat{C}(a,a)
      \arrow[dl, "\circ"] \\
      & \cat{C}(a,b)
    \end{tikzcd}
  \end{subfigure}
  \hspace{1cm}
  \begin{subfigure}
    \centering
    \begin{tikzcd}[row sep=large]
      i \otimes \cat{C}(a,b)
      \arrow[rr, "j_b \otimes \id"]
      \arrow[dr, "\lambda"]
      & & \cat{C}(b,b) \otimes \cat{C}(a,b)
      \arrow[dl, "\circ"] \\
      & \cat{C}(a,b)
    \end{tikzcd}
  \end{subfigure}
\end{figure}

\section{前順序}

前順序は薄い圏、つまり各ホム集合が空であるか単集合である圏として定義されます。非空集合$\cat{C}(a, b)$を$a$が$b$以下である証明と解釈します。そのような圏は、非常に単純なモノイダル圏である$\mathit{False}$と$\mathit{True}$ (時には$0$と$1$とも呼ばれる) の2つの対象だけを含むモノイダル圏に豊穣化されることができます。必須の恒等射の他に、この圏には$0 \to 1$と行く単一の射があります。それに対して単純なモノイダル構造が確立されており、テンソル積は$0$と$1$の単純な算術をモデリングします (つまり、唯一の非ゼロ積は$1 \otimes 1$です) 。この圏の恒等対象は$1$です。これは厳密なモノイダル圏であり、結合子と単位子は恒等射です。

前順序において、ホム集合は空または単集合なので、私たちはそれを容易に我々の小さな圏からのホム対象に置き換えることができます。豊穣化された前順序$\cat{C}$は、任意の対象ペア$a$と$b$に対するホム対象$\cat{C}(a, b)$を持ちます。もし$a$が$b$以下ならば、この対象は$1$です。そうでなければ$0$です。

合成について見てみましょう。任意の2つの対象のテンソル積は$0$ですが、両方が$1$の場合に限り$1$です。それが$0$の場合、合成射には2つの選択肢があります。$\idarrow[0]$か$0 \to 1$のどちらかです。しかし、それが$1$の場合、唯一の選択肢は$\idarrow[1]$です。これを関係に戻して解釈すると、$a \leqslant b$かつ$b \leqslant c$ならば$a \leqslant c$です。これはまさに必要な推移則です。

恒等射についてはどうでしょうか。それは$1$から$\cat{C}(a, a)$への射です。$1$から行く唯一の射は恒等射$\idarrow[1]$なので、$\cat{C}(a, a)$は$1$でなければなりません。これは$a \leqslant a$であり、前順序の反射則です。従って、推移性と反射性は、前順序を豊穣圏として実装することで自動的に強制されます。

\section{距離空間}

興味深い例の一つは
\urlref{http://www.tac.mta.ca/tac/reprints/articles/1/tr1.pdf}{William
  Lawvere}によるものです。彼は距離空間が豊穣圏を使って定義され得ることに気づきました。距離空間は任意の2つの対象間の距離を定義します。この距離は非負の実数です。無限大を可能な値として含めるのが便利です。距離が無限大の場合、始点の対象から目的の対象へ到達する方法はありません。

距離によって満たされなければならないいくつかの明白な性質があります。その一つは、対象からそれ自身への距離がゼロでなければならないことです。もう一つは三角不等式です。直接の距離は中間停止点での距離の和を超えることはありません。距離が対称的でないことを要求しないのは最初は奇妙に思えるかもしれませんが、Lawvereが説明したように、一方の方向では上り坂を歩いている間に、他方の方向では下り坂を歩いていると想像することができます。いずれにせよ、対称性は後で追加の制約として課されるかもしれません。

では、どのようにして距離空間を圏論の言葉に落とし込むことができるのでしょうか?ホム対象が距離である圏を構成する必要があります。注意してください、距離は射ではなくホム対象です。ホム対象が数である場合にのみ、これらの数が対象であるモノイダル圏$\cat{V}$を構成することができます。非負の実数 (プラス無限) は全順序を形成するので、薄い圏として扱うことができます。2つの数$x$と$y$の間の射は$x \geqslant y$の場合にのみ存在します (注: これは前順序の定義で通常使用される方向とは逆です) 。モノイダル構造は加算によって与えられ、ゼロが単位対象として機能します。言い換えると、2つの数のテンソル積はそれらの和です。

距離空間はそのようなモノイダル圏上で豊穣化された圏です。対象$a$から$b$へのホム対象$\cat{C}(a, b)$は、非負の (無限大の可能性もある) 数であり、それを$a$から$b$への距離と呼びます。恒等射と合成がそのような圏でどのように機能するか見てみましょう。

定義によれば、テンソル単位である数ゼロからホム対象$\cat{C}(a, a)$への射は関係です: 
\[0 \geqslant \cat{C}(a, a)\]
$\cat{C}(a, a)$が非負の数であるため、この条件は$a$から$a$への距離が常にゼロであることを私たちに伝えます。
チェック!

では合成について話しましょう。2つの隣接するホム対象$\cat{C}(b, c) \otimes \cat{C}(a, b)$から始めます。テンソル積は2つの距離の和として定義されました。合成は$\cat{V}$内のこの積から$\cat{C}(a, c)$への射です。$\cat{V}$内の射は大なりイコール関係として定義されます。言い換えると、$a$から$b$への距離と$b$から$c$への距離の和は、$a$から$c$への距離よりも大きいか等しいです。しかし、それは標準的な三角不等式です。チェック!

豊穣化された圏の概念を用いて距離空間を再構成することで、三角不等式と自己距離ゼロの性質を「無料で」得ることができます。

\section{豊穣関手}

関手の定義には射の写像が含まれます。豊穣設定では、個々の射の概念がないため、ホム対象を一括で扱う必要があります。ホム対象はモノイダル圏$\cat{V}$の対象であり、それらの間には射があります。従って、同じモノイダル圏$\cat{V}$に豊穣化された圏間で豊穣関手を定義することは理にかなっています。$\cat{V}$の射を使用して、2つの豊穣圏間のホム対象を写像することができます。

豊穣関手$F$は、2つの圏$\cat{C}$と$\cat{D}$の間で、対象から対象への写像に加えて、$\cat{C}$内の各対象ペアに対して$\cat{V}$の射を割り当てます: 
\[F_{a b} \Colon \cat{C}(a, b) \to \cat{D}(F a, F b)\]
関手は構造を保存する写像です。通常の関手では合成と恒等射を保存することが意味されました。豊穣設定では、合成の保存は以下の図が可換であることを意味します: 

\begin{figure}[H]
  \centering
  \begin{tikzcd}[column sep=large, row sep=large]
    \cat{C}(b,c) \otimes \cat{C}(a,b)
    \arrow[r, "\circ"]
    \arrow[d, "F_{bc} \otimes F_{ab}"]
    & \cat{C}(a,c)
    \arrow[d, "F_{ac}"] \\
    \cat{D}(F b, F c) \otimes \cat{D}(F a, F b)
    \arrow[r,  "\circ"]
    & \cat{D}(F a, F c)
  \end{tikzcd}
\end{figure}

\noindent
恒等射の保存は、「恒等を選ぶ」$\cat{V}$の射の保存に置き換えられます: 

\begin{figure}[H]
  \centering
  \begin{tikzcd}[row sep=large]
    & i \arrow[dl, "j_a"'] \arrow[dr, "j_{F a}"] & \\
    \cat{C}(a,a)
    \arrow[rr, "F_{aa}"]
    & & \cat{D}(F a, F a)
  \end{tikzcd}
\end{figure}

\section{自己豊穣化}

閉じた対称モノイダル圏は、内部ホムを使用してホム集合を置き換えることで自己豊穣化することができます (上記の定義を参照) 。これを実現するために、内部ホムの合成則を定義する必要があります。言い換えれば、次のシグネチャを持つ射を実装する必要があります: 
\[[b, c] \otimes [a, b] \to [a, c]\]
これは、圏論では通常点を使わない実装を使用することを除けば、他のプログラミングタスクとそれほど変わりません。まず、それが所属すべき集合を指定します。この場合、それはホム集合のメンバーです: 
\[\cat{V}([b, c] \otimes [a, b], [a, c])\]
このホム集合は次に同型です: 
\[\cat{V}(([b, c] \otimes [a, b]) \otimes a, c)\]
私たちがこの新しい集合で射を構成できれば、随伴が私たちを元の集合の射に指し示します。これは合成として使用することができます。私たちが利用可能ないくつかの射を合成することによって、この射を構成します。まず、結合子$\alpha_{{[}b, c{]}\ {[}a, b{]}\ a}$を使用して左側の式を再関連付けします: 
\[([b, c] \otimes [a, b]) \otimes a \to [b, c] \otimes ([a, b] \otimes a)\]
それに続いて随伴の余単位$\varepsilon_{a b}$を使用します: 
\[[b, c] \otimes ([a, b] \otimes a) \to [b, c] \otimes b\]
そして再び余単位$\varepsilon_{b c}$を使用して$c$に到達します。これにより、次の射が構成されます: 
\[\varepsilon_{b c}\ .\ (\idarrow[{[b, c]}] \otimes \varepsilon_{a b})\ .\ \alpha_{[b, c] [a, b] a}\]
これはホム集合のメンバーです: 
\[\cat{V}(([b, c] \otimes [a, b]) \otimes a, c)\]
随伴は私たちが探していた合成則を提供します。

同様に、恒等射: 
\[j_a \Colon i \to [a, a]\]
は次のホム集合のメンバーです: 
\[\cat{V}(i, [a, a])\]
これは随伴を通して次に同型です: 
\[\cat{V}(i \otimes a, a)\]
私たちはこのホム集合が左恒等子$\lambda_a$を含むことを知っています。私たちはそれを随伴の下でのイメージとして$j_a$として定義することができます。

自己豊穣化の実用的な例は、プログラミング言語の型のプロトタイプとして機能する$\Set$圏です。以前見たように、$\Set$はデカルト積に関して閉じたモノイダル圏です。$\Set$において、任意の2つの集合間のホム集合はそれ自体が集合なので、$\Set$の対象です。私たちはそれが指数集合に同型であることを知っているので、外部ホムと内部ホムは等価です。今、自己豊穣化を通じて、指数集合をホム対象として使用し、デカルト積の指数対象の組み合わせに関して合成を表現することができることも知っています。

\section{$\cat{2}$-圏との関連性}

私は以前、圏の圏である$\Cat$の文脈で$\cat{2}$-圏について話しました。圏間の射は関手ですが、さらに追加の構造があります: 関手間の自然変換です。$\cat{2}$-圏では、対象はしばしばゼロセル、射は$1$-セル、射間の射は$2$-セルと呼ばれます。$\Cat$では、$0$-セルは圏、$1$-セルは関手、そして$2$-セルは自然変換です。

しかし、2つの圏間の関手もまた圏を形成することに注意してください。したがって、$\Cat$では、私たちは本当にホム集合ではなく\emph{ホム圏}を持っています。実際には、$\Set$が$\Set$上で豊穣化された圏として扱うことができるように、$\Cat$は$\Cat$上で豊穣化された圏として扱うことができます。さらに一般的には、任意の圏が$\Set$上で豊穣化された圏として扱われるように、任意の$\cat{2}$-圏は$\Cat$上で豊穣化されると考えることができます。
