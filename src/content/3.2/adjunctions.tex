% !TEX root = ../../ctfp-print.tex

\lettrine[lhang=0.17]{数}{学}では、あるものが他のものと似ているということを様々な方法で表現します。最も厳密なのは等価性です。二つのものが等しいとは、一方を他方と区別できないことを意味します。一方をどんな想像上の文脈においても他方と置き換えることができます。例えば、可換図式について話すたびに、射の\newterm{等価性}を使ったことに気付きましたか?それは射が集合 (ホム集合) を形成し、集合の要素は等価性で比較できるからです。

しかし、等価性はしばしば強すぎます。実際に等しくないけれども、すべての意図と目的において同じであるものの多くの例があります。例えば、ペア型\code{(Bool, Char)}は厳密には\code{(Char, Bool)}と等しくありませんが、同じ情報を含んでいることがわかります。この概念は、二つの型の間の\newterm{同型}、すなわち可逆な射によって最もよく捉えられます。それが射であるため、構造を保持し、「iso」であることは、どちら側から始めても同じ場所に着地する往復の一部であることを意味します。ペアの場合、この同型は\code{swap}と呼ばれます: 

\src{snippet01}
\code{swap}はたまたま自己逆です。

\section{随伴と単位/余単位の対}

圏が同型であると言うとき、それは圏の間の写像、つまり関手の観点で表現されます。二つの圏$\cat{C}$と$\cat{D}$が同型である場合、$\cat{C}$から$\cat{D}$への関手$R$ (「右」) が存在し、それが可逆であると言えます。言い換えれば、$\cat{D}$から$\cat{C}$に戻る別の関手$L$ (「左」) が存在し、$R$と合成したものが恒等関手$I$と等しくなります。可能な合成は$R \circ L$と$L \circ R$の二つで、可能な恒等関手は$\cat{C}$に一つ、$\cat{D}$にもう一つあります。

\begin{figure}[H]
  \centering
  \includegraphics[width=0.5\textwidth]{images/adj-1.jpg}
\end{figure}

\noindent
しかし、ここで厄介な部分があります: 二つの関手が\emph{等しい}とはどういう意味ですか?以下の等式はどういう意味でしょうか: 
\[R \circ L = I_{\cat{D}}\]
あるいはこれは: 
\[L \circ R = I_{\cat{C}}\]
関手の等価性を対象の等価性の観点で定義するのは合理的でしょう。等しい対象に作用する二つの関手は等しい対象を生成すべきです。しかし、一般には、任意の圏における対象の等価性の概念を持っていません。それは単に定義の一部ではありません。 (「等価性が本当に何であるか」というこの問題の深淵に深く踏み込むと、私たちはホモトピー型理論に行き着くでしょう。) 

関手は圏の圏における射であるため、等価性比較が可能であるべきだと主張するかもしれません。そして確かに、私たちが小さい圏について話している限り、対象が集合を形成するところでは、集合の要素の等価性を使って対象を等価性比較することができます。

しかし、$\Cat$は実際には$\cat{2}$-圏です。$\cat{2}$-圏のホム集合には追加の構造があります―$1$-射間で作用する$2$-射があります。$\Cat$では、$1$-射は関手で、$2$-射は自然変換です。ですから、関手について話すときに等価性の代わりに自然同型を考慮する方が自然 (このダジャレは避けられません!) です。

したがって、圏の同型の代わりに、二つの圏$\cat{C}$と$\cat{D}$が\newterm{同値}であると考える方が一般的です。私たちは二つの圏の間を往復する関手を見つけることができ、その合成 (どちらの方向でも) が恒等関手と\newterm{自然に同型}です。つまり、合成$R \circ L$と恒等関手$I_{\cat{D}}$の間には二方向の自然変換があり、もう一方は$L \circ R$と恒等関手$I_{\cat{C}}$の間にあります。

随伴は同値よりもさらに弱いです。それは二つの関手の合成が恒等関手に\emph{同型}である必要はなく、$I_{\cat{D}}$から$R \circ L$への\newterm{一方向}の自然変換と、$L \circ R$から$I_{\cat{C}}$へのもう一方の存在を要求するだけです。これら二つの自然変換のシグネチャはこうです: 
\begin{gather*}
  \eta \Colon I_{\cat{D}} \to R \circ L \\
  \varepsilon \Colon L \circ R \to I_{\cat{C}}
\end{gather*}
$\eta$は単位と呼ばれ、$\varepsilon$は随伴の余単位です。

これら二つの定義の間の非対称性に注意してください。一般に、私たちは残りの二つの写像を持っていません: 
\begin{gather*}
  R \circ L \to I_{\cat{D}} \quad\quad\text{必ずしもではありません} \\
  I_{\cat{C}} \to L \circ R \quad\quad\text{必ずしもではありません}
\end{gather*}
この非対称性のために、関手$L$は関手$R$の\newterm{左随伴}と呼ばれ、関手$R$は$L$の右随伴です。 (もちろん、左と右はあなたが図を特定の方法で描く場合にのみ意味があります。) 

随伴のコンパクトな表記は: 
\[L \dashv R\]
随伴をより深く理解するために、単位と余単位をもっと詳しく分析しましょう。

\begin{figure}[H]
  \centering
  \includegraphics[width=0.5\textwidth]{images/adj-unit.jpg}
\end{figure}

\noindent
まず単位から始めましょう。それは自然変換なので、射の族です。$\cat{D}$の対象$d$を与えられたとき、$\eta$のコンポーネントは$I d$、つまり$d$自身と$(R \circ L) d$の間の射です。図では、$d'$と呼ばれています: 
\[\eta_d \Colon d \to (R \circ L) d\]
$R \circ L$の合成は$\cat{D}$の自己関手です。

この方程式は、任意の対象$d$を$\cat{D}$で始点として選び、$R \circ L$の往復関手を使って目標対象$d'$を選ぶことができることを教えてくれます。そして私たちは矢印 --- 射$\eta_d$ --- を目標に向けて撃ちます。

\begin{figure}[H]
  \centering
  \includegraphics[width=0.5\textwidth]{images/adj-counit.jpg}
\end{figure}

\noindent
同様に、余単位$\varepsilon$のコンポーネントは次のように記述できます: 
\[\varepsilon_{c} \Colon (L \circ R) c \to c\]
これは、任意の対象$c$を$\cat{C}$で目標として選び、往復関手$L \circ R$を使って出発点$c' = (L \circ R) c$を選ぶことができることを示しています。そして私たちは矢印 --- 射$\varepsilon_{c}$ --- を出発点から目標へ撃ちます。

単位と余単位を見るもう一つの方法は、単位が$\cat{D}$上の恒等関手のどこにでも合成$R \circ L$を\emph{導入}することを可能にし、余単位が$\cat{C}$上の恒等関手と置き換えることで合成$L \circ R$を\emph{排除}することを可能にするということです。これは、導入に続いて排除が何も変えないことを確実にするいくつかの「明白な」整合性条件につながります: 
\begin{gather*}
  L = L \circ I_{\cat{D}} \to L \circ R \circ L \to I_{\cat{C}} \circ L = L \\
  R = I_{\cat{D}} \circ R \to R \circ L \circ R \to R \circ I_{\cat{C}} = R
\end{gather*}
これらは三角恒等式と呼ばれるもので、以下の図式を可換にします: 

\begin{figure}[H]
  \centering

  \begin{subfigure}
    \centering
    \begin{tikzcd}[column sep=large, row sep=large]
      L \arrow[rd, equal] \arrow[r, "L \circ \eta"]
      & L \circ R \circ L \arrow[d, "\epsilon \circ L"] \\
      & L
    \end{tikzcd}
  \end{subfigure}%
  \hspace{1cm}
  \begin{subfigure}
    \centering
    \begin{tikzcd}[column sep=large, row sep=large]
      R \arrow[rd, equal] \arrow[r, "\eta \circ R"]
      & R \circ L \circ R \arrow[d, "R \circ \epsilon"] \\
      & R
    \end{tikzcd}
  \end{subfigure}
\end{figure}

\noindent
これらは関手圏の図式です: 射は自然変換であり、その合成は自然変換の水平合成です。コンポーネントでこれらの恒等式を表現すると、次のようになります: 
\begin{gather*}
  \varepsilon_{L d} \circ L \eta_d = \id_{L d} \\
  R \varepsilon_{c} \circ \eta_{R c} = \id_{R c}
\end{gather*}
単位と余単位はHaskellで異なる名前でよく見かけます。単位は\code{return} (または\code{Applicative}の定義における\code{pure}) として知られています: 

\src{snippet02}
余単位は\code{extract}として知られています: 

\src{snippet03}
ここで、\code{m}は$R \circ L$に対応する(自己)関手で、\code{w}は$L \circ R$に対応するv(自己)関手です。後ほど見るように、これらはモナドと余モナドの定義の一部です。

自己関手をコンテナと考えると、単位 (または\code{return}) は任意の型の値の周りにデフォルトのボックスを作成する多相的関数です。余単位 (または\code{extract}) はその逆で、コンテナから単一の値を取り出すか生成します。

後ほど見るように、随伴のペアごとにモナドと余モナドが定義されます。逆に、すべてのモナドまたは余モナドは随伴のペアに分解することができます―この分解は一意ではありませんが。

Haskellでは、私たちはモナドを頻繁に使用しますが、通常それらを随伴のペアに分解することは稀です。主に、それらの関手が通常私たちを$\Hask$の外に連れて行くからです。

しかし、Haskellでは\newterm{自己関手}の随伴を定義することができます。こちらが\code{Data.Functor.Adjunction}からの定義の一部です: 

\src{snippet04}
この定義はいくつかの説明が必要です。まず第一に、それは複数のパラメータ型クラスを記述しています―二つのパラメータは\code{f}と\code{u}です。それはこれら二つの型コンストラクタ間の関係として\code{Adjunction}を確立します。

垂直バーの後の追加条件は機能依存性を指定します。例えば、\code{f -> u}は\code{u}が\code{f}によって決定されることを意味します (ここでは型コンストラクタ上の関数としての関係です)。逆に、\code{u -> f}は、\code{u}を知っていれば、\code{f}が一意に決定されることを意味します。

\section{随伴とホム集合}

随伴の別の同等の定義は、ホム集合の自然同型の観点からです。この定義は、これまでに検討してきた普遍構成とうまく結びついています。何か特有の射が、何かの構成を因数分解するという声明を聞くたびに、それをある集合からホム集合への写像として考えるべきです。「特有の射を選ぶ」ということの意味です。

さらに、因数分解はしばしば自然変換の観点で記述されます。因数分解には可換図式が関与します―ある射が二つの射 (因数) の合成に等しいです。自然変換は射を可換図式に写像します。したがって、普遍構成では、私たちは射から可換図式へ、そして特有の射へ行きます。私たちは射から射へ、あるいは一つのホム集合から別のホム集合へ (通常は異なる圏内) の写像を終えます。この写像が可逆であり、それがすべてのホム集合に自然に拡張できる場合、私たちは随伴を持っています。

普遍構成と随伴の主な違いは、後者が全てのホム集合に対して大局的に定義されるということです。例えば、普遍構成を使用して、それがその圏の他のどのペアの対象に対しても存在しないかもしれないある選択された対象の積を定義することができます。すぐに見るように、\emph{任意のペア}の対象の積が圏内に存在する場合、それはまた随伴を通じて定義することもできます。

\begin{figure}[H]
  \centering
  \includegraphics[width=0.5\textwidth]{images/adj-homsets.jpg}
\end{figure}

\noindent
ここに、ホム集合を使用して随伴を定義するための別の方法があります。前と同様に、私たちは二つの関手$L \Colon \cat{D} \to \cat{C}$と$R \Colon \cat{C} \to \cat{D}$を持っています。私たちは二つの任意の対象を選びます: $\cat{D}$の出発対象$d$と$\cat{C}$の目標対象$c$です。$L$を使って出発対象$d$を$\cat{C}$に写像することができます。今、私たちは$\cat{C}$内の二つの対象、$L d$と$c$を持っています。彼らはホム集合を定義します: 
\[\cat{C}(L d, c)\]
同様に、$R$を使って目標対象$c$を写像することができます。今、私たちは$\cat{D}$内の二つの対象、$d$と$R c$を持っています。彼らもホム集合を定義します: 
\[\cat{D}(d, R c)\]
私たちは、$L$が$R$の左随伴であると言います、もしホム集合の同型が存在すれば: 
\[\cat{C}(L d, c) \cong \cat{D}(d, R c)\]
これは、$d$と$c$の両方において自然です。自然というのは、出発点$d$を$\cat{D}$を通してスムーズに変化させることができること、および目標点$c$を$\cat{C}$を通して変化させることができることを意味します。より正確には、私たちは以下の二つの (共変) 関手間の自然変換$\varphi$を持っています。これらは$\cat{C}$から$\Set$への関手の対象への作用です: 
\begin{gather*}
  c \to \cat{C}(L d, c) \\
  c \to \cat{D}(d, R c)
\end{gather*}
もう一方の自然変換、$\psi$は、以下の (反変) 関手間で作用します: 
\begin{gather*}
  d \to \cat{C}(L d, c) \\
  d \to \cat{D}(d, R c)
\end{gather*}
両方の自然変換は可逆でなければなりません。

二つの随伴の定義が等価であることを示すのは容易です。例えば、ホム集合の同型から出発して単位変換を導き出しましょう: 
\[\cat{C}(L d, c) \cong \cat{D}(d, R c)\]
この同型が任意の対象$c$に対して機能するので、それは$c = L d$に対しても機能しなければなりません: 
\[\cat{C}(L d, L d) \cong \cat{D}(d, (R \circ L) d)\]
左側には少なくとも一つの射、恒等射が含まれていることがわかります。自然変換はこの射を$\cat{D}(d, (R \circ L) d)$の要素、つまり恒等関手$I$を挿入して: 
\[\cat{D}(I d, (R \circ L) d)\]
の射へ写像します。$d$によってパラメータ化された射の族を得ます。それらは関手$I$と関手$R \circ L$間の自然変換を形成します (自然性条件は容易に検証できます)。これはまさに私たちの単位$\eta$です。

逆に、単位と余単位の存在から出発して、ホム集合間の変換を定義することができます。例えば、$\cat{C}(L d, c)$のホム集合の任意の射$f$を選びましょう。$\varphi$を定義したいと思いますが、それは$f$に作用して$\cat{D}(d, R c)$の射を生成します。

実際に選択肢はあまりありません。一つの試みとして、$f$を$R$を使って引き上げることができます。それは$R (L d)$から$R c$への射$R f$を生成します―$\cat{D}((R \circ L) d, R c)$の射です。

私たちが必要とする$\varphi$のコンポーネントは、$d$から$R c$への射です。それは問題ありません、なぜなら$\eta_d$のコンポーネントを使って$d$から$(R \circ L) d$へ行くことができます。私たちは得ます: 
\[\varphi_f = R f \circ \eta_d\]
逆方向は類推的であり、$\psi$の導出も同様です。

Haskellの\code{Adjunction}の定義に戻ると、自然変換$\varphi$と$\psi$はそれぞれ\code{leftAdjunct}と\code{rightAdjunct}という\code{a}と\code{b}における多相的 (polymorphic) 関数に置き換えられます。関手$L$と$R$は\code{f}と\code{u}と呼ばれます: 

\src{snippet05}
\code{unit}/\code{counit}の定義と\code{leftAdjunct}/\code{rightAdjunct}の定義との間の同等性は、これらの写像によって証明されます: 

\src{snippet06}
圏の記述からHaskellコードへの翻訳を追うことは非常に教育的です。これを練習として強く推奨します。

Haskellにおいて、右随伴が自動的に\hyperref[representable-functors]{表現可能関手}である理由を説明する準備ができました。これは、第一近似として、Haskellの型の圏を集合の圏として扱うことができるためです。

右圏$\cat{D}$が$\Set$の場合、右随伴$R$は$\cat{C}$から$\Set$への関手です。そのような関手が表現可能である場合、私たちは$\cat{C}$内のある対象$\mathit{rep}$を見つけることができ、ホム関手$\cat{C}(\mathit{rep}, \_)$が$R$と自然に同型であると言えます。実際には、もし$R$が$\Set$から$\cat{C}$へのいくつかの関手$L$の右随伴である場合、そのような対象は常に存在します―それは単集合$()$の下での$L$の像です: 
\[\mathit{rep} = L ()\]
確かに、随伴によれば、次の二つのホム集合が自然に同型です: 
\[\cat{C}(L (), c) \cong \Set((), R c)\]
与えられた$c$に対して、右側は単集合$()$から$R c$への関数の集合です。私たちは以前に見たように、各そのような関数が$R c$の集合から一つの要素を選びます。その関数の集合は$R c$の集合と同型です。ですから、私たちは持っています: 
\[\cat{C}(L (), -) \cong R\]
これは$R$が確かに表現可能であることを示しています。

\section{随伴からの積}

私たちは普遍構成を使用していくつかの概念を以前に紹介しました。それらの概念は、大局的に定義されるとき、随伴を使用して表現する方が容易です。最も簡単な非自明な例は積の概念です。\hyperref[products-and-coproducts]{積の普遍構成}の要点は、任意の積候補を普遍的積を通じて因数分解する能力です。

より正確には、二つの対象$a$と$b$の積は対象$(a\times{}b)$ (またはHaskellの記法で\code{(a, b)}) であり、二つの射$\mathit{fst}$と$\mathit{snd}$で装備されており、任意の他の候補$c$が二つの射
$p \Colon c \to a$と$q \Colon c \to b$で装備されている場合、$\mathit{fst}$と$\mathit{snd}$を通じて$p$と$q$を因数分解する一意の射$m \Colon c \to (a, b)$が存在します。

前に見たように、Haskellでは、この射を生成する\code{factorizer}を実装することができます: 

\src{snippet07}
因数分解条件が成り立つことを検証するのは簡単です: 

\src{snippet08}
私たちは、射のペア\code{p}と\code{q}を取り、別の射\code{m = factorizer p q}を生成する写像を持っています。

これを随伴の定義に必要な二つのホム集合間の写像にどのように翻訳するかですか?コツは$\Hask$の外に出て、射のペアを単一の射として扱う積圏を考えることです。

積圏とは何かを思い出させてください。任意の二つの圏$\cat{C}$と$\cat{D}$を取ります。積圏$\cat{C}\times{}\cat{D}$内の対象は、$\cat{C}$から一つと$\cat{D}$から一つ、二つの対象のペアです。射は、$\cat{C}$から一つと$\cat{D}$から一つの射のペアです。

ある圏$\cat{C}$内の積を定義するには、積圏$\cat{C}\times{}\cat{C}$から始めるべきです。$\cat{C}$からの射のペアは、積圏$\cat{C}\times{}\cat{C}$内の単一の射です。

\begin{figure}[H]
  \centering
  \includegraphics[width=0.5\textwidth]{images/adj-productcat.jpg}
\end{figure}

\noindent
最初は少し混乱するかもしれませんが、積を定義するために積圏を使用していることは、これらが非常に異なる積であることを理解してください。積圏を定義するためには普遍構成が必要ありません。対象のペアと射のペアの概念が必要です。

しかし、$\cat{C}$からの対象のペアは$\cat{C}$内の対象では\emph{ありません}。それは別の圏、$\cat{C}\times{}\cat{C}$内の対象です。形式的には、$\cat{C}$の対象$a$と$b$を$\langle a, b \rangle$として書くことができます。一方、普遍構成は、$\cat{C}$内の同じ圏内の対象$a\times{}b$ (またはHaskellでの\code{(a, b)}) を定義するために必要です。この対象は、普遍構成で指定された方法で$\langle a, b \rangle$を代表することになります。それは常に存在するわけではなく、ある対象のペアに対して存在する場合でも、$\cat{C}$内の他のペアに対しては存在しないかもしれません。

では、\code{factorizer}をホム集合の写像として見てみましょう。最初のホム集合は積圏$\cat{C}\times{}\cat{C}$内にあり、二番目のものは$\cat{C}$内にあります。一般的な射は$\cat{C}\times{}\cat{C}$内で射のペア$\langle f, g \rangle$です: 
\begin{gather*}
  f \Colon c' \to a \\
  g \Colon c'' \to b
\end{gather*}
$c''$は$c'$と異なる可能性があります。しかし、積を定義するためには、同じ出発対象$c$を共有する特別な射、$p$と$q$のペアに興味があります。それは大丈夫です: 随伴の定義では、左ホム集合の出発点は任意の対象ではありません―それは右圏からの何らかの対象に作用する左関手$L$の結果です。当てはまる関手は対角関手$\Delta$です。それは$\cat{C}$から$\cat{C}\times{}\cat{C}$への関手で、対象への作用は次のようになります: 
\[\Delta c = \langle c, c \rangle\]
したがって、私たちの随伴の左側のホム集合は次のようになります: 
\[(\cat{C}\times{}\cat{C})(\Delta c, \langle a, b \rangle)\]
それは積圏のホム集合です。その要素は\code{factorizer}への引数として認識される射のペアです: 
\[(c \to a) \to (c \to b) \ldots{}\]
右側のホム集合は$\cat{C}$内にあり、それは出発対象$c$といくつかの関手$R$が$\cat{C}\times{}\cat{C}$内の目標対象に作用した結果との間にあります。それは$\langle a, b \rangle$のペアを私たちの積対象$a\times{}b$に写像する関手です。このホム集合の要素は\code{factorizer}の\emph{結果}として認識されます: 
\[\ldots{} \to (c \to (a, b))\]

\begin{figure}[H]
  \centering
  \includegraphics[width=0.5\textwidth]{images/adj-product.jpg}
\end{figure}

\noindent
私たちはまだ完全な随伴を持っていません。それにはまず、私たちの\code{factorizer}が可逆である必要があります―私たちはホム集合間の\emph{同型}を構成しています。\code{factorizer}の逆は、ある対象$c$から積対象$a\times{}b$への射$m$から始まるべきです。言い換えると、$m$は以下の要素でなければなりません: 
\[\cat{C}(c, a\times{}b)\]
逆の因数分解器は$m$を$\langle c, c \rangle$から$\langle a, b \rangle$への射$\langle p, q \rangle$に写像する必要があります。言い換えると、それは以下の要素の射です: 
\[(\cat{C}\times{}\cat{C})(\Delta\ c, \langle a, b \rangle)\]
その写像が存在する場合、私たちは対角関手の右随伴が存在すると結論付けます。その関手は積を定義します。

Haskellでは、\code{m}をそれぞれ\code{fst}と\code{snd}で合成することによって、常に\code{factorizer}の逆を構成することができます。

\begin{snip}{haskell}
p = fst . m
q = snd . m
\end{snip}
二つの方法で積を定義することの等価性の証明を完了するためには、ホム集合間の写像が$a$、$b$、そして$c$で自然であることを示す必要があります。これは献身的な読者にとっての演習として残します。

要約すると、私たちが行ったこと: 圏の積は、対角関手の\newterm{右随伴}として大局的に定義されるかもしれません: 
\[(\cat{C}\times{}\cat{C})(\Delta c, \langle a, b \rangle) \cong \cat{C}(c, a\times{}b)\]
ここで、$a\times{}b$は私たちの右随伴関手$\mathit{Product}$の作用の結果です。$\langle a, b \rangle$のペアに作用します。任意の関手から$\cat{C}\times{}\cat{C}$へは双関手ですので、$\mathit{Product}$も双関手です。Haskellでは、$\mathit{Product}$双関手は単純に\code{(,)}として書かれます。それを二つの型に適用して、例えばその積型を得ることができます: 

\src{snippet09}

\section{随伴からの指数}

指数$b^a$、または関数対象$a \Rightarrow b$は、\hyperref[function-types]{普遍構成}を使用して定義することができます。この構成が全ての対象のペアに対して存在する場合、それは随伴として見ることができます。再び、コツは以下の声明に集中することです: 

\begin{quote}
  任意の他の対象$z$と射$g \Colon z\times{}a \to b$がある場合、一意の射$h \Colon z \to (a \Rightarrow b)$が存在します。
\end{quote}
この声明はホム集合間の写像を確立します。

この場合、私たちは同じ圏内の対象に取り組んでいるので、二つの随伴関手は自己関手です。左の自己関手$L$は、対象$z$に作用すると$z\times{}a$を生成します。それは、いくつかの固定された$a$との積を取る関手に対応します。

右の自己関手$R$は、$b$に作用すると関数対象$a \Rightarrow b$ (または$b^a$) を生成します。再び、$a$は固定されています。これら二つの関手の間の随伴はしばしば次のように書かれます: 
\[-\times{}a \dashv (-)^a\]
この随伴が基づいているホム集合の写像は、私たちが普遍構成で使用した図を再描画することによって最もよく見ることができます。

\begin{figure}[H]
  \centering
  \includegraphics[width=0.4\textwidth]{images/adj-expo.jpg}
\end{figure}

\noindent
$\mathit{eval}$射\footnote{第9章の\hyperref[function-types]{普遍構成}を参照。}はこの随伴の余単位に他なりません: 
\[(a \Rightarrow b)\times{}a \to b\]
ここで: 
\[(a \Rightarrow b)\times{}a = (L \circ R) b\]
私は以前に述べた通り、普遍構成は一意の対象を定義します、同型までです。だから私たちは「積」と「指数」を持っています。これは随伴にも翻訳されます: もし関手が随伴を持つならば、この随伴は同型までに一意です。

\section{チャレンジ}

\begin{enumerate}
  \tightlist
  \item
    $\psi$の自然性四角形を導き出してください。これは二つの (反変) 関手間の変換です: 
        \begin{gather*}
          a \to \cat{C}(L a, b) \\
          a \to \cat{D}(a, R b)
        \end{gather*}
  \item
    第二の随伴定義のホム集合の同型から余単位$\varepsilon$を導き出してください。
  \item
    二つの随伴定義の等価性の証明を完成させてください。
  \item
    余積が随伴によって定義できることを示してください。余積の因数分解器の定義から始めてください。
  \item
    余積が対角関手の左随伴であることを示してください。
  \item
    Haskellで積と関数対象間の随伴を定義してください。
\end{enumerate}



