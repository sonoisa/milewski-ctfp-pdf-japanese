% !TEX root = ../../ctfp-print.tex

\lettrine[lhang=0.17]{こ}{れまで}に、圏$\cat{C}$の対象$a$を固定した場合、写像$\cat{C}(a, -)$は$\cat{C}$から$\Set$への (共変) 関手であることを見てきました。
\[x \to \cat{C}(a, x)\]
 (終域は$\Set$です、なぜならホム集合$\cat{C}(a, x)$は\emph{集合}だからです。) この写像をホム関手と呼びます。--- 以前に射に対するその作用も定義しました。

さて、この写像において$a$を変化させてみましょう。新しい写像が得られ、それは任意の$a$に対してホム\emph{関手}$\cat{C}(a, -)$を割り当てます。
\[a \to \cat{C}(a, -)\]
これは対象から関手への写像であり、つまり$\cat{C}$の対象から関手圏の\emph{対象}への写像です (関手圏については\hyperref[natural-transformations]{自然変換}の節を参照してください) 。関手圏$\cat{C}$から$\Set$への記法として$[\cat{C}, \Set]$を使いましょう。また、ホム関手は典型的な\hyperref[representable-functors]{表現可能関手}であることを覚えているかもしれません。

二つの圏の間に対象の写像がある場合、その写像が関手であるかどうか、つまり一方の圏の射を他方の圏の射に持ち上げることができるかどうかを自然に考えます。$\cat{C}$の射は$\cat{C}(a, b)$の要素ですが、関手圏$[\cat{C}, \Set]$の射は自然変換です。したがって、私たちは射を自然変換に写す写像を探しています。

射$f \Colon a \to b$に対応する自然変換を見つけられるか見てみましょう。まず、$a$と$b$が何に写されるかを見てみましょう。それらは二つの関手、$\cat{C}(a, -)$と$\cat{C}(b, -)$に写されます。この二つの関手の間の自然変換が必要です。

ここでコツがあります。米田の補題を使います: 
\[[\cat{C}, \Set](\cat{C}(a, -), F) \cong F a\]
そして一般的な$F$をホム関手$\cat{C}(b, -)$に置き換えます。私たちは得ます: 
\[[\cat{C}, \Set](\cat{C}(a, -), \cat{C}(b, -)) \cong \cat{C}(b, a)\]

\noindent
これはまさに私たちが探していた二つのホム関手間の自然変換ですが、少し捻りがあります。自然変換と射---$\cat{C}(b, a)$の要素---との間には「逆」方向への写像があります。しかし、それは問題ありません。これは私たちが見ている関手が反変であることを意味します。

\noindent
実際、私たちは想定以上のものを得ました。$\cat{C}$から$[\cat{C}, \Set]$への写像は反変関手だけでなく、\emph{充満忠実}関手です。充満性と忠実性は、関手がホム集合をどのように写像するかを記述する関手の性質です。

\emph{忠実}関手はホム集合において\newterm{単射}であり、つまりそれは異なる射を異なる射に写像します。言い換えれば、それは射をまとめてしまうことはありません。

\emph{充満}関手はホム集合において\newterm{全射}であり、つまりそれは一つのホム集合を別のホム集合に\emph{上へ}写像し、後者を完全に覆います。

充満忠実関手$F$はホム集合上の\newterm{全単射}です---両方の集合の全ての要素の一対一の対応です。源の圏$\cat{C}$の任意の対象のペア$a$と$b$に対して、$\cat{C}(a, b)$と$\cat{D}(F a, F b)$の間には全単射があります。ここで$\cat{D}$は$F$の対象圏です (この場合、関手圏$[\cat{C}, \Set]$です) 。これは$F$が\emph{対象}上の全単射であるとは言いません。$\cat{D}$には$F$の像にない対象が存在するかもしれませんし、そのような対象のホム集合については何も言えません。

\section{埋め込み}

私たちがちょうど記述した (反変) 関手、つまり$\cat{C}$の対象を$[\cat{C}, \Set]$の関手に写す関手: 
\[a \to \cat{C}(a, -)\]
は\newterm{米田埋め込み}を定義します。それは圏$\cat{C}$ (厳密には反変性のために$\cat{C}^\mathit{op}$) を関手圏$[\cat{C}, \Set]$内に\emph{埋め込む}ものです。それは$\cat{C}$の対象を関手に写すだけでなく、それらの間の全ての接続を忠実に保存します。

これは非常に有用な結果です。なぜなら、数学者は関手の圏、特に終域が$\Set$である関手について多くのことを知っているからです。私たちは任意の圏$\cat{C}$について、それを関手圏に埋め込むことによって多くの洞察を得ることができます。

もちろん、米田埋め込みには双対バージョンがあり、時に余米田埋め込みと呼ばれることもあります。私たちが各ホム集合の対象 (ソース対象ではなく) を固定することから始めることができたことに注意してください、$\cat{C}(-, a)$。それは私たちに反変ホム関手を与えます。$\cat{C}$から$\Set$への反変関手は私たちがよく知る前層 (例えば、\hyperref[limits-and-colimits]{極限と余極限}を参照してください) です。余米田埋め込みは圏$\cat{C}$を前層の圏に埋め込むことを定義します。射に対するその作用は以下の通りです: 
\[[\cat{C}, \Set](\cat{C}(-, a), \cat{C}(-, b)) \cong \cat{C}(a, b)\]
再び、前層の圏について多くのことが知られているので、任意の圏をそこに埋め込むことができるのは大きな利点です。

\section{Haskellへの応用}

Haskellでは、米田埋め込みはreader関手間の自然変換と、それとは逆方向の関数との間の同型として表現できます: 

\begin{snipv}
forall x. (a -> x) -> (b -> x) \ensuremath{\cong} b -> a
\end{snipv}
 (reader関手は\code{((->) a)}と同等です。) 

この恒等式の左辺は、\code{a}から\code{x}への関数と\code{b}の型の値を与えられたときに、\code{x}の型の値を生成することができる多相的関数です (私は関数\code{b -> x}の周りの括弧を落としています---非Curry化しています) 。これがすべての\code{x}に対して行える唯一の方法は、関数が\code{b}を\code{a}に変換する方法を密かに知っている場合です。

そのようなコンバーター、\code{btoa}がある場合、左辺を\code{fromY}と呼び、次のように定義することができます: 

\src{snippet01}
逆に、関数\code{fromY}が与えられれば、identityを呼び出すことによってコンバーターを回復することができます: 

\src{snippet02}
これは型\code{fromY}と\code{btoa}の関数の間の全単射を確立します。

この同型を見る別の方法は、それが\code{b}から\code{a}への関数の\acronym{CPS}エンコーディングであるということです。引数\code{a -> x}は継続 (ハンドラー) です。結果は\code{b}から\code{x}への関数で、\code{b}の型の値が与えられたときに、エンコードされている関数を事前に合成して継続を実行します。

米田埋め込みはまた、特にそれが\code{Control.Lens}ライブラリからのレンズの\urlref{https://bartoszmilewski.com/2015/07/13/from-lenses-to-yoneda-embedding/}{非常に有用な表現}を提供することによって、Haskellのデータ構造の代替表現を説明します。

\section{前順序の例}

この例はRobert Harperによって提案されました。それは前順序によって定義される圏への米田埋め込みの適用です。前順序は$\leqslant$ (以下) として伝統的に書かれる要素間の順序関係を持つ集合です。前順序における「pre」は、関係が推移的で反射的であることが求められるのに対し、必ずしも反対称的でないということを意味します (したがって、サイクルが可能です) 。

前順序関係を持つ集合は圏を生み出します。対象はこの集合の要素です。対象$a$から$b$への射は、対象が比較できないか、$a \leqslant b$ でない場合には存在しないか、または $a \leqslant b$ の場合に存在し、$a$ から $b$ へと向かいます。一つの対象から別の対象への射は決して一つ以上存在しません。従って、このような圏の任意のホム集合は空集合か単集合です。このような圏を\emph{薄い}と呼びます。

$a \leqslant b$ かつ $b \leqslant c$ ならば $a \leqslant c$ であることから、射は合成可能ですし、合成は結合的です。また、各要素は自分自身に等しい (関係の反射性) ので、恒等射も存在します。

前順序圏に余米田埋め込みを適用することができます。特に、射の作用に興味があります。
\[[\cat{C}, \Set](\cat{C}(-, a), \cat{C}(-, b)) \cong \cat{C}(a, b)\]
右側のホム集合は $a \leqslant b$ の場合に限り空でないです — その場合は単集合です。従って、$a \leqslant b$ ならば、左側には一つの自然変換が存在します。そうでない場合は自然変換は存在しません。

前順序におけるホム関手間の自然変換はどのようなものでしょうか?それは集合 $\cat{C}(-, a)$ と $\cat{C}(-, b)$ の間の関数の族です。前順序では、これらの各集合は空か単集合です。利用可能な関数の種類を見てみましょう。

空集合から自身への関数 (空集合に対する恒等関数) 、空集合から単集合への関数 (absurd、それは定義される必要があるが空集合の要素はないので何もしません) 、単集合から自身への関数 (単集合に対する恒等関数) があります。唯一許されていない組み合わせは、単集合から空集合への写像です (そのような関数が単一の要素に対してどのような値を取るか?) 。

したがって、自然変換は決して単ホム集合を空ホム集合に接続しません。言い換えれば、もし $x \leqslant a$  (単ホム集合 $\cat{C}(x, a)$) ならば、$\cat{C}(x, b)$ は空であってはなりません。非空の $\cat{C}(x, b)$ は $x \leqslant b$ であることを意味します。従って、問題の自然変換の存在は、すべての $x$ に対して、$x \leqslant a$ ならば $x \leqslant b$ が必要です。
\[\text{すべての } x に対して, x \leqslant a \Rightarrow x \leqslant b\]
一方、余米田はこの自然変換の存在が $\cat{C}(a, b)$ が空でないこと、つまり $a \leqslant b$ であることと同等であると述べています。これらを合わせると、以下のことが得られます: 
\[a \leqslant b \text{ は } すべての x に対して, x \leqslant a \Rightarrow x \leqslant b \text{ と同等です}\]
この結果には直接到達することもできますが、米田埋め込みを通じてこの結果に到達することは、はるかに興奮します。

\section{自然性}

米田の補題は、自然変換の集合と$\Set$内の対象との間の同型を確立します。自然変換は関手圏$[\cat{C}, \Set]$内の射です。任意の二つの関手間の自然変換の集合は、その圏のホム集合です。米田の補題は次の同型です: 
\[[\cat{C}, \Set](\cat{C}(a, -), F) \cong F a\]
この同型は、$F$と$a$の両方について自然です。つまり、$[\cat{C}, \Set] \times \cat{C}$という積圏からのペア$(F, a)$について自然です。ここで、$F$を関手圏の\newterm{対象}として扱っていることに注目してください。

これが意味することを少し考えてみましょう。自然同型は、二つの関手間の可逆な\emph{自然変換}です。そして確かに、私たちの同型の右側は関手です。それは$[\cat{C}, \Set] \times \cat{C}$から$\Set$への関手で、ペア$(F, a)$に対する作用は集合---関手$F$が対象$a$で評価された結果です。これは評価関手と呼ばれます。

左側もまた関手で、$(F, a)$を自然変換の集合$[\cat{C}, \Set](\cat{C}(a, -), F)$に写像します。

これらが本当に関手であることを示すためには、射に対するそれらの作用も定義する必要があります。しかし、ペア$(F, a)$と$(G, b)$の間の射とは何でしょうか?それは射のペア、$(\Phi, f)$です。一つ目は関手間の射---自然変換、二つ目は$\cat{C}$内の通常の射です。

評価関手はこのペア$(\Phi, f)$を取り、二つの集合$F a$と$G b$の間の関数に写像します。私たちは$\Phi$の$a$でのコンポーネント ($F a$から$G a$への写像) と、$G$によって持ち上げられた射$f$から、容易にそのような関数を構成することができます: 
\[(G f) \circ \Phi_a\]
$\Phi$の自然性により、これは以下と同じです: 
\[\Phi_b \circ (F f)\]
私はここでは同型の全体の自然性を証明しませんが、関手と自然変換から構成されているので、それが誤る余地はありません。

\section{チャレンジ}

\begin{enumerate}
  \tightlist
  \item
        Haskellで余米田埋め込みを表現せよ。
  \item
        \code{fromY} と \code{btoa} の間に確立した全単射が同型であることを示せ (二つの写像は互いに逆である) 。
  \item
        モノイドの米田埋め込みを詳しく調べよ。モノイドの単一の対象に対応する関手は何か?モノイドの射に対応する自然変換は何か?
  \item
        前順序に対する\emph{共変}米田埋め込みの応用は何か? (Gershom Bazermanによる提案) 
  \item
        米田埋め込みを使って、任意の関手圏$[\cat{C}, \cat{D}]$を関手圏$[[\cat{C}, \cat{D}], \Set]$に埋め込む方法を考えよ。この場合、射 (つまり自然変換) に対する作用を図り出せ。
\end{enumerate}
