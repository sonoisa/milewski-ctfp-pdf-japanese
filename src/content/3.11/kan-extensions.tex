% !TEX root = ../../ctfp-print.tex

\lettrine[lhang=0.17]{こ}{れまで}のところ、私たちは主に単一の圏や一組の圏と一緒に作業をしてきました。場合によっては、これが少し制約となることもありました。

例えば、圏 $\cat{C}$ で極限を定義する際には、私たちはその錐の形成の基礎となるパターンのテンプレートとして索引圏 $\cat{I}$ を導入しました。錐の頂点のテンプレートとして別の圏、たとえば単純なものを導入することも理にかなっていました。代わりに、私たちは $\cat{I}$ から $\cat{C}$ への定数関手 $\Delta_c$ を使いました。

この不格好さを修正する時が来ました。3つの圏を使って極限を定義しましょう。まず、索引圏 $\cat{I}$ から圏 $\cat{C}$ への関手 $D$ から始めます。これは錐の基盤となる図式を選ぶ関手、つまり図式関手です。

\begin{figure}[H]
  \centering
  \includegraphics[width=0.4\textwidth]{images/kan2.jpg}
\end{figure}

\noindent
新しい追加は、単一の対象 (そして単一の恒等射) を含む圏 $\cat{1}$ です。$\cat{I}$ からこの圏への可能な関手 $K$ は一つだけです。それはすべての対象を $\cat{1}$ の唯一の対象にマッピングし、すべての射を恒等射にマッピングします。任意の関手 $F$ から $\cat{1}$ への $\cat{C}$ は私たちの錐のための潜在的な頂点を選びます。

\begin{figure}[H]
  \centering
  \includegraphics[width=0.4\textwidth]{images/kan15.jpg}
\end{figure}

\noindent
錐は、$F \circ K$ から $D$ への自然変換 $\varepsilon$ です。$F \circ K$ は私たちの元の $\Delta_c$ と全く同じことをします。次の図はこの変換を示しています。

\begin{figure}[H]
  \centering
  \includegraphics[width=0.4\textwidth]{images/kan3-e1492120491591.jpg}
\end{figure}

\noindent
今、私たちは「最良」な関手 $F$ を選ぶ普遍的な性質を定義できます。この $F$ は $\cat{1}$ を $D$ の極限である対象にマッピングし、$F \circ K$ から $D$ への自然変換 $\varepsilon$ は対応する射影を提供します。この普遍的な関手は、$D$ に沿った $K$ の右Kan拡張と呼ばれ、$\Ran_{K}D$ と表記されます。

普遍的な性質を定式化しましょう。別の錐、つまり別の関手 $F'$ と、$F' \circ K$ から $D$ への自然変換 $\varepsilon'$ を持っていると仮定します。

\begin{figure}[H]
  \centering
  \includegraphics[width=0.4\textwidth]{images/kan31-e1492120512209.jpg}
\end{figure}

\noindent
もしKan拡張 $F = \Ran_{K}D$ が存在するならば、$F'$ からそれに向けての唯一の自然変換 $\sigma$ が存在し、$\varepsilon'$ は $\varepsilon$ を通じて因数分解されることが求められます。つまり:
\[\varepsilon' = \varepsilon\ .\ (\sigma \circ K)\]
ここで、$\sigma \circ K$ は2つの自然変換 (そのうちの1つは $K$ 上の恒等自然変換) の水平合成です。この変換はその後 $\varepsilon$ と垂直に合成されます。

\begin{figure}[H]
  \centering
  \includegraphics[width=0.4\textwidth]{images/kan5.jpg}
\end{figure}

\noindent
コンポーネントでは、$\cat{I}$ の対象 $i$ に作用するとき、私たちは次の式を得ます:
\[\varepsilon'_i = \varepsilon_i \circ \sigma_{K i}\]
私たちの場合、$\sigma$ は $\cat{1}$ の単一の対象に対応する唯一のコンポーネントを持っています。だから、これは実際には $F'$ によって定義された錐の頂点から $\Ran_{K}D$ によって定義された普遍錐の頂点への唯一の射です。交換条件は極限の定義によって要求されるものです。

しかし、重要なのは、私たちは単純な圏 $\cat{1}$ を任意の圏 $\cat{A}$ に置き換える自由があり、右Kan拡張の定義は有効のままであるということです。

\section{右Kan拡張}

索引圏 $\cat{I}$ から圏 $\cat{C}$ への関手 $D \Colon \cat{I} \to \cat{C}$ に沿った関手 $K \Colon \cat{I} \to \cat{A}$ の右Kan拡張は、関手 $F \Colon \cat{A} \to \cat{C}$  ($\Ran_{K}D$ と表記されます) と自然変換
\[\varepsilon \Colon F \circ K \to D\]
の組み合わせです。その他の任意の関手 $F' \Colon \cat{A} \to \cat{C}$ と
自然変換
\[\varepsilon' \Colon F' \circ K \to D\]
が与えられた場合には、$\varepsilon'$ を因数分解する唯一の自然変換
\[\sigma \Colon F' \to F\]
が存在します:
\[\varepsilon' = \varepsilon\ .\ (\sigma \circ K)\]
これはかなり難解ですが、次の素敵な図で視覚化できます:

\begin{figure}[H]
  \centering
  \includegraphics[width=0.4\textwidth]{images/kan7.jpg}
\end{figure}

\noindent
これを見ると、ある意味で、Kan拡張は「関手の乗算の逆」として作用することに気づけます。一部の著者は、$\Ran_{K}D$ のために $D/K$ という表記を使います。確かに、この表記では、$\varepsilon$ の定義も、右Kan拡張の余単位とも呼ばれるものも、単純な相殺のように見えます:
\[\varepsilon \Colon D/K \circ K \to D\]
Kan拡張のもう一つの解釈は、関手 $K$ が圏 $\cat{I}$ を $\cat{A}$ の内部に埋め込むことを考慮するものです。最も単純なケースでは、$\cat{I}$ は $\cat{A}$ の部分圏かもしれません。私たちは $\cat{I}$ を $\cat{C}$ にマッピングする関手 $D$ を持っています。$D$ を $\cat{A}$ 全体に定義された関手 $F$ に拡張できますか?理想的には、そのような拡張は $F \circ K$ を $D$ と同型にするでしょう。言い換えれば、$F$ は $D$ の始域を $\cat{A}$ に拡張するでしょう。しかし、完全な同型は通常求められるものとしては多すぎるので、私たちはただその半分、つまり $F \circ K$ から $D$ への片方向の自然変換 $\varepsilon$ で満足できます。 (左Kan拡張は別の方向を選びます。) 

\begin{figure}[H]
  \centering
  \includegraphics[width=0.4\textwidth]{images/kan6.jpg}
\end{figure}

\noindent
もちろん、関手 $K$ が対象に単射でない場合や射集合に忠実でない場合は、埋め込みの絵が崩れます。その場合、Kan拡張は失われた情報を最善を尽くして外挿しようとします。

\section{Kan拡張と随伴}

今、任意の $D$  (固定された $K$ と共に) に対して右Kan拡張が存在すると仮定します。その場合、$\Ran_{K}-$ (ダッシュが $D$ を置き換える) は、関手圏 ${[}\cat{I}, \cat{C}{]}$ から関手圏 ${[}\cat{A}, \cat{C}{]}$ への関手です。実は、この関手は前合成関手 $- \circ K$ の右随伴です。後者は ${[}\cat{A}, \cat{C}{]}$ の関手を ${[}\cat{I}, \cat{C}{]}$ の関手にマッピングします。随伴は以下のようになります:
\[[\cat{I}, \cat{C}](F' \circ K, D) \cong [\cat{A}, \cat{C}](F', \Ran_{K}D)\]
これは私たちが $\varepsilon'$ と呼んだすべての自然変換に対応する唯一の自然変換が $\sigma$ であるという事実の単なる再表現です。

\begin{figure}[H]
  \centering
  \includegraphics[width=0.4\textwidth]{images/kan92.jpg}
\end{figure}

\noindent
さらに、私たちが圏 $\cat{I}$ を $\cat{C}$ と同じものと選んだ場合、$D$ の代わりに恒等関手 $I_{\cat{C}}$ を置き換えることができます。そうすると、次の等式が得られます:
\[[\cat{C}, \cat{C}](F' \circ K, I_{\cat{C}}) \cong [\cat{A}, \cat{C}](F', \Ran_{K}I_{\cat{C}})\]
今度は、$F'$ を $\Ran_{K}I_{\cat{C}}$ と同じものとして選べば、右側に恒等自然変換が含まれ、それに対応して左側には次の自然変換が得られます:
\[\varepsilon \Colon \Ran_{K}I_{\cat{C}} \circ K \to I_{\cat{C}}\]
これは随伴の余単位とよく似ています:
\[\Ran_{K}I_{\cat{C}} \dashv K\]
実際に、関手 $K$ に沿って恒等関手の右Kan拡張を使用すると、$K$ の左随伴を計算するために使用できます。そのためには、もう一つの条件が必要です: 右Kan拡張は関手 $K$ によって保存されなければなりません。拡張の保存とは、$K$ で事前合成された関手のKan拡張を計算した場合、それが $K$ で事前合成された元のKan拡張と同じ結果になることを意味します。私たちの場合、この条件は次のように単純化されます:
\[K \circ \Ran_{K}I_{\cat{C}} \cong \Ran_{K}K\]
注意すべきは、$K$ による除算の表記を使用すると、随伴は次のように書けるということです:
\[I/K \dashv K\]
これは、随伴が何らかの逆を記述することを確認してくれます。保存条件は次のようになります:
\[K \circ I/K \cong K/K\]
関手に沿った関手自体の右Kan拡張、$K/K$、は密度モナドと呼ばれます。

随伴の公式は重要な結果であり、私たちはすぐに見るように、エンド (余エンド) を使用してKan拡張を計算できるため、実用的な方法を提供してくれます。

\section{左Kan拡張}

左Kan拡張は、右Kan拡張の双対構造です。いくつかの直感を得るために、まず単一の圏 $\cat{1}$ を使用して余極限の定義を構造化してみましょう。私たちは、関手 $D \Colon \cat{I} \to \cat{C}$ を使用してその基盤を形成し、関手 $F \Colon \cat{1} \to \cat{C}$ を使用してその頂点を選択します。

\begin{figure}[H]
  \centering
  \includegraphics[width=0.4\textwidth]{images/kan81.jpg}
\end{figure}

\noindent
余錐の側面、つまり射入れは、$D$ から $F \circ K$ への自然変換 $\eta$ のコンポーネントです。

\begin{figure}[H]
  \centering
  \includegraphics[width=0.4\textwidth]{images/kan10a.jpg}
\end{figure}

\noindent
余極限は普遍余錐です。つまり、他の任意の関手 $F'$ と自然変換
\[\eta' \Colon D \to F' \circ K\]

\begin{figure}[H]
  \centering
  \includegraphics[width=0.4\textwidth]{images/kan10b.jpg}
\end{figure}

\noindent
に対して、$F$ から $F'$ への唯一の自然変換 $\sigma$ が存在します。

\begin{figure}[H]
  \centering
  \includegraphics[width=0.4\textwidth]{images/kan14.jpg}
\end{figure}

\noindent
そのようにして、次の図で示されるようになります:

\begin{figure}[H]
  \centering
  \includegraphics[width=0.4\textwidth]{images/kan112.jpg}
\end{figure}

\noindent
単一の圏 $\cat{1}$ を $\cat{A}$ に置き換えると、この定義は自然に左Kan拡張の定義に一般化されます。これは $\Lan_{K}D$ と表記されます。

\begin{figure}[H]
  \centering
  \includegraphics[width=0.4\textwidth]{images/kan12.jpg}
\end{figure}

\noindent
自然変換:
\[\eta \Colon D \to \Lan_{K}D \circ K\]
は左Kan拡張の単位と呼ばれます。

以前のように、自然変換間の一対一の対応を随伴の観点から述べ直すことができます:
\[[\cat{A}, \cat{C}](\Lan_{K}D, F') \cong [\cat{I}, \cat{C}](D, F' \circ K)\]
言い換えると、左Kan拡張は前合成と $K$ の左随伴であり、右Kan拡張は右随伴です。

恒等関手の右Kan拡張を使用して $K$ の左随伴を計算できるのと同様に、恒等関手の左Kan拡張は $K$ の右随伴になります ($\eta$ は随伴の単位です) :
\[K \dashv \Lan_{K}I_{\cat{C}}\]
この二つの結果を組み合わせると、次が得られます:
\[\Ran_{K}I_{\cat{C}} \dashv K \dashv \Lan_{K}I_{\cat{C}}\]

\section{エンドとしてのKan拡張}

Kan拡張の真の力は、エンドと余エンドを使用して計算できるという事実から来ています。単純化のために、私たちはターゲットの圏 $\cat{C}$ を $\Set$ として考えますが、公式は任意の圏に拡張することができます。

Kan拡張を使用して関手の作用をその元の始域の外側に拡張するというアイデアを再考してみましょう。$K$ が $\cat{I}$ を $\cat{A}$ の内部に埋め込むと仮定します。関手 $D$ は $\cat{I}$ を $\Set$ にマッピングします。$K$ の像内の任意の対象 $a$ 、つまり $a = K i$ に対して、拡張された関手は $a$ を $D i$ にマッピングすると単に言えばよいでしょう。しかし、$K$ の像の外側にある $\cat{A}$ の対象はどうでしょうか?各対象は潜在的に$K$の像のすべての対象に多くの射を介して接続されています。関手はこれらの射を保存しなければなりません。対象 $a$ から $K$ の像への射の全体は、ホム関手によって特徴づけられます:
\[\cat{A}(a, K -)\]

\begin{figure}[H]
  \centering
  \includegraphics[width=0.4\textwidth]{images/kan13.jpg}
\end{figure}

\noindent
このホム関手は2つの関手の合成です:
\[\cat{A}(a, K -) = \cat{A}(a, -) \circ K\]
右Kan拡張は関手合成の右随伴です:
\[[\cat{I}, \Set](F' \circ K, D) \cong [\cat{A}, \Set](F', \Ran_{K}D)\]
ホム関手を $F'$ に置き換えるとどうなるか見てみましょう:
\[[\cat{I}, \Set](\cat{A}(a, -) \circ K, D) \cong [\cat{A}, \Set](\cat{A}(a, -), \Ran_{K}D)\]
そして、合成をインライン展開します:
\[[\cat{I}, \Set](\cat{A}(a, K -), D) \cong [\cat{A}, \Set](\cat{A}(a, -), \Ran_{K}D)\]
右側は米田の補題を使用して簡略化できます:
\[[\cat{I}, \Set](\cat{A}(a, K -), D) \cong \Ran_{K}D a\]
これで、右Kan拡張の非常に便利な式を自然変換の集合としてエンドを書き換えることができます:
\[\Ran_{K}D a \cong \int_i \Set(\cat{A}(a, K i), D i)\]
左Kan拡張に対しても、余エンドを用いた類似の公式があります:
\[\Lan_{K}D a = \int^i \cat{A}(K i, a)\times{}D i\]
これが実際に関手合成の左随伴であることを示すために、左側の式を代入して見ましょう:
\[[\cat{A}, \Set](\Lan_{K}D, F') \cong [\cat{I}, \Set](D, F' \circ K)\]
左側を置換して:
\[[\cat{A}, \Set](\int^i \cat{A}(K i, -)\times{}D i, F')\]
これは自然変換の集合なので、エンドとして書き直せます:
\[\int_a \Set(\int^i \cat{A}(K i, a)\times{}D i, F' a)\]
ホム関手の連続性を使用して、余エンドをエンドと置き換えます:
\[\int_a \int_i \Set(\cat{A}(K i, a)\times{}D i, F' a)\]
積-指数随伴を使用します:
\[\int_a \int_i \Set(\cat{A}(K i, a),\ (F' a)^{D i})\]
指数は対応するホム集合と同型です:
\[\int_a \int_i \Set(\cat{A}(K i, a),\ \cat{A}(D i, F' a))\]
Fubiniの定理を使用して、2つのエンドを交換できます:
\[\int_i \int_a \Set(\cat{A}(K i, a),\ \cat{A}(D i, F' a))\]
内側のエンドは2つの関手間の自然変換の集合を表しているので、米田の補題を使用できます:
\[\int_i \cat{A}(D i, F' (K i))\]
これは、随伴の証明をしようとした右側の自然変換の集合です:
\[[\cat{I}, \Set](D, F' \circ K)\]
エンド、余エンド、米田の補題を使用するこの種の計算は、「エンドの計算」の典型的なものです。

\section{HaskellでのKan拡張}

Kan拡張のエンド/余エンドの式は、Haskellに簡単に翻訳できます。右拡張から始めましょう:
\[\Ran_{K}D a \cong \int_i \Set(\cat{A}(a, K i), D i)\]
この定義を見ると、\code{Ran}は関数が適用される型 \code{a} の値を含むべきであり、関手 \code{k} と \code{d} の間の自然変換も必要です。例えば、\code{k} が木関手で、\code{d} がリスト関手であるとし、\code{Ran Tree {[}{]} String} を与えられた場合、関数を渡すと:

\src{snippet02}
\code{Int} のリストが返されます。右Kan拡張はあなたの関数を使って木を生成し、それをリストに再梱包します。例えば、文字列からパース木を生成するパーサーを渡すと、その木の深さ優先探索に対応するリストが得られます。

右Kan拡張は、\code{d} を恒等関手に置き換えることで、与えられた関手の左随伴を計算するために使用することができます。これにより、関手 \code{k} の左随伴が、次の型の多相関数の集合で表されます:

\src{snippet03}
例えば、\code{k} がモノイドの圏から忘却関手であると仮定します。すると、全称量化子はすべてのモノイドを渡ります。もちろん、Haskellではモノイド則を表現することはできませんが、以下は結果として得られる自由関手 (忘却関手 \code{k} は対象に対して恒等的です) のまともな近似です:

\src{snippet04}
予想通り、これは自由モノイド、つまりHaskellのリストを生成します:

\src{snippet05}
左Kan拡張は余エンドです:
\[\Lan_{K}D a = \int^i \cat{A}(K i, a)\times{}D i\]
したがって、存在量化子として表されます。記号的には:

\begin{snip}{haskell}
Lan k d a = exists i. (k i -> a, d i)
\end{snip}
これは、\acronym{GADT}を使用するか、または全称量化子を持つデータコンストラクタを使用して、Haskellでエンコードできます:

\src{snippet06}
このデータ構造の解釈は、それが未指定の \code{i} のコンテナを取り、\code{a} を生成する関数を含んでいるというものです。また、それらの \code{i} のコンテナも持っています。どのような \code{i} であるかわからないので、このデータ構造でできる唯一のことは、\code{i} のコンテナを取り出し、自然変換を使用して関手 \code{k} で定義されたコンテナに再梱包し、関数を呼び出して \code{a} を得ることです。例えば、\code{d} が木であり、\code{k} がリストであれば、木をシリアライズし、結果のリストで関数を呼び出し、\code{a} を得ることができます。

左Kan拡張は、関手の右随伴を計算するために使用することができます。積関手の右随伴が指数関数であることがわかっているので、Kan拡張を使用して実装してみましょう:

\src{snippet07}
これは関数型と同型であることが、以下の一対の関数によって証明されます:

\src{snippet08}
以前に一般的なケースで説明したように、私たちは次のステップを実行しました:

\begin{enumerate}
  \tightlist
  \item
        \code{x} のコンテナを取り出し (ここでは、単なる自明な恒等コンテナ)、関数 \code{f} を取り出しました。
  \item
        自然変換を使用してコンテナを恒等関手からペア関手へと再梱包しました。
  \item
        関数 \code{f} を呼び出しました。
\end{enumerate}

\section{自由関手}

Kan拡張の興味深い応用の一つは自由関手の構成です。これは、次のような実用的な問題の解決策です: 型コンストラクタ、つまり対象のマッピングを持っている場合、この型コンストラクタに基づいて関手を定義することは可能でしょうか?言い換えれば、この型コンストラクタを完全な自己関手に拡張するための射のマッピングを定義できますか?

重要な観察は、型コンストラクタが離散圏の関手として記述できるということです。離散圏は恒等射以外の射を持たない圏です。与えられた圏 $\cat{C}$ に対して、私たちはすべての非恒等射を単純に捨てることによって離散圏 $\cat{|C|}$ を常に構成できます。$\cat{|C|}$ から $\cat{C}$ への関手 $F$ は、その後、対象の単純なマッピング、つまりHaskellで言うところの型コンストラクタです。また、$\cat{|C|}$ を $\cat{C}$ に注入する標準的な関手 $J$ もあります。それは対象に対して (そして恒等射に対しても) 恒等的です。$F$ の $J$ に沿った左Kan拡張が存在する場合、それは $\cat{C}$ から $\cat{C}$ への関手です:
\[\Lan_{J}F a = \int^i \cat{C}(J i, a)\times{}F i\]
これは、$F$ に基づく自由関手と呼ばれます。

Haskellでは、それを次のように書くでしょう:

\src{snippet09}
実際には、任意の型コンストラクタ \code{f} に対して、\code{FreeF f} は関手です:

\src{snippet10}
ご覧のとおり、自由関手は関数の持ち上げを、その関数とその引数の両方を記録することによって偽装します。それは関数の合成を記録することによって持ち上げられた関数を蓄積します。関手の規則は自動的に満たされます。この構成は、論文 \urlref{http://okmij.org/ftp/Haskell/extensible/more.pdf}{Freer Monads, More Extensible Effects} で使用されました。

代わりに、同じ目的のために右Kan拡張を使用することもできます:

\src{snippet11}
これが実際に関手であることは簡単に確認できます:

\src{snippet12}



