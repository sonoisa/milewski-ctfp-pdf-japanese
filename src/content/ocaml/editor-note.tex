% !TEX root = ctfp-print.tex

\lettrine[lhang=0.17]{こ}{れは}、\emph{プログラマのための圏論}のOCaml版です。
これは大成功を収め、Bartosz Milewskiのブログ投稿シリーズを、きれいにタイプセットされた\acronym{PDF}やハードカバーの本として利用可能にしました。この本を改善するために、誤字や誤りの修正、コードスニペットを他のプログラミング言語に翻訳するなど、数多くの貢献がありました。

私は、元のHaskellコードに続いてそのOCaml対応を含む、この版の本を提供することにわくわくしています。OCamlのコードスニペットは、\urlref{https://github.com/ArulselvanMadhavan/ocaml-ctfp}{ocaml-ctfp}の貢献者によって寛大に提供され、この本の形式に合うようにわずかに修正されました。

複数言語のコードスニペットをサポートするために、外部ファイルからコードスニペットをロードする\LaTeX{}マクロを使用しています。これにより、他の言語で本を簡単に拡張しながら、元のテキストをそのままにしておくことができます。このため、テキストで「Haskellで」というテキストを見るたびに心の中で「そしてOCamlで」と言葉を付け足すべきです。

コードは以下のようにレウアウトされています: 元のHaskellコードに続いてOCamlコードがあります。それらを区別するために、コードスニペットはそれぞれの言語のロゴ \raisebox{-.2mm}{\includegraphics[height=.3cm]{fig/icons/haskell.png}} および \raisebox{-.2mm}{\includegraphics[height=.3cm]{fig/icons/ocaml.png}} の主要色を使用した縦棒で左側が囲まれています。例えば:

\srcsnippet{content/1.1/code/haskell/snippet03.hs}{blue}{haskell}
\unskip
\srcsnippet{content/1.1/code/ocaml/snippet03.ml}{orange}{ocaml}
\NoIndentAfterThis
