% !TEX root = ../../ctfp-print.tex

\lettrine[lhang=0.17]{圏}{は}驚くほど単純な概念です。圏 (category) は\newterm{対象 (object)} とそれらの間を結ぶ\newterm{矢印 (arrow)} で構成されます。そのため圏はとても図示しやすいです。対象は円や点として描かれ、矢印は\ldots{}矢印です。(たまに変わり種として、対象を豚として、矢印を花火として描くこともあります。) しかし、圏の本質は\emph{合成 (composition)}です。合成の本質が圏であると言ってもいいです。矢印は合成できるので、対象$A$から$B$への矢印があり、さらに$B$から$C$への矢印があるならば --- それらの合成である --- $A$から$C$への矢印が存在しなければなりません。

\begin{figure}
  \centering
  \includegraphics[width=0.8\textwidth]{images/img_1330.jpg}
  \caption{圏では、$A$から$B$への矢印と、$B$から$C$への矢印があるならば、それらの合成である$A$から$C$への直接の矢印も存在しなければなりません。この図には、恒等射 (後述) が欠けているため完全な圏ではありません。}
\end{figure}

\section{関数としての射}

抽象的すぎてナンセンスに感じますか?落胆しないでください。具体例で考えましょう。矢印、つまり\newterm{射 (morphism)} を関数と考えてみましょう。型$A$の引数を取り、$B$を返す関数$f$があります。そして、$B$を取り、$C$を返す別の関数$g$があります。それらを、$f$の結果を$g$に渡すことで合成することができます。これにより、$A$を取り、$C$を返す新しい関数が得られます。

数学では、このような合成は関数の間に小さな円を置いて表されます: $g \circ f$。合成の順番が右から左であることに注意してください。この順序が混乱の原因となることもあります。Unixのパイプ記法
\begin{snip}{text}
lsof | grep Chrome
\end{snip}
もしくは、F\#のchevron記法\code{>>}はご存知でしょうか。これらはどちらも左から右に処理が進みます。
しかし、数学やHaskellでは関数は右から左に合成されます。$g \circ f$を「f\emph{の後に}g」と読むと理解しやすいかもしれません。

さらに具体的にC言語のコードを書いてみましょう。型\code{A}の引数を取り、型\code{B}の値を返す関数\code{f}があるとします:

\begin{snip}{text}
B f(A a);
\end{snip}
他にももう一つ:

\begin{snip}{text}
C g(B b);
\end{snip}
これらの合成は次のようになります:

\begin{snip}{text}
C g_after_f(A a)
{
    return g(f(a));
}
\end{snip}
ここでも再び、\code{g(f(a))} という右から左への合成が見られます。今回はC言語でです。

残念ながら、C++標準ライブラリには二つの関数を取ってその合成を返すテンプレートがありません。そのため、少しHaskellで試してみましょう。AからBへの関数の宣言は以下の通りです:

\src{snippet01}
同様に:

\src{snippet02}
これらの合成は:

\src{snippet03}
Haskellのシンプルさを知ると、C++で端的に関数概念を表現できないことに少し恥ずかしくなるでしょう。実際、HaskellではUnicode文字を使用して合成を書くことができます:
% don't 'mathify' this block
\begin{snip}{text}
g ◦ f
\end{snip}

さらに、Unicodeの二重コロンと矢印も使用できます:
% don't 'mathify' this block
\begin{snipv}
f \ensuremath{\Colon} A → B
\end{snipv}
ここで最初のHaskellのレッスンです。二重コロンは「\ldots{}の型を持つ」という意味です。関数型は二つの型の間に矢印を挿入することで作成されます。二つの関数を合成するには、それらの間にピリオド (またはUnicodeの円) を挿入します。

\section{合成の性質}

任意の圏において合成が満たさなければならない二つの非常に重要な性質があります。

\begin{enumerate}
  \item
        合成が結合的であること。三つの射、$f$、$g$、および$h$が合成できる (つまり、それらの対象が初めから終わりまで一致する) 場合、それらを合成するために括弧は不要ということ。数学的な記法では、これは以下のように表されます:
        \[h \circ (g \circ f) = (h \circ g) \circ f = h \circ g \circ f\]
        (擬似) Haskellでは以下のようになります:

        \src{snippet04}[b]
        (「擬似」と言ったのは、関数に対する等価性は定義されていないからです。)

        関数を扱う場合、結合則はかなり明白ですが、他の圏ではそうとは限りません。

  \item
        任意の対象$A$に対して、合成の単位となる射が存在すること。この射は対象自体をループします。合成の単位であるとは、$A$から始まるか、または$A$で終わるいかなる射と合成しても、同じ射を返すということです。対象Aの単位射は$\idarrow[A]$ ($A$上の\newterm{恒等射 (identity)}) と呼ばれます。数学的な記法では、$f$が$A$から$B$へ行く場合:
        \[f \circ \idarrow[A] = f\]
        そして
        \[\idarrow[B] \circ f = f\]
\end{enumerate}
関数を扱う場合、恒等射はその引数をそのまま返す恒等関数として実装されます。この実装はすべての型に対して同じであり、これはこの関数が全称多相的であることを意味します。C++ではテンプレートとして定義することができます:

\begin{snip}{cpp}
template<class T> T id(T x) { return x; }
\end{snip}
もちろん、C++では単純にはいかず、何を渡すかだけでなく、どのように渡すかも考慮に入れなければなりません (つまり、値として、参照として、const参照として、ムーブとしてなど)。

Haskellでは、恒等関数は標準ライブラリ (Preludeと呼ばれる) の一部です。これがその宣言と定義です:

\src{snippet05}
ご覧の通り、Haskellでは多相的関数は簡単です。宣言において、型を型変数に置き換えるだけです。ポイントは、具体的な型の名前は常に大文字で始まり、型変数の名前は小文字で始まることです。ここでは\code{a}はすべての型を表しています。

Haskellの関数定義は関数名に続けて形式パラメータが続きます --- ここではただ一つ、\code{x}です。関数の本体は等号に続きます。この簡潔さは初心者にとってしばしば衝撃ですが、すぐにそれが完全に理にかなっていることがわかります。関数の定義と関数の呼び出しは関数型プログラミングの中心であるため、その構文は最小になるよう簡略化されています。引数リストの周りに括弧がないだけでなく、引数間にコンマがないことにも気づくでしょう (後で、複数の引数を持つ関数を定義するときにそれを見ることになります)。

関数の本体は常に式です --- 関数内には文がありません。関数の結果はこの式です --- ここでは、単に\code{x}です。

これで第二のHaskellレッスンは終了です。

恒等性に関する条件は (再び、擬似Haskellで) 以下のように書かれます:

\src{snippet06}
あなたは疑問を持ったかもしれません。何もしない恒等関数を一体誰が気にするのだろうかと。では、なぜ私たちはゼロという数字を気にするのでしょうか?ゼロは無の象徴です。古代ローマ人はゼロのない数字システムを持っていましたが、それでも彼らは優れた道路や水道を建設しました。そのいくつかは今日まで残っています。

記号的な変数を扱うとき、ゼロや$\id$のような中立的な値は非常に便利です。それが、代数を得意としなかったローマ人と、ゼロの概念に精通していたアラブ人やペルシャ人の違いです。そのように、高階関数への引数として、または返り値として、恒等関数は非常に便利です。高階関数は関数の記号的操作を可能にします。関数の代数ということです。

まとめると、圏は対象と矢印 (射) で構成されます。射は合成でき、その合成は結合的です。すべての対象には合成の単位となる恒等射があります。

\section{合成とはプログラミングの本質}

関数型プログラマは問題に対して独特の方法でアプローチします。彼らは非常に禅的な質問から始めます。例えば、対話型プログラムを設計する際には、対話とは何なのかと問います。Conwayのライフゲームを実装する際には、生命の意味について考えるでしょう。この精神で、私たちはプログラミングとは何かという問いを立てます。最も基本的なレベルでは、プログラミングとはコンピュータにすべきことを教えることです。「メモリアドレスxの内容をEAXレジスタの内容に加えてください。」のように。しかし、たとえアセンブリ言語でプログラミングしていても、私たちがコンピュータに指示する命令は、より意味のある何かの表現です。私たちは自明ではない問題を解決しています (もし、それが些細な問題であれば、コンピュータの助けは必要ないでしょう)。では、そういう問題はどのようにして解決するでしょうか?私たちは、大きな問題をより小さな問題に分解します。もし小さな問題がまだ大きすぎる場合は、さらに分解を続けます。そして最終的に、すべての小さな問題を解決するコードを書きます。そして、プログラミングの本質が現れます: それらのコードの断片を組み合わせて、より大きな問題の解決策を作り上げます。分解は、断片から元を復元できなければ意味がありません。

この階層的な分解と再構成(合成)のプロセスは、コンピュータによって私たちに課されたものではありません。それは人間の頭脳の限界を反映しています。私たちの脳は、一度に扱える概念の数が限られています。心理学で最も引用される論文の一つである「\urlref{http://en.wikipedia.org/wiki/The_Magical_Number_Seven,_Plus_or_Minus_Two}{The Magical Number Seven, Plus or Minus Two}」は、私たちが心に留めることができる情報の「チャンク 」の数は $7 \pm 2$ であると主張しています。人間の短期記憶に対する理解の詳細は変わるかもしれませんが、それが限られていることは確かです。要するに、私たちにはオブジェクトのスープやコードのスパゲッティを処理することはできません。私たちは構造を必要とするのです。それは、うまく構造化されたプログラムが見た目に楽しいからではなく、そうでなければ私たちの脳が効率的に処理できないからです。私たちはしばしばあるコードの断片をエレガントあるいは美しいと表現しますが、本当の意味は、それが私たちの限られた人間の頭脳にとって消化しやすいということです。エレガントなコードは、私たちの精神的な消化システムに取り込むのにちょうど良いサイズと数のチャンクを作り出します。

では、プログラムの構成に適したチャンクとは何でしょうか?その表面積は体積より遅く増加する必要があります。(私はこのアナロジーが好きです。幾何学的物体の表面積はそのサイズの2乗で増加し、これは体積の増加がサイズの3乗で増加することに比べてより進みが遅いという直感があるです。) ここで表面積とは、チャンクを合成するために必要な情報です。体積とは、それらを実装するために必要な情報です。このアイディアは、一旦チャンクが実装されたら、その実装の詳細を忘れることができ、他のチャンクとどのように相互作用するかに集中できるということです。オブジェクト指向プログラミングでは、表面はオブジェクトのクラス宣言やその抽象インターフェースです。関数型プログラミングでは、それは関数の宣言です。 (少し単純化していますが、それが要点です。) 

圏論は極端な意味で、私たちに対象の内部を見ることを積極的に抑制します。圏論の対象は抽象的で曖昧な実体です。あなたがそれについて知り得るすべては、それがどのように他の対象と関係しているか、つまりそれが射を使ってどのように接続しているかだけです。これはインターネット検索エンジンが、インバウンドリンクとアウトバウンドリンクを分析することによってウェブサイトをランク付けする方法と似ています (彼らが不正を働かない限り)。オブジェクト指向プログラミングでは、理想化されたオブジェクトはその抽象インターフェース (純粋な表面、体積なし) を通じてのみ可視化され、メソッドは射の役割を果たします。オブジェクトの実装を掘り下げて、他のオブジェクトとどのように組み合わせるかを理解する必要がある瞬間に、あなたはプログラミングパラダイムの利点を失います。

\section{チャレンジ}

\begin{enumerate}
  \tightlist
  \item
        あなたのお気に入りの言語で恒等関数を、できる限り良く実装してください (もしHaskellがあなたのお気に入りの言語であれば、2番目に好きな言語で試してください)。
  \item
        あなたのお気に入りの言語で合成関数を実装してください。それは2つの関数を引数として取り、それらの合成を返す関数です。
  \item
        あなたの合成関数が恒等性条件を遵守することをテストするプログラムを書いてください。
  \item
        ワールド・ワイド・ウェブはいかなる意味で圏ですか?リンクは射ですか?
  \item
        人々を対象とし、友人関係を射とした場合、Facebookは圏ですか?
  \item
        有向グラフはどういう場合に圏になりますか?
\end{enumerate}
