% !TEX root = ctfp-print.tex

\lettrine[lhang=0.17]{こ}{れは}\emph{プログラマのための圏論}のScala版です。
Bartosz Milewski氏のブログ投稿シリーズを美しいタイプセットの\acronym{PDF}、
さらにはハードカバーの書籍として利用できるようになり、非常に成功しています。
本書を改善するために、誤字やエラーを修正したり、コードスニペットを他のプログラミング言語に翻訳したりする多数の貢献がありました。

私は、元のHaskellコードに続いてScalaコードを含む、この版の本を紹介することにわくわくしています。
Scalaのコードスニペットは、\urlref{https://github.com/typelevel/CT_from_Programmers.scala}{Typelevel}の貢献者によって寛大にも提供され、この本の形式に合わせて若干修正されています。

複数言語のコードスニペットをサポートするために、私は外部ファイルからコードスニペットを読み込むための\LaTeX{}マクロを使用しています。
これにより、元のテキストをそのままにしながら、他の言語で本を簡単に拡張できます。
このため、「Haskellで」というテキストを見たときは、心の中で「そしてScalaでも」と言葉を付け足すべきです。

コードは以下のようにレイアウトされています: 元のHaskellコードに続いてScalaコードがあります。
それらを区別するために、コードスニペットは、それぞれの言語のロゴ \raisebox{-.2mm}{\includegraphics[height=.3cm]{fig/icons/haskell.png}} および \raisebox{-.2mm}{\includegraphics[height=.3cm]{fig/icons/scala.png}} の主要色を使用した縦棒で左側が囲まれています。例えば: 

\srcsnippet{content/3.6/code/haskell/snippet03.hs}{blue}{haskell}
\unskip
\srcsnippet{content/3.6/code/scala/snippet03.scala}{red}{scala}
\NoIndentAfterThis
さらに、一部のScalaスニペットでは、部分的に適用された型に対してより良い構文をサポートする\urlref{https://github.com/non/kind-projector}{Kind Projector}コンパイラプラグインを使用しています。
