% !TEX root = ../../ctfp-print.tex

\lettrine[lhang=0.17]{圏}{論}およびプログラミングにおいて、モノイドは重要な概念です。圏は厳密な型付け言語に対応し、
モノイドは非型付け言語に対応します。それはモノイドでは任意の二つの矢印を合成できるように、
非型付け言語では任意の二つの関数を合成できるからです (もちろん、プログラムを実行した時に
実行時エラーになるかもしれませんが) 。

単一の対象を持つ圏としてモノイドを記述できることがわかりました。ここでは、射の合成則に
全てのロジックがエンコードされています。この圏論的モデルは、集合を「掛ける」伝統的な
集合論的定義のモノイドと完全に等価です。二つの集合の要素を掛け合わせて第三の要素を得る
このプロセスは、さらに分解して、最初に要素のペアを形成し、次にこのペアを既存の要素と
同一視することによって「積」が得られます。

積の第二部分  ---  ペアを既存の要素と同一視すること  ---  を省略したらどうなるでしょうか?
たとえば、任意の集合から始めて、要素の全可能なペアを形成し、それらを新しい要素と呼ぶことができます。
次に、これらの新しい要素を全可能な要素とペアにし、という具合に続けます。これは連鎖反応です  --- 
新しい要素を永遠に追加し続けます。結果として無限集合が得られますが、それは\emph{ほぼ}モノイドです。
しかしモノイドには単位要素と結合則も必要です。問題ないです、特別な単位要素を追加し、
ペアのいくつかを同一視するだけで、単位則と結合則をサポートすることができます。

シンプルな例でこの仕組みを見てみましょう。二つの要素$\{a, b\}$の集合から始めます。
これらを自由モノイドの生成器と呼びます。まず、単位として機能する特別な要素$e$を追加します。
次に、要素のペアを全て追加して「積」と呼びます。$a$と$b$の積はペア$(a, b)$になります。
$b$と$a$の積はペア$(b, a)$、$a$と$a$の積は$(a, a)$、$b$と$b$の積は
$(b, b)$になります。$e$とペアを形成することもできますが、$(a, e)$、$(e, b)$などは
$a$、$b$などと同一視します。したがって、このラウンドでは$(a, a)$、$(a, b)$、$(b, a)$、
$(b, b)$のみを追加し、集合$\{e, a, b, (a, a), (a, b), (b, a), (b, b)\}$を得ます。

\begin{figure}[H]
  \centering
  \includegraphics[width=0.8\textwidth]{images/bunnies.jpg}
\end{figure}

\noindent
次のラウンドでは、$(a, (a, b))$、$((a, b), a)$などの要素を追加し続けます。
この時点で結合則が成り立つことを確認する必要がありますから、$(a, (b, a))$を$((a, b), a)$などと
同一視します。言い換えると、内部の括弧は必要ありません。

このプロセスの最終結果を想像できるでしょう: $a$と$b$の全可能なリストを作成します。
実際、$e$を空リストとして表現すれば、私たちの「掛け算」はリストの連結に他なりません。

このような構成、つまり要素の全可能な組み合わせを生成し、規則をサポートするために必要最小限の
同一視を行うことは、自由構成と呼ばれます。私たちが行ったことは、生成器の集合$\{a, b\}$から
\newterm{自由モノイド}を構成することでした。

\section{Haskellにおける自由モノイド}

Haskellにおける二要素の集合は\code{Bool}型に相当し、この集合によって生成される自由モノイドは
\code{{[}Bool{]}}型 (\code{Bool}のリスト) に相当します。 (私は無限リストの問題を意図的に無視しています。) 

Haskellにおけるモノイドは型クラスによって定義されます: 

\src{snippet01}
これは、すべての\code{Monoid}には\code{mempty}と呼ばれる中立要素と、\code{mappend}と呼ばれる二項関数 (掛け算) が必要であることを意味します。単位則と結合則はHaskellでは表現できず、モノイドをインスタンス化するたびにプログラマによって検証されなければなりません。

任意の型のリストがモノイドを形成するという事実は、このインスタンス定義によって記述されています: 

\src{snippet02}
これは、空リスト\code{{[}{]}}が単位要素であり、リストの連結\code{(++)}が二項演算であることを示しています。

見てきたように、型\code{a}のリストは、生成器としての集合\code{a}を持つ自由モノイドに対応しています。自然数の集合とその積は自由モノイドではありません。なぜなら、多くの積が同一視されるからです。例えば: 

\src{snippet03}
これは簡単ですが、問題は、圏論において、私たちが対象の内部を見ることが許されない場合に、この自由構成を行うことができるかということです。私たちは普遍構成を使用します。

二つ目の興味深い問題は、任意のモノイドが、規則が要求する最小限の要素数以上を同一視することによって、何らかの自由モノイドから得られるかどうかです。普遍構成から直接導かれることを示します。

\section{自由モノイドの普遍構成}

以前の普遍構成の経験を振り返ると、それが何かを構成することよりも、与えられたパターンに最も適合する対象を選択することについてであることに気付くかもしれません。ですから、自由モノイドを「構成」するために普遍構成を使用する場合、私たちは選択するために一連のモノイドを考慮しなければなりません。私たちは選択するために、モノイドの全圏が必要です。しかし、モノイドは圏を形成しますか?

まず、単位と掛け算によって定義される追加構造を備えた集合としてモノイドを見てみましょう。モノイダル構造を保存する関数を射として選びます。
このような構造保存関数は\newterm{準同型}と呼ばれます。モノイドの準同型は、二つの要素の積を、
二つの要素のマッピングの積にマップしなければなりません: 

\src{snippet04}
そして、単位を単位にマップしなければなりません。

たとえば、整数のリストから整数への準同型を考えてみましょう。もし\code{{[}2{]}}を2に、\code{{[}3{]}}を3にマップするなら、
\code{{[}2, 3{]}}を6にマップしなければなりません。なぜなら、連結

\src{snippet05}
は掛け算になります

\src{snippet06}
では、個々のモノイドの内部構造を忘れて、それらを対応する射とともに対象としてのみ見ます。$\cat{Mon}$のモノイドの圏を得ます。

まあ、内部構造を忘れる前に、重要な特性に気づいておきましょう。$\cat{Mon}$のすべての対象は、自明な方法で集合にマッピングすることができます。それは単にその要素の集合です。この集合は\newterm{台となる}集合と呼ばれます。実際、$\cat{Mon}$の対象だけでなく、$\cat{Mon}$の射 (準同型) も$\Set$にマップすることができます。これは種類のトリビアルなように思えるかもしれませんが、すぐに役立ちます。この対象と射のマッピングは実際には関手です。この関手はモノイダル構造を「忘れる」ため、一度ただの集合の中に入れば、単位要素を区別したり、掛け算を気にしたりする必要はありません。それで、これは\newterm{忘却関手}と呼ばれます。忘却関手は圏論において定期的に現れます。

私たちは今、$\cat{Mon}$を二つの異なる視点から見ることができます。対象と射のある他の圏と同じにそれを扱うことができます。この視点では、モノイドの内部構造を見ることはできません。$\cat{Mon}$の特定の対象について言えるのは、それが自身および他の対象に対して射を通じて接続されていることだけです。射の「掛け算」表 --- 合成則 --- は他の視点、つまり集合としてのモノイドから派生します。圏論に移行してもこの視点を完全には失っていません --- 忘却関手を通じてそれにアクセスできます。

普遍構成を適用するには、モノイドの圏を通じて探索し、自由モノイドに最適な候補を選ぶための特別な特性を定義する必要があります。しかし、自由モノイドはその生成器によって定義されます。異なる生成器の選択は異なる自由モノイドを生み出します (\code{Bool}のリストは\code{Int}のリストとは異なります) 。私たちの構成は生成器の集合から始める必要があります。だから、再び集合に戻ります!

ここで忘却関手が役立ちます。それを使って私たちのモノイドをX線写真で見ることができます。それらのブロブのX線画像内で生成器を識別することができます。こうして機能します: 

私たちは、$\Set$内の生成器の集合$x$から始めます。

私たちが一致させようとするパターンは、$\cat{Mon}$の対象であるモノイド$m$と$\Set$内の関数$p$で構成されます: 

\src{snippet07}
ここで、$U$は$\cat{Mon}$から$\Set$への忘却関手です。これは$\cat{Mon}$と$\Set$の半分が混在した奇妙な異種パターンです。

この考えは、関数$p$が$m$のX線画像内の生成器を識別するというものです。関数が集合内の点を識別するのが下手でも問題ありません (それらは点を崩壊させるかもしれません) 。普遍構成によって、このパターンの最良の代表者を選ぶことがすべて整理されます。

\begin{figure}[H]
  \centering
  \includegraphics[width=0.4\textwidth]{images/monoid-pattern.jpg}
\end{figure}

\noindent
候補者の中からランク付けするために、別の候補者を考えてみましょう: モノイド$n$と、そのX線画像内で生成器を識別する関数: 

\src{snippet08}
$m$が$n$よりも優れていると言えるのは、モノイドの射 (構造保存準同型) が存在し: 

\src{snippet09}
その$U$下での像 ($U$は関手なので、射を関数にマップします) が$p$を介して分解される場合です: 

\src{snippet10}
$p$が$m$内の生成器を選び、$q$が$n$内の「同じ」生成器を選ぶと考えると、$h$はこれらの生成器を二つのモノイド間でマッピングしていると考えることができます。$h$は、定義上、モノイダル構造を保存します。つまり、一つのモノイド内の二つの生成器の積は、二番目のモノイド内の対応する二つの生成器の積にマッピングされることを意味します。

\begin{figure}[H]
  \centering
  \includegraphics[width=0.4\textwidth]{images/monoid-ranking.jpg}
\end{figure}

\noindent
このランキングは、自由モノイドである最良の候補を見つけるために使用されます。定義は次のようになります: 

\begin{quote}
  $m$ (および関数$p$) が生成器$x$を持つ\textbf{自由モノイド}であると言えるのは、$m$から他の任意のモノイド$n$ (および関数$q$) への\emph{一意の}射$h$が存在し、上記の分解特性を満たす場合のみです。
\end{quote}
ちなみにこれは私たちの二番目の質問にも答えます。$U h$は、$U m$の多くの要素を$U n$の単一の要素に崩壊させる力を持っています。この崩壊は、自由モノイドのいくつかの要素を同一視することに対応します。したがって、生成器$x$を持つ任意のモノイドは、自由モノイドが$x$に基づいている場合、いくつかの要素を同一視することによって得られます。自

由モノイドは、最小限の同一視のみが行われたものです。

自由モノイドについては、随伴について話すときに戻ってきます。

\section{チャレンジ}

\begin{enumerate}
  \tightlist
  \item
        あなたは (私が最初にしたように) モノイドの準同型が単位を保存する要件が冗長だと思うかもしれません。なぜなら、すべての$a$に対して

        \begin{snip}{text}
h a * h e = h (a * e) = h a
\end{snip}
        ですから、$h e$は右単位 (そして類推により左単位) のように振る舞います。問題は、すべての$a$に対する$h a$が、ターゲットモノイドの部分モノイドをカバーするだけかもしれないということです。$h$の像の外側に「真の」単位が存在するかもしれません。乗法を保存するモノイド間の同型が自動的に単位を保存することを示してください。
  \item
        整数のリストと連結に対するモノイド準同型を整数とその積に対するモノイド準同型として考えてみてください。空リスト\code{{[}{]}}の像は何ですか?すべての単一リストがそれらが含む整数にマップされると仮定します。つまり、\code{{[}3{]}}は3にマップされます。では、\code{{[}1, 2, 3, 4{]}}の像は何でしょうか?何個の異なるリストが整数12にマップされますか?他にも両方のモノイド間の準同型が存在しますか?
  \item
        単集合によって生成される自由モノイドは何ですか?それが何と同型であるかわかりますか?
\end{enumerate}

