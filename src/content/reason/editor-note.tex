% !TEX root = ctfp-print.tex

\lettrine[lhang=0.17]{こ}{れは}、\emph{プログラマのための圏論}のReasonML版です。Bartosz Milewski氏のブログ記事シリーズを、美しく
組版された\acronym{PDF}、そしてハードカバーの書籍として利用できるようになり、大変な成功を収めました。多くの貢献があり、
タイプミスや誤りの修正、コードスニペットを他のプログラミング言語に翻訳する作業が行われました。

この本のこの版を紹介できることにわくわくしています。元のHaskellコードに続いて、そのReasonMLでの対応版を含んでいます。
ReasonMLコードスニペットは、ocaml-ctfpの貢献者によって提供されたOCamlスニペットから変換され、この本の形式に合わせてわずかに修正されました。

複数の言語のコードスニペットをサポートするために、外部ファイルからコードスニペットを読み込む\LaTeX{}マクロを使用しています。
これにより、元のテキストをそのままにしつつ、他の言語での本を簡単に拡張できます。そのため、テキスト内で「Haskellで」という言葉を見るたびに、心の中で「そしてReasonMLで」と言葉を足すべきです。

コードは以下のようにレイアウトされています: 元のHaskellコードに続いて、ReasonMLコードがあります。これらを区別するために、
コードスニペットはそれぞれの言語のロゴ \raisebox{-.2mm}{\includegraphics[height=.3cm]{fig/icons/haskell.png}} と \raisebox{-.2mm}{\includegraphics[height=.3cm]{fig/icons/reason.png}} の主要色を使用した縦棒で左側が囲まれています。例えば: 

\srcsnippet{content/1.1/code/haskell/snippet03.hs}{blue}{haskell}
\unskip
\srcsnippet{content/1.1/code/reason/snippet03.re}{RedOrange}{reason}
\NoIndentAfterThis
