% !TEX root = ../../ctfp-print.tex

\lettrine[lhang=0.17]{圏}{論}では、すべてが何かしらと関連しており、多角的に見ることができます。例えば、\hyperref[products-and-coproducts]{積}の普遍構成を取り上げてみましょう。関手や\hyperref[natural-transformations]{自然変換}についてより深く理解すると、これを単純化し、場合によっては一般化できるのではないでしょうか?試してみましょう。

\begin{figure}[H]
  \centering
  \includegraphics[width=0.3\textwidth]{images/productpattern.jpg}
\end{figure}

\noindent
積の構成は、積を構成したい二つの対象$a$と$b$を選択することから始まります。しかし、「対象を選択する」とはどういう意味でしょうか?これをより圏論的な用語で言い換えることはできるでしょうか?二つの対象は非常に単純なパターンを形成します。このパターンを、非常に単純だけれども圏である$\cat{2}$と呼ぶ圏に抽象化できます。これにはただ二つの対象$1$と$2$が含まれ、それ以外の射は二つの必須の恒等射だけです。これで、二つの対象を$\cat{C}$で選択する行為を、圏$\cat{2}$から圏$\cat{C}$への関手$D$を定義することと言い換えることができます。関手は対象を対象に写像しますから、その像はただの二つの対象です (一つになることもありますが、それも問題ありません) 。また、射も写像しますが、この場合は単に恒等射を恒等射に写像します。

このアプローチの素晴らしい点は、祖先の狩猟採集語彙から直接取られた「対象を選択する」といった不正確な表現を避けて、圏論の概念に基づいていることです。そして、偶然にも、$\cat{2}$より複雑な圏を使用してパターンを定義することを止めるものは何もないので、簡単に一般化できます。

\begin{figure}[H]
  \centering
  \includegraphics[width=0.35\textwidth]{images/two.jpg}
\end{figure}

\noindent
さて、積の定義の次のステップは、候補となる対象$c$を選択することです。ここでも、単一の対象からの関手を用いて選択を言い換えることができます。そして実際、Kan拡張を使用していれば、それが正しい方法です。しかし、まだKan拡張の準備ができていないので、別の技を使います: 同じ圏$\cat{2}$から$\cat{C}$への定数関手$\Delta$です。$\cat{C}$で$c$を選択することは、$\Delta_c$で行えます。$\Delta_c$はすべての対象を$c$に、そしてすべての射を$\idarrow[c]$に写像します。

\begin{figure}[H]
  \centering
  \includegraphics[width=0.35\textwidth]{images/twodelta.jpg}
\end{figure}

\noindent
これで二つの関手、$\Delta_c$と$D$が$\cat{2}$と$\cat{C}$の間にできましたから、それらの間の自然変換について考えるのは自然です。$\cat{2}$には二つの対象しかないので、自然変換には二つのコンポーネントがあります。$\cat{2}$の対象$1$は$\Delta_c$によって$c$に、そして$D$によって$a$に写像されます。したがって、$\Delta_c$と$D$の間の自然変換のコンポーネント$1$は、$c$から$a$への射です。これを$p$と呼びましょう。同様に、二番目のコンポーネントは$c$から$b$への射$q$です --- $D$による$\cat{2}$の対象$2$の像です。しかし、これらはまさに私たちが積の元の定義で使用した二つの射影のようです。したがって、対象と射影を選ぶ代わりに、関手と自然変換を選ぶだけでよいのです。この単純な場合では、自然変換に対する自然性条件は、$\cat{2}$に恒等射以外の射がないため、自明に満たされます。

\begin{figure}[H]
  \centering
  \includegraphics[width=0.35\textwidth]{images/productcone.jpg}
\end{figure}

\noindent
この構成を、$\cat{2}$以外の圏、例えば非自明な射を含むものに一般化すると、$\Delta_c$と$D$の間の変換に自然性条件を課すことになります。そのような変換を\emph{錐}と呼びます。なぜなら、$\Delta$の像は、自然変換のコンポーネントによって形成される錐/ピラミッドの頂点であり、$D$の像が錐の底面を形成するからです。

一般に、錐を構成するには、パターンを定義する小さな、しばしば有限の圏$\cat{I}$から始めます。$\cat{I}$から$\cat{C}$への関手$D$を選び、それ (またはその像) を\emph{図式}と呼びます。私たちの錐の頂点として何らかの$c$を$\cat{C}$で選びます。それを使って、$\cat{I}$から$\cat{C}$への定数関手$\Delta_c$を定義します。$\Delta_c$から$D$への自然変換が、私たちの錐です。有限の$\cat{I}$に対しては、それは単に$c$を図式に接続するいくつかの射です: $D$による$\cat{I}$の像です。

\begin{figure}[H]
  \centering
  \includegraphics[width=0.35\textwidth]{images/cone.jpg}
\end{figure}

\noindent
自然性は、この図のすべての三角形 (ピラミッドの壁) が交差することを要求します。実際、$\cat{I}$の任意の射$f$を取ります。関手$D$はそれを$\cat{C}$の射$Df$に写像します。これは、いくつかの三角形の底面を形成する射です。定数関手$\Delta_c$は$f$を$c$上の恒等射に写像します。$\Delta$は射の両端を一つの対象に押し潰し、自然性の四角形が交差する三角形になります。この三角形の二つの腕は自然変換のコンポーネントです。

\begin{figure}[H]
  \centering
  \includegraphics[width=0.35\textwidth]{images/conenaturality.jpg}
\end{figure}

\noindent
これが一つの錐です。私たちが興味を持っているのは、\newterm{普遍錐}です --- 積の定義で私たちが選んだ普遍対象のように。

それにはいくつかの方法があります。例えば、与えられた関手$D$に基づいて\emph{錐の圏}を定義することができます。その圏の対象は錐です。しかし、$\cat{C}$のすべての対象$c$が錐の頂点になるわけではありません。なぜなら、$\Delta_c$と$D$の間に自然変換が存在しないかもしれないからです。

それを圏にするためには、錐の間の射も定義する必要があります。これらは頂点間の射によって完全に決定されます。しかし、どんな射でも良いわけではありません。私たちが積の構成で課した条件は、射影の共通因子である頂点間の射の条件でした。例えば: 

\src{snippet01}

\begin{figure}[H]
  \centering
  \includegraphics[width=0.35\textwidth]{images/productranking.jpg}
\end{figure}

この条件は、一般的な場合には、因子化射 (factorizing morphism) の一辺が三角形を形成する条件に翻訳されます。

\begin{figure}[H]
  \centering
  \includegraphics[width=0.35\textwidth]{images/conecommutativity.jpg}
  \caption{二つの錐を接続する交差する三角形、因子化射$h$ (ここで、下の錐は普遍的なもので、その頂点は$\Lim[D]$です) }
\end{figure}

\noindent
因子化射を錐の圏の射として取ります。これらの射が確かに合成され、恒等射が因子化射であることを確認するのは簡単です。したがって、錐は圏を形成します。

これで、普遍錐を錐の圏の\emph{終対象}として定義することができます。終対象の定義は、他のどの対象からもその対象に一意の射があることを意味します。私たちの場合、それは他のどの錐の頂点からも普遍錐の頂点への一意の因子化射があることを意味します。この普遍錐を図式$D$の\emph{極限}、$\Lim[D]$と呼びます (文献では、しばしば$\Lim$記号の下に$I$に向かって左向きの矢印を見ます) 。しばしば、この錐の頂点を単に極限 (または極限対象) と呼びます。

直感的には、極限は図式全体の特性を単一の対象に体現するということです。例えば、二つの対象の図式の極限は、二つの対象の積です。積 (とその二つの射影と共に) は、両方の対象に関する情報を含んでいます。そして、普遍であるということは、余分なものが何も含まれていないことを意味します。

\section{極限と自然同型}

しかし、この極限の定義にはまだ満足できない点があります。確かに、これは実用的ですが、二つの錐を繋ぐ三角形の交換条件を何らかの自然性条件に置き換えることができれば、もっとエレガントになるでしょう。しかし、どうすればよいのでしょうか?

ここで我々は一つの錐ではなく、錐の全体的なコレクション (実際には、錐の圏) を扱っています。もし極限が存在する (そして --- 明らかにするが --- それは保証されていない) 、それらの錐の一つは普遍錐です。他のどの錐に対しても、その頂点を普遍錐の頂点である$\Lim[D]$へ写像する特別な種類の一意の射があります (実際、「他の」という言葉を省略することができます。なぜなら、恒等射は普遍錐をそれ自身に写像し、それは自身を通して自明に因子化されるからです) 。重要な部分を繰り返します: 任意の錐が与えられると、特別な種類の一意の射の写像があります。これは錐を特別な射への写像であり、それは一対一の写像です。

この特別な射はホム集合$\cat{C}(c, \Lim[D])$のメンバーです。このホム集合の他のメンバーは、二つの錐の写像を因子化しないという意味で不幸です。我々が望むのは、各$c$に対して、特定の交換条件を満たす$\cat{C}(c, \Lim[D])$の集合から一つの射を選ぶことができるようにすることです。これは自然変換を定義するようなものに聞こえますか?確かにそうです!

しかし、この変換に関連する二つの関手は何でしょうか?

一つの関手は、$c$を集合$\cat{C}(c, \Lim[D])$に写像するものです。これは$\cat{C}$から$\Set$への関手です --- 対象を集合に写像します。実際、これは反変関手です。射の作用を定義するために、$c'$から$c$への射$f$を取ります: 
\[f \Colon c' \to c\]
私たちの関手は$c'$を集合$\cat{C}(c', \Lim[D])$に写像します。この関手の射$f$に対する作用を定義するために (つまり、$f$を持ち上げるために) 、集合$\cat{C}(c, \Lim[D])$と$\cat{C}(c', \Lim[D])$の間の対応する写像を定義する必要があります。そこで、集合$\cat{C}(c, \Lim[D])$の一要素$u$を選び、それを何らかの$\cat{C}(c', \Lim[D])$の要素に写像することができるかを見ます。ホム集合の要素は射ですから、我々は次のように持っています: 
\[u \Colon c \to \Lim[D]\]
$u$を$f$で前合成すると、次のものが得られます: 
\[u \circ f \Colon c' \to \Lim[D]\]
そして、これは$\cat{C}(c', \Lim[D])$の要素です --- したがって、確かに、我々は射の写像を見つけました: 

\src{snippet02}
$c$と$c'$の順序の反転に注意してください --- これは\emph{反変}関手の特徴です。

\begin{figure}[H]
  \centering
  \includegraphics[width=0.35\textwidth]{images/homsetmapping.jpg}
\end{figure}

\noindent
自然変換を定義するために、我々はもう一つの関手が必要で、それも$\cat{C}$から$\Set$への写像です。しかし今回は、錐の集合を考えます。錐はただの自然変換ですから、我々は自然変換$\mathit{Nat}(\Delta_c, D)$の集合を見ています。$c$からこの特定の自然変換の集合への写像は、 (反変) 関手です。それを示すにはどうすればよいでしょうか?再び、射に対するその作用を定義しましょう: 
\[f \Colon c' \to c\]
$f$の持ち上げは、$\cat{I}$から$\cat{C}$への二つの関手間の自然変換の写像でなければなりません: 
\[\mathit{Nat}(\Delta_c, D) \to \mathit{Nat}(\Delta_{c'}, D)\]
自然変換をどのように写像しますか?各自然変換は射の選択です --- そのコンポーネントは一つずつです --- $\cat{I}$の各要素に一つの射です。ある$\alpha$ ($\mathit{Nat}(\Delta_c, D)$のメンバー) の$\cat{I}$の対象$a$におけるコンポーネントは射です: 
\[\alpha_a \Colon \Delta_c a \to D a\]
または、定数関手$\Delta$の定義を使って、
\[\alpha_a \Colon c \to D a\]
$f$と$\alpha$が与えられた場合、$\mathit{Nat}(\Delta_{c'}, D)$のメンバーである$\beta$を構成する必要があります。その$a$におけるコンポーネントは次のような射でなければなりません: 
\[\beta_a \Colon c' \to D a\]
我々は容易に後者 ($\beta_a$) を前者 ($\alpha_a$) から前合成して$f$で得ることができます: 
\[\beta_a = \alpha_a \circ f\]
これらのコンポーネントが実際に自然変換を加えることは比較的簡単に示せます。

\begin{figure}[H]
  \centering
  \includegraphics[width=0.4\textwidth]{images/natmapping.jpg}
\end{figure}

\noindent
射$f$が与えられた我々は、二つの自然変換間の写像をコンポーネントごとに構成しました。この写像は関手の\code{contramap}を定義します: 
\[c \to \mathit{Nat}(\Delta_c, D)\]
私が今示したのは、我々が$\cat{C}$から$\Set$への二つの (反変) 関手を持っているということです。私はどんな仮定もしていません --- これらの関手は常に存在します。

ちなみに、これらの関手のうちの最初のものは圏論において重要な役割を果たし、我々が米田の補題について話すときに再び登場します。任意の圏$\cat{C}$から$\Set$への反変関手には名前があります: それらは「前層」と呼ばれます。この一つは\newterm{表現可能前層}と呼ばれます。二番目の関手も前層です。

これで、我々は二つの関手を持っているので、それらの間の自然変換について話すことができます。それでは遠慮なく、ここが結論です: $\cat{I}$から$\cat{C}$への関手$D$は、私が今定義した二つの関手の間に自然同型が存在する場合、そしてその場合に限り、極限$\Lim[D]$を持ちます: 
\[\cat{C}(c, \Lim[D]) \simeq \mathit{Nat}(\Delta_c, D)\]
自然同型とは何かを思い出してください。それは、その全てのコンポーネントが同型、つまり可逆な射である自然変換です。

この声明の証明を通過するつもりはありません。その手順は、退屈であってもまっすぐです。自然変換を扱うときは、通常は射であるコンポーネントに焦点を当てます。この場合、両方の関手のターゲットが$\Set$なので、自然同型のコンポーネントは関数です。これらは高次の関数です。なぜなら、それらはホム集合から自然変換の集合へ行くからです。再び、関数をその引数が何をするかで分析することができます: ここでは引数は射です --- $\cat{C}(c, \Lim[D])$のメンバーです --- そして結果は自然変換です --- $\mathit{Nat}(\Delta_c, D)$のメンバー、または私たちが錐と呼んだものです。この自然変換は、その独自のコンポーネントを持っており、それらは射です。それで、それは射ですべての方法であり、それらを追跡することができれば、声明を証明することができます。

最も重要な結果は、この同型の自然性条件が、錐の写像の交換条件と正確に一致することです。

これからのアトラクションのプレビューとして、集合$\mathit{Nat}(\Delta_c, D)$を関手圏のホム集合と見なすことができると言及しておきます。したがって、私たちの自然同型は二つのホム集合を関連付けており、これは随伴と呼ばれるさらに一般的な関係を指しています。

\section{極限の例}

我々は、単純な圏$\cat{2}$によって生成される図式の極限として、圏論的な積を見てきました。

さらに単純な極限の例があります: 終対象です。最初の衝動は、単一の対象から終対象に至ると考えるかもしれませんが、真実はそれよりも更にシンプルです: 終対象は空の圏によって生成される極限です。空の圏からの関手は対象を選択しませんので、錐は単に頂点だけになります。普遍錐は、他のどの頂点からも唯一の射が来る単独の頂点です。これは終対象の定義として認識されます。

次に興味深い極限は\emph{イコライザ}と呼ばれます。これは二つの要素と、それらの間に二つの平行な射がある二要素の圏によって生成される極限です (恒等射も常に含まれます) 。この圏は$\cat{C}$の図式で、二つの対象$a$と$b$、そして二つの射を選択します: 

\src{snippet03}

この図式の上に錐を構成するためには、頂点$c$と二つの射を加える必要があります: 

\src{snippet04}

\begin{figure}[H]
  \centering
  \includegraphics[width=0.35\textwidth]{images/equalizercone.jpg}
\end{figure}

\noindent
二つの三角形が交差する必要があります: 

\src{snippet05}

これは$q$がこれらの方程式の一つ、例えば\code{q = f . p}によって一意に決まることを教えてくれます。したがって、私たちは図からそれを省略することができます。ですから、私たちが残されるのは次の一つの条件です: 

\src{snippet06}

これについて考える方法は、我々が$\Set$に注目を限定するならば、関数$p$の像が$a$の部分集合を選択するということです。この部分集合に限定すると、関数$f$と$g$は等しくなります。

例として、$a$が$x$と$y$の座標によってパラメータ化される二次元平面を、$b$を実数の線とし、次のように取ります: 

\src{snippet07}

これら二つの関数のイコライザは、実数の集合 (頂点$c$) と関数: 

\src{snippet08}

です。$(p~t)$は二次元平面における直線を定義します。この線上で、二つの関数は等しくなります。

もちろん、他の集合$c'$と関数$p'$が存在して、等式: 

\src{snippet09}

を満たすかもしれませんが、それらはすべて$p$を通じて一意に因子化されます。例えば、単集合$\cat{()}$を$c'$として、関数: 

\src{snippet10}

を取ることができます。これは$f (0, 0) = g (0, 0)$なので良い錐です。しかし、それは$h$を通じて因子化されるため、普遍的ではありません: 

\src{snippet11}

以下のようになります: 

\src{snippet12}

\begin{figure}[H]
  \centering
  \includegraphics[width=0.35\textwidth]{images/equilizerlimit.jpg}
\end{figure}

\noindent
従って、イコライザは$f~x = g~x$のような方程式を解くために使用することができます。しかし、それは物と射の観点で定義されているため、はるかに一般的です。

方程式を解くというさらに一般的なアイデアは、別の極限 --- \emph{引き戻し}に体現されています。ここでも、私たちは等しいとしたい二つの射がありますが、今回はそれらの出発点が異なります。$1\rightarrow2\leftarrow3$の形をした三要素の圏で始まります。この圏に対応する図式は三つの対象$a$、$b$、そして$c$、そして二つの射から成ります: 

\src{snippet13}

この図式はしばしば\emph{余スパン}と呼ばれます。

この図式の上に構成された錐は、頂点$d$と三つの射から成ります: 

\src{snippet14}

\begin{figure}[H]
  \centering
  \includegraphics[width=0.35\textwidth]{images/pullbackcone.jpg}
\end{figure}

\noindent
交換性の条件は$r$が他の射によって完全に決定されることを示しており、図から省略することができます。そこで、我々が残されるのは次の条件のみです: 

\src{snippet15}

引き戻しはこの形の普遍錐です。

\begin{figure}[H]
  \centering
  \includegraphics[width=0.35\textwidth]{images/pullbacklimit.jpg}
\end{figure}

\noindent
もしあなたが集合に焦点を絞るならば、$d$の対象を$a$と$c$の要素のペアで考えることができます。これらの要素のペアで、$f$が第一要素に作用するのと同じ結果を、$g$が第二要素に作用するとします。これが一般的には難しい場合は、$g$が定数関数、たとえば$g~\_ = 1.23$である特殊なケースを考えてみてください ($b$が実数の集合であると仮定します) 。

この場合、あなたは実際に方程式: 

\src{snippet16}

を解いています。

このケースでは、$c$の選択は関係ありません (空集合でない限り) 。ですから、$c$として単集合を取ることができます。集合$a$は例えば、三次元ベクトルの集合であり、そして$f$はベクトルの長さであるとします。すると、引き戻しは1.23の長さを持つベクトル$v$ (方程式$\sqrt{(x^{2}+y^{2}+z^{2})} = 1.23$の解) と、単集合のダミー要素$()$のペア$(v, ())$の集合になります。

しかし、引き戻しはプログラミングでもより一般的な用途を持っています。例えば、C++のクラスを射がサブクラスからスーパークラスへの矢印となる圏として考えます。継承を推移的な性質と見なし、もし\code{C}が\code{B}から、そして\code{B}が\code{A}から継承するならば、我々は\code{C}が\code{A}から継承すると言います (結局、\code{C}へのポインターを\code{A}が要求される場所に渡すことができます) 。また、\code{C}が\code{C}自身から継承するとも見なしますので、各クラスに対して恒等射があります。これにより、サブクラス化はサブタイピングと一致します。C++はまた、複数継承をサポートしているので、\code{A}から継承する\code{B}と\code{C}の二つのクラス、そして\code{B}と\code{C}から複数継承する第四のクラス\code{D}を持つダイヤモンド継承図を構成することができます。通常、\code{D}は\code{A}の二つのコピーを得るでしょうが、これはめったに望まれることではありません。しかし、仮想継承を使用して\code{D}内の\code{A}のコピーを一つだけにすることができます。

\code{D}がこの図で引き戻しであることが意味するのは何でしょうか?それは、\code{B}と\code{C}の両方から複数継承する任意のクラス\code{E}が、また\code{D}のサブクラスでもあるということを意味します。これはC++では直接表現できません。サブタイピングは名目的であり、C++コンパイラはこの種のクラス関係を推論しません --- それは「ダックタイピング」が必要です。しかし、サブタイピング関係の外に出て、\code{E}から\code{D}へのキャストが安全かどうかを尋ねることはできます。このキャストは安全である可能性がありますが、\code{D}が\code{B}と\code{C}の裸の組み合わせであり、データを追加せず、メソッドをオーバーライドせずに、そして\code{B}と\code{C}の間にあるメソッドの名前の衝突がない場合に限ります。

\begin{figure}[H]
  \centering
  \includegraphics[width=0.25\textwidth]{images/classes.jpg}
\end{figure}

\noindent
また、型推論における引き戻しのより高度な使用もあります。二つの式の型を\emph{統合}する必要がある場合がよくあります。例えば、コンパイラが関数の型を推論したいとします: 

\begin{snip}{haskell}
twice f x = f (f x)
\end{snip}
それはすべての変数と部分式に暫定的な型を割り当てます。特に、次のように割り当てます: 

\begin{snip}{haskell}
f       :: t0
x       :: t1
f x     :: t2
f (f x) :: t3
\end{snip}
これから次のように推論します: 

\begin{snip}{haskell}
twice :: t0 -> t1 -> t3
\end{snip}
関数適用の規則から一連の制約が生じます: 

\begin{snip}{haskell}
t0 = t1 -> t2 -- fがxに適用されるから
t0 = t2 -> t3 -- fが(f x)に適用されるから
\end{snip}
これらの制約は、両方の式で未知の型に同じ型 (または型変数) を代入することにより同じ型を生成する一連の型 (または型変数) を見つけることで統合されなければなりません。そのような代入の一つは次のようになります: 

\begin{snip}{haskell}
t1 = t2 = t3 = Int
twice :: (Int -> Int) -> Int -> Int
\end{snip}
しかし、明らかに、これは最も一般的な代入ではありません。最も一般的な代入は引き戻しを使用して得られます。詳細に入ることはこの本の範囲を超えていますが、結果として次のようになることを自分自身で納得させることができます: 

\begin{snip}{haskell}
twice :: (t -> t) -> t -> t
\end{snip}
ここで\code{t}は自由な型変数です。

\section{余極限}

圏論の全ての構成は、逆圏においてその双対像を持ちます。錐の全ての射の方向を逆にすると、余錐が得られ、その普遍的なものが余極限と呼ばれます。因子化射も反転され、普遍余錐から他の任意の余錐へ流れます。

\begin{figure}[H]
  \centering
  \includegraphics[width=0.35\textwidth]{images/colimit.jpg}
  \caption{因子化射$h$が二つの頂点を接続する余錐。}
\end{figure}

\noindent
余極限の典型的な例は余積であり、積の定義に使用した$\cat{2}$の圏によって生成された図式に対応します。

\begin{figure}[H]
  \centering
  \includegraphics[width=0.35\textwidth]{images/coproductranking.jpg}
\end{figure}

\noindent
積と余積はそれぞれ異なる方法で一対の対象の本質を体現しています。

終対象が極限であったように、始対象は空の圏に基づく図式の余極限に対応する余極限です。

引き戻しの双対は\emph{押し出し}と呼ばれます。それは$1\leftarrow2\rightarrow3$の形の圏によって生成される図式、スパンと呼ばれる図式に基づいています。

\section{連続性}

私は以前、関手が圏の連続写像のアイデアに近いものであると述べました。それは、既存の接続 (射) を決して破壊しないという意味でです。実際の\emph{連続関手} $F$の定義は、$\cat{C}$から$\cat{C'}$への関手が極限を保つという要件を含んでいます。$\cat{C}$の任意の図式$D$は、単純に二つの関手を合成することによって$\cat{C'}$の図式$F \circ D$に写像されます。$F$の連続性条件は、図式$D$が極限$\Lim[D]$を持つ場合、図式$F \circ D$も極限を持ち、それが$F (\Lim[D])$であることを要求します。

\begin{figure}[H]
  \centering
  \includegraphics[width=0.6\textwidth]{images/continuity.jpg}
\end{figure}

\noindent
関手が射を射に写像し、合成を合成に写像するので、錐の像は常に錐です。交差する三角形は常に交差する三角形に写像されます (関手は合成を保存します) 。因子化射も同様です: 因子化射の像も因子化射です。だから、全ての関手は\emph{ほとんど}連続です。何が間違っていく可能性があるかというと、一意性の条件です。$\cat{C'}$における因子化射は一意であるとは限りません。$\cat{C'}$には、$\cat{C}$には存在しなかった他の「より良い錐」もあるかもしれません。

ホム関手は連続関手の一例です。ホム関手、$\cat{C}(a, b)$は、最初の変数において反変であり、第二の変数において共変です。言い換えれば、それは関手です: 
\[\cat{C}^\mathit{op} \times \cat{C} \to \Set\]
第二引数が固定されると、ホム集合関手 (表現可能前層になる) は$\cat{C}$における余極限を$\Set$における極限に写像します。第一引数が固定されると、極限を極限に写像します。

Haskellにおいて、ホム関手は任意の二つの型を関数型に写像するものです。ですから、それは単にパラメータ化された関数型です。第二パラメータを固定すると、例えば\code{String}にすると、反変関手が得られます: 

\src{snippet17}
連続性は、例えば\code{ToString}が余積、例えば\code{Either b c}を適用されると極限を生み出すことを意味します;このケースでは二つの関数型の積です: 

\src{snippet18}
実際、任意の\code{Either b c}の関数は、二つのケースをサービスする一対の関数によるケース文として実装されます。

同様に、

ホム集合の第一引数を固定すると、お馴染みのReader関手が得られます。その連続性は、例えば、積を返す任意の関数が関数の積に相当することを意味します。特に: 

\src{snippet19}
私が知っているあなたが考えていることは、これらのことを理解するのに圏論は必要ないということです。そして、あなたは正しいです!それでも、これらの結果がビットとバイト、プロセッサーアーキテクチャ、コンパイラ技術、さらにはラムダ計算に一切頼ることなく、第一原理から導出されることに私は驚嘆します。

「極限」と「連続性」の名前がどこから来ているのか気になる場合、それらは微積分からの対応する概念の一般化です。微積分における極限と連続性は開集合に関して定義されます。開集合はトポロジーを定義し、トポロジーは圏 (半順序集合) を形成します。

\section{チャレンジ}

\begin{enumerate}
  \tightlist
  \item
        C++のクラスの圏において押し出しをどのように説明しますか?
  \item
        恒等関手$\mathbf{Id} \Colon \cat{C} \to \cat{C}$の極限が始対象であることを示してください。
  \item
        与えられた集合の部分集合は圏を形成します。この圏における射は、一つがもう一つの部分集合である二つの集合を結ぶ矢印として定義されます。このような圏における二つの集合の引き戻しは何ですか?押し出しは何ですか?始対象と終対象は何ですか?
  \item
        余イコライザが何であるか推測できますか?
  \item
        終対象が存在する圏において、終対象に向けての引き戻しが積であることを示してください。
  \item
        同様に、始対象からの押し出しが (存在する場合) 余積であることを示してください。
\end{enumerate}

