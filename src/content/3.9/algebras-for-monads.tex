% !TEX root = ../../ctfp-print.tex

\lettrine[lhang=0.17]{も}{し}自己関手を式の定義方法として解釈するならば、代数はそれらを評価する方法を提供し、モナドはそれらを形成し操作する方法を提供します。代数とモナドを組み合わせることで、多くの機能を得られるだけでなく、いくつかの興味深い問いにも答えることができます。

そのような問いの一つは、モナドと随伴の関係についてです。見てきたように、すべての随伴はモナド (および余モナド) を定義します。問題は、すべてのモナド (余モナド) が随伴から導出されるかどうかです。この答えは肯定的です。与えられたモナドを生成する随伴の全ての族が存在します。私はそのような随伴を二つ示します。

\begin{figure}[H]
  \centering
  \includegraphics[width=0.25\textwidth]{images/pigalg.png}
\end{figure}

\noindent
まず定義を見直しましょう。モナドは自己関手$m$で、いくつかの自然変換とそれらが満たすべき整合性条件で装備されています。これらの変換のコンポーネントは、$a$において以下のようになります: 
\begin{align*}
  \eta_a & \Colon a \to m\ a         \\
  \mu_a  & \Colon m\ (m\ a) \to m\ a
\end{align*}
同じ自己関手のための代数は、特定の対象---担い手$a$---と射: 
\[\mathit{alg} \Colon m\ a \to a\]
を選択するものです。最初に気づくべきは、代数が$\eta_a$とは逆の方向に進むということです。直感的には、$\eta_a$は型$a$の値からトリビアルな式を作ります。代数をモナドと互換性のあるようにする最初の整合性条件は、担い手が$a$である代数を使ってこの式を評価すると、元の値が得られるということです: 
\[\mathit{alg} \circ \eta_a = \id_a\]
二番目の条件は、二重にネストされた式$m\ (m\ a)$を評価する二つの方法があるという事実から生じます。まず$\mu_a$を適用して式を平らにし、その後代数の評価子を使うか、あるいは持ち上げられた評価子を適用して内部の式を評価し、その結果に評価子を適用するかです。二つの戦略が等価であることが望ましいです: 
\[\mathit{alg} \circ \mu_a = \mathit{alg} \circ m\ \mathit{alg}\]
ここで、\code{m alg}は関手$m$を使って$\mathit{alg}$から持ち上げられた射です。次の交換図は二つの条件を記述します (以下では何が来るかを予測して$m$を$T$に置き換えています) : 

\begin{figure}[H]
  \centering
  \begin{subfigure}
    \centering
    \begin{tikzcd}[column sep=large, row sep=large]
      a \arrow[rd, equal] \arrow[r, "\eta_a"]
      & Ta \arrow[d, "\mathit{alg}"] \\
      & a
    \end{tikzcd}
  \end{subfigure}
  \hspace{1cm}
  \begin{subfigure}
    \centering
    \begin{tikzcd}[column sep=large, row sep=large]
      T(Ta) \arrow[r, "T\ \mathit{alg}"] \arrow[d, "\mu_a"]
      & Ta \arrow[d, "\mathit{alg}"] \\
      Ta \arrow[r, "\mathit{alg}"]
      & a
    \end{tikzcd}
  \end{subfigure}
\end{figure}

\noindent
これらの条件をHaskellで表現することもできます。

\src{snippet01}
小さな例を見てみましょう。リスト自己関手の代数は、型\code{a}と、\code{a}のリストから\code{a}を生成する関数で構成されます。この関数は\code{foldr}を使って表現することができます。ここで要素の型も蓄積器の型も\code{a}に等しいと選択します: 

\src{snippet02}
この特定の代数は二引数の関数、それを\code{f}と呼びましょう、と値\code{z}によって指定されます。リスト関手は偶然にもモナドで、\code{return}は値を単一のリストに変換します。代数の合成、ここでは\code{foldr f z}、\code{return}の後に適用されると、\code{x}を以下に変換します: 

\src{snippet03}
ここで\code{f}の作用は中置記法で書かれています。代数がモナドと互換性がある場合、次の整合性条件がすべての\code{x}に対して満たされます: 

\src{snippet04}
\code{f}を二項演算子と見るならば、この条件は\code{z}が右単位要素であることを教えてくれます。

二番目の整合性条件はリストのリストに作用します。\code{join}の作用は個々のリストを連結します。結果のリストを畳み込むことができます。あるいは、個々のリストを最初に畳み込み、その結果のリストを畳み込むこともできます。再び、\code{f}を二項演算子と解釈すると、この条件はこの二項演算が結合的であることを教えてくれます。これらの条件は、\code{(a, f, z)}がモノイドであるとき確かに満たされます。

\section{T-代数}

数学者が自分たちのモナドを$T$と呼ぶのを好むので、それらと互換性のある代数をT-代数と呼びます。ある圏$\cat{C}$の中で与えられたモナド$T$のT-代数は、Eilenberg-Moore圏と呼ばれることが多い圏を形成し、通常$\cat{C}^T$と表記されます。その圏の射は代数の準同型です。これらはF-代数に定義されたのと同じ準同型です。

T-代数は担い手対象と評価子のペアです、$(a, f)$。明らかに$\cat{C}^T$から$\cat{C}$への忘却関手$U^T$があります。これは$(a, f)$を$a$に写します。またT-代数の準同型を$\cat{C}$の対応する担い手対象間の射に写します。随伴についての議論から覚えているかもしれませんが、忘却関手の左随伴は自由関手と呼ばれます。

$U^T$の左随伴は$F^T$と呼ばれます。これは$\cat{C}$の対象$a$を$\cat{C}^T$の自由代数に写します。この自由代数の担い手は$T a$です。その評価子は$T (T a)$から$T a$への射です。$T$がモナドであるので、モナドの$\mu_a$ (Haskellでは\code{join}) を評価子として使用できます。

これがT-代数であることを示すためには、二つの整合性条件が満たされている必要があります: 
\begin{align*}
  \mathit{alg} & \circ \eta_{Ta} = \id_{Ta}     \\
  \mathit{alg} & \circ \mu_a = \mathit{alg} \circ T\ \mathit{alg}
\end{align*}
しかし、これらは代数に$\mu$を代入した場合のモナド則にすぎません。

すでに見たように、すべての随伴はモナドを定義します。実は$F^T$と$U^T$の間の随伴は、その構成に使用されたモナド$T$そのものを定義します。すべてのモナドに対してこの構成を行うことができるので、すべてのモナドは随伴から生成されると結論づけることができます。後で、同じモナドを生成する別の随伴を示します。

計画は次の通りです。まず$F^T$が実際に$U^T$の左随伴であることを示します。これを行うには、この随伴の単位と余単位を定義し、対応する三角恒等式が満たされていることを証明します。次に、この随伴によって生成されたモナドが実際に元のモナドであることを示します。

随伴の単位は自然変換です: 
\[\eta \Colon I \to U^T \circ F^T\]
この変換の$a$コンポーネントを計算しましょう。恒等関手は私たちに$a$を与えます。自由関手は自由代数$(T a, \mu_a)$を生成し、忘却関手はそれを$T a$に減らします。合わせると、$a$から$T a$への写像を得ます。単純にモナド$T$の単位をこの随伴の単位として使用します。

余単位を見てみましょう: 
\[\varepsilon \Colon F^T \circ U^T \to I\]
これのT-代数$(a, f)$でのコンポーネントを計算しましょう。忘却関手は$f$を忘れ、自由関手はペア$(T a, \mu_a)$を生成します。したがって、余単位$\varepsilon$のコンポーネントをT-代数$(a, f)$で定義するためには、Eilenberg-Moore圏内の正しい射、またはT-代数の準同型が必要です: 
\[(T a, \mu_a) \to (a, f)\]
このような準同型は担い手$T a$を$a$に写すべきです。忘れられた評価子$f$を復活させましょう。今度はT-代数の準同型としてそれを使用します。実際に、$f$がT-代数であることを示すのと同じ交換図は、それがT-代数の準同型であることを再解釈することで示すことができます: 

\begin{figure}[H]
  \centering
  \begin{tikzcd}[column sep=large, row sep=large]
    T(Ta) \arrow[r, "T f"] \arrow[d, "\mu_a"]
    & Ta \arrow[d, "f"] \\
    Ta \arrow[r, "f"]
    & a
  \end{tikzcd}
\end{figure}

\noindent
これにより、T-代数圏の対象$(a, f)$での余単位自然変換$\varepsilon$のコンポーネントを$f$として定義しました。

単位と余単位が三角恒等式を満たしていることを示すためには、それらも証明する必要があります。これらは次のようになります: 

\begin{figure}[H]
  \centering
  \begin{subfigure}
    \centering
    \begin{tikzcd}[column sep=large, row sep=large]
      Ta \arrow[rd, equal] \arrow[r, "T \eta_a"]
      & T(Ta) \arrow[d, "\mu_a"] \\
      & Ta
    \end{tikzcd}
  \end{subfigure}%
  \hspace{1cm}
  \begin{subfigure}
    \centering
    \begin{tikzcd}[column sep=large, row sep=large]
      a \arrow[rd, equal] \arrow[r, "\eta_a"]
      & Ta \arrow[d, "f"] \\
      & a
    \end{tikzcd}
  \end{subfigure}
\end{figure}

\noindent
最初の恒等式はモナド$T$の単位則によって支持されます。二番目のものはT-代数$(a, f)$の規則です。

これにより、二つの関手が随伴を形成することが確立されました: 
\[F^T \dashv U^T\]
すべての随伴はモナドを引き起こします。往復
\[U^T \circ F^T\]
は、対応するモナドを生じさせる$\cat{C}$の自己関手です。対象$a$に対するその作用を見てみましょう。$F^T$によって作成された自由代数は$(T a, \mu_a)$です。忘却関手$U^T$は評価子を落とします。したがって、確かに: 
\[U^T \circ F^T = T\]
予想通り、随伴の単位はモナド$T$の単位です。

随伴の余単位がモナドの乗算を次の式を通じて生成することを覚えておくかもしれません: 
\[\mu = R \circ \varepsilon \circ L\]
これは、それぞれ$L$から$L$へ、$R$から$R$への恒等自然変換である二つの自然変換と、中央にある余単位である三つの自然変換の水平合成です。余単位は、T-代数$(a, f)$でのコンポーネントが$f$である自然変換です。

$\mu_a$のコンポーネントを計算しましょう。最初に$F^T$の後に$\varepsilon$を水平合成します。これは$F^T a$での$\varepsilon$のコンポーネントを結果とします。$F^T$が$a$を代数$(T a, \mu_a)$に写し、$\varepsilon$が評価子を選ぶので、$\mu_a$で終わります。左側に$U^T$との水平合成は何も変えません、なぜなら$U^T$の射の作用は自明だからです。したがって、確かに、随伴から得られる$\mu$は元のモナド$T$の$\mu$と同じです。

\section{Kleisli圏}

以前にKleisli圏について見ました。これは他の圏$\cat{C}$とモナド$T$から構成される圏です。この圏を$\cat{C}_T$と呼びます。Kleisli圏$\cat{C}_T$の対象は$\cat{C}$の対象ですが、射は異なります。Kleisli圏の射$f_{\cat{K}}$は、オリジナルの圏で$a$から$T b$への射$f$に対応します。この射を$a$から$b$へのKleisli射と呼びます。

Kleisli圏の射の合成は、Kleisli射のモナド的合成に関して定義されます。例えば、$g_{\cat{K}}$を$f_{\cat{K}}$の後に合成しましょう。Kleisli圏では: 
\begin{gather*}
  f_{\cat{K}} \Colon a \to b \\
  g_{\cat{K}} \Colon b \to c
\end{gather*}
これは、圏$\cat{C}$では以下に対応します: 
\begin{gather*}
  f \Colon a \to T b \\
  g \Colon b \to T c
\end{gather*}
合成を定義します: 
\[h_{\cat{K}} = g_{\cat{K}} \circ f_{\cat{K}}\]
$\cat{C}$のKleisli射として
\begin{align*}
  h & \Colon a \to T c          \\
  h & = \mu \circ (T g) \circ f
\end{align*}
Haskellでは次のように書きます: 

\src{snippet05}
$\cat{C}$から$\cat{C}_T$への関手$F$があり、これは対象に対して自明に作用します。射に対しては、$\cat{C}$の射$f$を、$f$の戻り値を飾るKleisli射を作成することによって$\cat{C}_T$の射に写します。射が与えられたとき: 
\[f \Colon a \to b\]
それは、対応するKleisli射を持つ$\cat{C}_T$の射を作成します: 
\[\eta \circ f\]
Haskellでは次のように書きます: 

\src{snippet06}
また$\cat{C}_T$から$\cat{C}$への関手$G$を定義することもできます。これはKleisli圏の対象$a$を$\cat{C}$の対象$T a$に写します。Kleisli射に対応する射$f_{\cat{K}}$: 
\[f \Colon a \to T b\]
は、$\cat{C}$の射として
\[T a \to T b\]
です。これは最初に$f$を持ち上げてから$\mu$を適用することによって与えられます: 
\[\mu_{T b} \circ T f\]
Haskellの表記ではこれは次のようになります: 

\begin{snipv}
G f\textsubscript{T} = join . fmap f
\end{snipv}
これは、\code{join}を使ってモナド的な束縛を定義する方法として認識できます。

二つの関手が随伴を形成することは容易に見て取れます: 
\[F \dashv G\]
そして、それらの合成$G \circ F$は元のモナド$T$を再生産します。

これが同じモナドを生成する二番目の随伴です。実際、同じモナド$T$を結果とする随伴の全体の圏$\cat{Adj}(\cat{C}, T)$があります。先ほど見たKleisli随伴はこの圏の始対象であり、Eilenberg-Moore随伴は終対象です。

\section{余代数と余モナド}

余モナド
$W$に対しても類似の構成が行われます。私たちは余代数の圏を定義することができ、それは余モナドと互換性のある余代数です。それらは次の図を可換にします: 

\begin{figure}[H]
  \centering
  \begin{subfigure}
    \centering
    \begin{tikzcd}[column sep=large, row sep=large]
      a \arrow[rd, equal]
      & Wa \arrow[l, "\epsilon_a"'] \\
      & a \arrow[u, "\mathit{coa}"']
    \end{tikzcd}
  \end{subfigure}%
  \hspace{1cm}
  \begin{subfigure}
    \centering
    \begin{tikzcd}[column sep=large, row sep=large]
      W(Wa)
      & Wa \arrow[l, "W\ \mathit{coa}"'] \\
      Wa \arrow[u, "\delta_a"]
      & a \arrow[u, "\mathit{coa}"] \arrow[l, "\mathit{coa}"']
    \end{tikzcd}
  \end{subfigure}
\end{figure}

\noindent
ここで$\mathit{coa}$は余代数の余評価射で、その担い手は$a$です: 
\[\mathit{coa} \Colon a \to W a\]
そして$\varepsilon$と$\delta$は余モナドを定義する二つの自然変換です (Haskellでは、それらのコンポーネントは\code{extract}と\code{duplicate}と呼ばれます) 。

これらの余代数から$\cat{C}$への明らかな忘却関手$U^W$があります。これは単に余評価を忘れます。右随伴$F^W$を考えます。
\[U^W \dashv F^W\]
忘却関手の右随伴は余自由関手と呼ばれます。$F^W$は$\cat{C}$の対象$a$に対して、余代数$(W a, \delta_a)$を割り当てます。随伴は合成$U^W \circ F^W$として元の余モナドを再生産します。

同様に、余Kleisli圏を構成し、対応する随伴から余モナドを再生成することができます。

\section{レンズ}

レンズに関する議論に戻りましょう。レンズは余代数として書くことができます: 
\[\mathit{coalg}_s \Colon a \to \mathit{Store}\ s\ a\]
関手$\mathit{Store}\ s$に対して: 

\src{snippet07}
この余代数は、関数のペアとしても表現できます: 
\begin{align*}
  \mathit{set} & \Colon a \to s \to a \\
  \mathit{get} & \Colon a \to s
\end{align*}
 (ここで$a$は「全部」と考え、$s$はそれの「小さい」部分としてください。) このペアの観点から、私たちは持っています: 
\[\mathit{coalg}_s\ a = \mathit{Store}\ (\mathit{set}\ a)\ (\mathit{get}\ a)\]
ここで、$a$は型$a$の値です。部分適用された\code{set}は関数$s \to a$です。

また、$\mathit{Store}\ s$が余モナドであることも知っています: 

\src{snippet08}
問題は、レンズがこの余モナドのための余代数である条件は何かということです。最初の整合性条件: 
\[\varepsilon_a \circ \mathit{coalg} = \idarrow[a]\]
は次のように翻訳されます: 
\[\mathit{set}\ a\ (\mathit{get}\ a) = a\]
これはレンズの規則で、構造体$a$のフィールドを以前の値に設定しても何も変わらないという事実を表しています。

二番目の条件: 
\[\mathit{fmap}\ \mathit{coalg} \circ \mathit{coalg} = \delta_a \circ \mathit{coalg}\]
はもう少し作業が必要です。まず、\code{Store}関手の\code{fmap}の定義を思い出してください: 

\src{snippet09}
\code{fmap coalg}を\code{coalg}の結果に適用すると、私たちは持っています: 

\src{snippet10}
一方、\code{duplicate}を\code{coalg}の結果に適用すると、生じるのは: 

\src{snippet11}
これら二つの式が等しいためには、\code{Store}の下の二つの関数が任意の\code{s}に作用するとき等しくなければなりません: 

\src{snippet12}
\code{coalg}を展開すると、私たちは持っています: 

\src{snippet13}
これは残りの二つのレンズ則に相当します。最初の規則: 

\src{snippet14}
は、フィールドの値を二回設定するのと一回設定するのが同じであると言っています。二番目の規則: 

\src{snippet15}
は、$s$に設定されたフィールドの値を取得すると、$s$が戻ってくると言っています。

言い換えれば、行儀の良いレンズは確かに\code{Store}関手のための余モナド余代数です。

\section{チャレンジ}

\begin{enumerate}
  \tightlist
  \item
        自由関手
        $F \Colon C \to C^T$の射に対する作用は何か。ヒント: モナドの$\mu$の自然性条件を使用します。
  \item
        随伴を定義してください: 
        \[U^W \dashv F^W\]
  \item
        上記の随伴が元の余モナドを再現することを証明してください。
\end{enumerate}

