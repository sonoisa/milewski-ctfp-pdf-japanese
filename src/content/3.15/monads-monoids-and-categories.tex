% !TEX root = ../../ctfp-print.tex

\lettrine[lhang=0.17]{圏}{論}に関する本を終わらせる最適な場所はありません。学ぶべきことは常にあります。圏論は広大な主題です。同時に、同じテーマ、概念、パターンが何度も何度も現れることが明らかです。すべての概念がKan拡張であるという言葉があります。実際、Kan拡張を使用して極限、余極限、随伴、モナド、米田の補題などを導き出すことができます。圏の概念自体は、あらゆる抽象化レベルで現れ、モノイドやモナドの概念も同様です。どれが最も基本的かというと、実はすべてが相互関連しており、一つが次へとつながる終わりのない抽象化のサイクルになっています。これらの相互接続を示すことが、この本を終わらせる良い方法かもしれません。

\section{双圏}

圏論の最も難しい側面の一つは、視点の絶え間ない切り替えです。たとえば、集合の圏を考えてみましょう。私たちは、要素の観点から集合を定義することに慣れています。空集合には要素がありません。単集合には要素が一つあります。二つの集合の直積は、ペアの集合です、等々。しかし、集合の圏$\Set$について話すとき、集合の内容を忘れて、その間の射 (矢印) に集中するように求めました。たまには、特定の普遍構成が$\Set$の要素の観点から何を記述しているのかを覗き見ることが許されていました。終対象は一つの要素を持つ集合であることがわかります、等々。しかし、これらは単なる健全性チェックでした。

関手は圏の写像として定義されます。写像を圏の射として考えるのは自然なことです。関手は圏の圏 (小さな圏であれば、サイズに関する質問を避けることができます) の射としてわかりました。関手を射として扱うことにより、圏の内部 (その対象と射) に対するその作用の情報を放棄することになります。これは、集合の要素に対する関数の作用の情報を放棄するのと同じです、それを$\Set$の射として扱うときです。しかし、任意の二つの圏間の関手もまた圏を形成します。今度は、一つの圏の射だったものを、別の圏の対象として考えるように求められます。関手圏では、関手は対象であり、自然変換は射です。同じものが、一つの圏では射であり、別の圏では対象であることがわかりました。対象を名詞、射を動詞とする素朴な視点は成り立ちません。

二つの視点を切り替える代わりに、一つに統合しようとすることができます。こうして$\cat{2}$-圏の概念が得られます。ここでは、対象を$0$-セル、射を$1$-セル、射間の射を$2$-セルと呼びます。

\begin{figure}[H]
  \centering
  \includegraphics[width=0.35\textwidth]{images/twocat.png}
  \caption{$0$-セル $a, b$;$1$-セル $f, g$;そして$2$-セル $\alpha$.}
\end{figure}

\noindent
圏の圏$\Cat$はすぐに思いつく例です。ここでは、圏が$0$-セル、関手が$1$-セル、自然変換が$2$-セルです。$\cat{2}$-圏の規則は、任意の二つの$0$-セル間の$1$-セルが圏を形成することを教えてくれます (言い換えると、$\cat{C}(a, b)$はホム集合ではなくホム圏です) 。これは、任意の二つの圏間の関手が関手圏を形成するという以前の主張とよく合います。

特に、任意の$0$-セルからそれ自身への$1$-セルも圏、ホム圏$\cat{C}(a, a)$を形成します。しかし、この圏はさらに多くの構造を持っています。$\cat{C}(a, a)$のメンバーは、$\cat{C}$の射として、または$\cat{C}(a, a)$の対象として見ることができます。射として、それらは互いに合成することができます。しかし、対象として見たとき、合成は対象のペアから対象への写像となります。実際、それは積―正確にはテンソル積―と非常に似ています。このテンソル積には単位があります: 恒等$1$-セルです。実際には、任意の$\cat{2}$-圏のホム圏$\cat{C}(a, a)$は、$1$-セルの合成として定義されたテンソル積を持つモノイダル圏と自動的になります。結合則と単位則は、対応する圏の規則から単純に導かれます。

さて、このことが我々の標準的な$\cat{2}$-圏$\Cat$の例で何を意味するのかを見てみましょう。ホム圏$\Cat(a, a)$は、$a$の上の自己関手の圏です。自己関手の合成は、テンソル積の役割を果たします。恒等関手は、この積に対する単位です。以前、自己関手がモノイダル圏を形成することを見ました (これはモナドの定義に使用されましたが) 、しかし今、これがより一般的な現象であることがわかります: 任意の$\cat{2}$-圏の自己$1$-セルはモノイダル圏を形成します。後で、モナドを一般化するときに、これに戻ってきます。

一般的なモノイダル圏では、モノイド則が厳密に満たされる必要はないことを覚えているかもしれません。単位則と結合則が同型まで満たされれば十分でした。$\cat{2}$-圏では、$\cat{C}(a, a)$のモノイダル則は$1$-セルの合成則から導かれます。これらの規則は厳密なので、常に厳密なモノイダル圏を得ます。ただし、これらの規則を緩和することも可能です。たとえば、恒等$1$-セル$\idarrow[a]$と別の$1$-セル、$f \Colon a \to b$の合成が、$f$と等しいという代わりに、$f$と同型であると言うことができます。$1$-セルの同型は$2$-セルを使用して定義されます。言い換えると、以下のような$2$-セルがあります。
\[\rho \Colon f \circ \idarrow[a] \to f\]
それは逆を持っています。

\begin{figure}[H]
  \centering
  \includegraphics[width=0.35\textwidth]{images/bicat.png}
  \caption{双圏の恒等性は同型 (可逆な$2$-セル $\rho$) まで成立します。}
\end{figure}

\noindent
左恒等性と結合則についても同じことができます。このような緩和された$\cat{2}$-圏は双圏と呼ばれます (ここでは省略しますが、いくつかの追加の整合性規則があります) 。

予想通り、双圏の自己$1$-セルは非厳密な規則を持つ一般的なモノイダル圏を形成します。

双圏の興味深い例は、スパンの圏です。二つの対象$a$と$b$の間のスパンは、対象$x$と一対の射のペアです: 
\begin{gather*}
  f \Colon x \to a \\
  g \Colon x \to b
\end{gather*}

\begin{figure}[H]
  \centering
  \includegraphics[width=0.35\textwidth]{images/span.png}
\end{figure}

\noindent
圏論的な積の定義にスパンを使用したことを思い出すかもしれません。ここでは、双圏の$1$-セルとしてスパンを見てみたいと思います。スパンの合成を定義する最初のステップはです。隣接するスパンを持っていると仮定します: 
\begin{gather*}
  f' \Colon y \to b \\
  g' \Colon y \to c
\end{gather*}

\begin{figure}[H]
  \centering
  \includegraphics[width=0.5\textwidth]{images/compspan.png}
\end{figure}

\noindent
合成は第三のスパンであり、その頂点は何らかの$z$です。それに対する最も自然な選択は、$g$と$f'$の引き戻しです。引き戻しとは、$z$と二つの射のペアです: 
\begin{align*}
  h  & \Colon z \to x \\
  h' & \Colon z \to y
\end{align*}
そのような: 
\[g \circ h = f' \circ h'\]
これは、すべてのそのような対象の中で普遍的なものです。

\begin{figure}[H]
  \centering
  \includegraphics[width=0.5\textwidth]{images/pullspan.png}
\end{figure}

\noindent
今のところ、集合の圏におけるスパンに集中しましょう。この場合、引き戻しは単にペア$(p, q)$の集合です$x \times y$からで、次のようになります: 
\[g\ p = f'\ q\]
二つのスパンが同じ終点を共有している場合のスパン間の射は、それらの頂点間の射$h$として定義され、適切な三角形が可換であることが求められます。

\begin{figure}[H]
  \centering
  \includegraphics[width=0.4\textwidth]{images/morphspan.png}
  \caption{スパン内の$2$-セル $\cat{Span}$.}
\end{figure}

\noindent
要約すると、双圏$\cat{Span}$では: $0$-セルは集合、$1$-セルはスパン、$2$-セルはスパン射です。恒等$1$-セルは、すべての三つの対象が同じで、二つの射が恒等である退化したスパンです。

以前にも双圏の例を見ました: プロ関手の双圏$\cat{Prof}$では、0-セルは圏、1-セルはプロ関手、2-セルは自然変換です。プロ関手の合成は余エンドによって与えられました。

\section{モナド}

この時点で、モナドが自己関手の圏のモノイドとしての定義にかなり慣れているはずです。双圏$\Cat$の小さいホム圏の自己$1$-セルのみならず、モナドの新しい理解を持ってこの定義を再訪しましょう。自己関手の圏はモノイダル圏です: テンソル積は自己関手の合成から来ます。モノイドは、ここでは終関手$T$を持つモノイダル圏の対象として定義されます—二つの射と共に。射は自己関手間の射です、すなわち自然変換です。一つの射はモノイダル単位—恒等自己関手—を$T$に写します: 
\[\eta \Colon I \to T\]
二つ目の射は$T \otimes T$のテンソル積を$T$に写します。テンソル積は自己関手の合成によって与えられるので、私たちは得ます: 
\[\mu \Colon T \circ T \to T\]

\begin{figure}[H]
  \centering
  \includegraphics[width=0.3\textwidth]{images/monad.png}
\end{figure}

\noindent
これらはモナドを定義する二つの操作として認識します (Haskellでは\code{return}と\code{join}と呼ばれます) 、そしてモノイド則はモナド則になります。

さて、この定義から自己関手の言及をすべて取り除きましょう。双圏$\cat{C}$を始めにして、それにおける$0$-セル$a$を選びます。以前に見たように、ホム圏$\cat{C}(a, a)$はモノイダル圏です。したがって、$\cat{C}(a, a)$におけるモノイドを定義することができます。$1$-セル、$T$を選び、そして二つの$2$-セルを選びます: 
\begin{align*}
  \eta & \Colon I \to T         \\
  \mu  & \Colon T \circ T \to T
\end{align*}
モノイド則を満たすようにします。これをモナドと呼びます。

\begin{figure}[H]
  \centering
  \includegraphics[width=0.3\textwidth]{images/bimonad.png}
\end{figure}

\noindent
これは、$0$-セル、$1$-セル、そして$2$-セルを使ってのモナドのはるかに一般的な定義です。それは双圏$\Cat$に適用されるとき通常のモナドに簡約されます。しかし、他の双圏で何が起こるかを見てみましょう。

双圏$\cat{Span}$におけるモナドを構成しましょう。$0$-セルを選びます。これは、すぐに明らかになる理由から、私が$\mathit{Ob}$と呼ぶことにする集合です。次に、自己$1$-セルを選びます: $\mathit{Ob}$から$\mathit{Ob}$へのスパンです。頂点にある集合を$\mathit{Ar}$と呼びます。二つの関数で装備されています: 
\begin{align*}
  \mathit{dom} & \Colon \mathit{Ar} \to \mathit{Ob} \\
  \mathit{cod} & \Colon \mathit{Ar} \to \mathit{Ob}
\end{align*}

\begin{figure}[H]
  \centering
  \includegraphics[width=0.3\textwidth]{images/spanmonad.png}
\end{figure}

\noindent
集合$\mathit{Ar}$の要素を「射」と呼びましょう。また、$\mathit{Ob}$の要素を「対象」と呼ぶことにすれば、どこに行くのかヒントが得られるかもしれません。二つの関数$\mathit{dom}$と$\mathit{cod}$は、「射」に始域と終域を割り当てます。

私たちのスパンをモナドにするためには、二つの$2$-セル、$\eta$と$\mu$が必要です。この場合のモノイダル単位は、頂点が$\mathit{Ob}$にある$\mathit{Ob}$から$\mathit{Ob}$への自明なスパンです、そして二つの恒等関数です。$2$-セル$\eta$は、頂点$\mathit{Ob}$と$\mathit{Ar}$の間の関数です。言い換えると、$\eta$は「対象」ごとに「射」を割り当てます。$\cat{Span}$内の$2$-セルは、可換性条件を満たさなければなりません―この場合は: 
\begin{align*}
  \mathit{dom} & \circ \eta = \id \\
  \mathit{cod} & \circ \eta = \id
\end{align*}

\begin{figure}[H]
  \centering
  \includegraphics[width=0.4\textwidth]{images/spanunit.png}
\end{figure}

\noindent
コンポーネントで表すと、これは以下のようになります: 
\[\mathit{dom}\ (\eta\ \mathit{ob}) = \mathit{ob} = \mathit{cod}\ (\eta\ \mathit{ob})\]
ここで、$\mathit{ob}$は$\mathit{Ob}$の「対象」です。言い換えると、$\eta$は始域と終域がその「対象」である「射」をすべての「対象」に割り当てます。この特別な「射」を「恒等射」と呼びましょう。

二つ目の$2$-セル$\mu$は、スパン$\mathit{Ar}$の自己合成に作用します。合成は引き戻しとして定義されるので、その要素は$\mathit{Ar}$の要素のペア --- 「射」のペア$(a_1, a_2)$です。引き戻し条件は次のようになります: 
\[\mathit{cod}\ a_1 = \mathit{dom}\ a_2\]
私たちは$a_1$と$a_2$が「合成可能」であると言います。なぜなら、一方の始域が他方の終域です。

\begin{figure}[H]
  \centering
  \includegraphics[width=0.5\textwidth]{images/spanmul.png}
\end{figure}

\noindent
$2$-セル$\mu$は、合成可能な射のペア$(a_1, a_2)$を集合$\mathit{Ar}$の単一の射$a_3$に写します。言い換えると$\mu$は射の合成を定義します。

モナド則が射の恒等性と結合性の規則に対応することを確認するのは簡単です。私たちはちょうど圏 (小さな圏です、対象と射が集合を形成します) を定義しました。

つまり、圏はただの双圏のスパンにおけるモナドです。

この結果の驚くべき点は、それが圏をモナドやモノイドなどの他の代数的構造と同じ立場に置くことです。圏であることには何も特別なことはありません。それは単に二つの集合と四つの関数です。実際、対象のために別の集合が必要とされるわけではなく、対象は恒等射 (一対一の対応関係にあります) と同一視することができます。ですから、それは実際にはただの集合といくつかの関数です。数学のあらゆる分野で圏論が果たす中心的な役割を考えると、これは非常に謙虚な実現です。

\section{チャレンジ}

\begin{enumerate}
  \tightlist
  \item
        双圏における自己$1$-セルの合成として定義されるテンソル積の単位則と結合則を導出してください。
  \item
        $\cat{Span}$内のモナドのモナド則が、結果の圏における恒等性と結合性の規則に対応することを確認してください。
  \item
        $\cat{Prof}$内のモナドが対象上同一の関手であることを示してください。
  \item
        $\cat{Span}$内のモナドのモナド代数とは何か?
\end{enumerate}

\section{参考文献}
\begin{enumerate}
  \tightlist
  \item
        \urlref{https://graphicallinearalgebra.net/2017/04/16/a-monoid-is-a-category-a-category-is-a-monad-a-monad-is-a-monoid/}{Paweł Sobocińskiのブログ}。
\end{enumerate}

