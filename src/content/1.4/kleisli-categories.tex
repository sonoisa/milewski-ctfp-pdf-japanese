% !TEX root = ../../ctfp-print.tex

\lettrine[lhang=0.17]{型}{と}純粋関数を圏としてモデル化する方法を見てきました。また、非純粋関数や副作用を圏論でモデル化する方法があることにも触れました。実行中にログやトレースを生成する関数の例を見てみましょう。このようなことは、命令型言語では、おそらく何らかのグローバル状態を変更することによって実装されるでしょう。例えば:

\begin{snip}{cpp}
string logger;

bool negate(bool b) {
    logger += "Not so! ";
    return !b;
}
\end{snip}
この関数が純粋ではないことは、メモ化されたバージョンがログを生成できないためです。この関数には副作用があります。

現代のプログラミングでは、可能な限りグローバルな可変状態から遠ざかるようにします --- 特に並行処理の複雑さのためです。そして、このようなコードをライブラリに入れることはありません。

幸いなことに、この関数を純粋にすることは可能です。ログを明示的に渡すだけです。引数に文字列を追加し、通常の出力と更新されたログを含む文字列とをペアにしましょう:

\begin{snip}{cpp}
pair<bool, string> negate(bool b, string logger) {
    return make_pair(!b, logger + "Not so! ");
}
\end{snip}
この関数は純粋で、副作用がなく、同じ引数で呼ばれるたびに同じペアを返し、必要に応じてメモ化できます。しかし、ログの累積的な性質を考えると、与えられた呼び出しに至るまでのすべての可能な履歴をメモ化する必要があります。別々のメモエントリが必要になります:

\begin{snip}{cpp}
negate(true, "It was the best of times. ");
\end{snip}
そして

\begin{snip}{cpp}
negate(true, "It was the worst of times. ");
\end{snip}
などがあります。

これはまた、ライブラリ関数のインターフェースとしてあまり良くありません。呼び出し元は戻り値の文字列を無視することができますが、入力として文字列を渡すことを強制されます。これは不便かもしれません。

もっと侵入的でない方法で同じことをする方法はありませんか?関心の分離はできませんか?この単純な例では、関数\code{negate}の主な目的は、あるブール値を別のブール値に変換することです。ログは二次的なものです。確かに、記録されるメッセージは関数に特有のものですが、メッセージを一つの連続したログに集約するタスクは別の問題です。私たちはまだ関数が文字列を生成することを望んでいますが、ログの生成からはそれを解放したいと思います。そこで妥協案です:

\begin{snip}{cpp}
pair<bool, string> negate(bool b) {
    return make_pair(!b, "Not so! ");
}
\end{snip}
アイディアは、ログが関数呼び出しの間に集約されるというものです。

これをどのように実行するかを見るために、もう少し現実的な例に切り替えましょう。文字列から文字列への関数が一つあり、小文字を大文字に変換します:

\begin{snip}{cpp}
string toUpper(string s) {
    string result;
    int (*toupperp)(int) = &toupper; // toupper is overloaded
    transform(begin(s), end(s), back_inserter(result), toupperp);
    return result;
}
\end{snip}
そして、文字列を空白の境界で文字列のベクトルに分割する別の関数があります:

\begin{snip}{cpp}
vector<string> toWords(string s) {
    return words(s);
}
\end{snip}
実際の作業は補助関数\code{words}で行われます:

\begin{snip}{cpp}
vector<string> words(string s) {
    vector<string> result{""};
    for (auto i = begin(s); i != end(s); ++i)
    {
        if (isspace(*i))
            result.push_back(""); 
        else
            result.back() += *i;
    }
    return result;
}
\end{snip}
私たちは関数\code{toUpper}と\code{toWords}を変更して、それらが通常の戻り値の上にメッセージ文字列を載せるようにしたいと思っています。

\begin{figure}[H]
  \centering
  \includegraphics[width=0.3\textwidth]{images/piggyback.jpg}
\end{figure}
\noindent
これらの関数の戻り値を「飾る」ために、任意の型\code{A}の値と文字列のペアをカプセル化するテンプレート\code{Writer}を定義することによって、一般的な方法で行います:

\begin{snip}{cpp}
template<class A>
using Writer = pair<A, string>;
\end{snip}
ここに飾られた関数があります:

\begin{snip}{cpp}
Writer<string> toUpper(string s) {
    string result;
    int (*toupperp)(int) = &toupper;
    transform(begin(s), end(s), back_inserter(result), toupperp);
    return make_pair(result, "toUpper "); 
}

Writer<vector<string>> toWords(string s) { 
    return make_pair(words(s), "toWords ");
}
\end{snip}
これらの二つの関数を、文字列を大文字に変換し、単語に分割し、その行動のログを生成する別の飾られた関数に合成したいと思っています。それを行う方法はこうです:

\begin{snip}{cpp}
Writer<vector<string>> process(string s) {
    auto p1 = toUpper(s);
    auto p2 = toWords(p1.first);
    return make_pair(p2.first, p1.second + p2.second);
}
\end{snip}
目標を達成しました: ログの集約はもはや個々の関数の懸念事項ではありません。それらは自分自身のメッセージを生成し、それから外部的に大きなログに連結されます。

こうして書かれた全プログラムを想像してみてください。これは繰り返し、間違いやすいコードの悪夢です。しかし、私たちはプログラマーです。繰り返しコードをどう扱うか知っています: それを抽象化します!ただし、これはごく普通の抽象化ではありません --- 関数合成自体を抽象化する必要があります。しかし、合成は圏論の本質ですので、もっとコードを書く前に、問題を圏論的な視点から分析しましょう。

\section{Writer圏}

関数の戻り値を飾ることで、何らかの追加機能を背負わせるというアイデアは非常に有益です。これに関する多くの例を見ることになるでしょう。出発点は私たちの通常の型と関数の圏です。型を対象として残しますが、飾られた関数をもつ射に再定義します。

例えば、\code{int}から\code{bool}への関数\code{isEven}を飾りたいとします。それを、ペアを返す飾られた関数で表現される射に変えます。重要な点は、この射がまだ\code{int}と\code{bool}の対象間の射と見なされることですが、飾られた関数がペアを返します:

\begin{snip}{cpp}
pair<bool, string> isEven(int n) {
    return make_pair(n % 2 == 0, "isEven ");
}
\end{snip}
圏の規則によれば、私たちはこの射を、\code{bool}の対象から何らかのものへの別の射と合成することができるはずです。特に、以前の\code{negate}と合成することができるはずです:

\begin{snip}{cpp}
pair<bool, string> negate(bool b) {
    return make_pair(!b, "Not so! ");
}
\end{snip}
明らかに、これら二つの射を通常の関数のように合成することはできません。なぜなら、入出力の不一致のためです。その合成は次のようになるはずです:

\begin{snip}{cpp}
pair<bool, string> isOdd(int n) {
    pair<bool, string> p1 = isEven(n);
    pair<bool, string> p2 = negate(p1.first);
    return make_pair(p2.first, p1.second + p2.second);
}
\end{snip}
そこで、この新しい圏での二つの射の合成のレシピです: 

\begin{enumerate}
  \tightlist
  \item
        最初の射に対応する飾られた関数を実行する
  \item
        結果ペアの最初のコンポーネントを抽出し、それを第二の射に対応する飾られた関数に渡す
  \item
        最初の結果の二番目のコンポーネント (文字列) と、第二の結果の二番目のコンポーネント (文字列) を連結する
  \item
        最終結果の最初のコンポーネントと連結された文字列を組み合わせた新しいペアを返す
\end{enumerate}

C++でこの合成を高階関数として抽象化したい場合、私たちは、私たちの圏の3つの対象に対応する3つの型でパラメータ化されたテンプレートを使わなければなりません。それは、私たちの規則に従って合成可能な二つの飾られた関数を取り、第三の飾られた関数を返します:

\begin{snip}{cpp}
template<class A, class B, class C>
function<Writer<C>(A)> compose(function<Writer<B>(A)> m1,
                               function<Writer<C>(B)> m2)
{
    return [m1, m2](A x) {
        auto p1 = m1(x);
        auto p2 = m2(p1.first);
        return make_pair(p2.first, p1.second + p2.second); 
    };
}
\end{snip}
これで、以前の例に戻って、この新しいテンプレートを使って\code{toUpper}と\code{toWords}の合成を実装できます。

\begin{snip}{cpp}
Writer<vector<string>> process(string s) { 
    return compose<string, string, vector<string>>(toUpper, toWords)(s);
}
\end{snip}
型の渡しに関するノイズはまだ多いです。しかし、一般化されたラムダ関数と戻り型の推論をサポートしているC++14準拠のコンパイラを使用していれば、これを回避できます (このコードのクレジットはEric Nieblerにあります) :

\begin{snip}{cpp}
auto const compose = [](auto m1, auto m2) { 
    return [m1, m2](auto x) { 
        auto p1 = m1(x);
        auto p2 = m2(p1.first);
        return make_pair(p2.first, p1.second + p2.second);
    };
};
\end{snip}
この新しい定義では、\code{process}の実装が簡単になります:

\begin{snip}{cpp}
Writer<vector<string>> process(string s) {
    return compose(toUpper, toWords)(s);
}
\end{snip}
しかし、まだ終わっていません。私たちは新しい圏での合成を定義しましたが、恒等射は何でしょうか?これらは通常の恒等関数ではありません!それらは型Aから型Aへ戻る射であり、それは以下の形式の飾られた関数です:

\begin{snip}{cpp}
Writer<A> identity(A);
\end{snip}
それらは合成に関して単位要素として振る舞う必要があります。合成の定義を見ると、恒等射は引数を変更せずに渡し、ログに空文字列を追加するだけでなければならないことがわかります:

\begin{snip}{cpp}
template<class A> Writer<A> identity(A x) {
    return make_pair(x, "");
}
\end{snip}
この新しく定義された圏が確かに正当な圏であることを簡単に確認できます。特に、私たちの合成は自明に結合的です。各ペアの最初のコンポーネントで何が起こっているかを追えば、それはただの通常の関数合成であり、それは結合的です。二番目のコンポーネントは連結されており、連結もまた結合的です。

洞察力のある読者は、この構造を文字列モノイドだけでなく、任意のモノイドに一般化するのは簡単だと気づくかもしれません。\code{compose}内で\code{mappend}を使用し、\code{identity}内で\code{mempty}を使用します (\code{+}と\code{""}の代わりに)。ログに文字列のみを使用することに制限する理由は実際にはありません。良いライブラリの作者は、ライブラリが機能するために必要な最小限の制約を特定できるはずです --- ここでは、ログライブラリの唯一の要件は、ログがモノイダルな特性を持つことです。

\section{HaskellでのWriter}

Haskellでの同じことは少し簡潔であり、コンパイラからも多くの助けを得られます。まず\code{Writer}型を定義しましょう:

\src{snippet01}
ここで私は単なる型エイリアスを定義しています。これはC++の\code{typedef} (または\code{using}) と同等です。\code{Writer}型は型変数\code{a}によってパラメータ化されており、\code{a}と\code{String}のペアに相当します。ペアの構文は最小限です: ただ二つのアイテムをカンマで区切って括弧に入れるだけです。

私たちの射は任意の型からある\code{Writer}型への関数です:

\src{snippet02}
合成は時々「魚」と呼ばれる面白い中置演算子として宣言します:

\src{snippet03}
これはそれぞれが独自の関数である二つの引数の関数です。そして、関数を返します。最初の引数は\code{(a -> Writer b)}型で、二番目は\code{(b -> Writer c)}型であり、結果は\code{(a -> Writer c)}型です。

ここにその中置演算子の定義があります --- 二つの引数\code{m1}と\code{m2}は魚のシンボルの両側に現れます:

\src{snippet04}
結果は一つの引数\code{x}のラムダ関数です。ラムダはバックスラッシュで書かれています --- これは、足を切断されたギリシャ文字の$\lambda$と考えてください。

\code{let}式を使うと、補助変数を宣言できます。ここでは、\code{m1}への呼び出しの結果が変数のペア\code{(y, s1)}にパターンマッチされ、最初のパターンからの引数\code{y}での\code{m2}への呼び出しの結果が\code{(z, s2)}

にマッチされます。

Haskellでは、ペアに対するパターンマッチを行うことが一般的です。C++でアクセッサを使用した場合のようには行いません。それ以外には、二つの実装間にかなり直接的な対応関係があります。

\code{let}式の全体的な値は、その\code{in}節で指定されます。ここでは、それは\code{z}の最初のコンポーネントと二つの文字列の連結\code{s1++s2}の二番目のコンポーネントのペアです。

私たちの圏のための恒等射も定義しますが、後で明らかになる理由から、それを\code{return}と呼びます。

\src{snippet05}
完全性のために、飾られた関数\code{upCase}と\code{toWords}のHaskellバージョンを持ちましょう:

\src{snippet06}
関数\code{map}はC++の\code{transform}に相当します。それは文字関数\code{toUpper}を文字列\code{s}に適用します。補助関数\code{words}は標準のPreludeライブラリで定義されています。

最後に、二つの関数の合成は魚演算子の助けを借りて達成されます:

\src{snippet07}

\section{Kleisli圏}

私がこの圏をその場で発明したわけではないと推測したかもしれません。これは、モナドに基づくいわゆるKleisli圏の一例です。私たちはまだモナドについて議論する準備ができていませんが、それらが何を達成できるかの一端をお見せしたかったのです。私たちの限られた目的のために、Kleisli圏は、基礎となるプログラミング言語の型を対象として持ちます。型$A$から型$B$への射は、特定の装飾を使用して$B$から派生した型への関数です。各Kleisli圏は、そのような射の合成方法と、その合成に関しての恒等射を定義します。 (後に見るように、「装飾」という漠然とした用語は、圏内の自己関手の概念に対応します。) 

この章で使用した特定のモナドは、\newterm{writerモナド}と呼ばれ、関数の実行をロギングまたはトレースするために使用されます。これはまた、純粋な計算に効果を埋め込むためのより一般的なメカニズムの例でもあります。以前に見たように、我々はプログラミング言語の型と関数を (通常のように底を無視して) 集合の圏でモデル化することができました。ここでは、射が飾られた関数によって表され、それらの合成がただ一つの関数の出力を別の関数の入力に渡す以上のことを行う、少し異なる圏へとこのモデルを拡張しました。合成自体で遊ぶことができる自由度がもう一つあります。それはちょうど、副作用を使って命令型言語で伝統的に実装されているプログラムに単純な表示的意味論を与えることができる自由度です。

\section{チャレンジ}

引数のすべての可能な値に対して定義されていない関数は、部分関数と呼ばれます。それは数学的な意味での本当の関数ではないので、標準的な圏論的型には当てはまりません。しかし、それは\code{optional}という装飾された型を返す関数によって表現することができます:

\begin{snip}{cpp}
template<class A> class optional {
    bool _isValid;
    A _value;
public: 
    optional()    : _isValid(false) {}
    optional(A v) : _isValid(true), _value(v) {}
    bool isValid() const { return _isValid; }
    A value() const { return _value; }
};
\end{snip}
例えば、次のような\code{safe\_root}関数の実装があります:

\begin{snip}{cpp}
optional<double> safe_root(double x) {
    if (x >= 0) return optional<double>{sqrt(x)}; 
    else return optional<double>{};
}
\end{snip}
ここで、チャレンジです:

\begin{enumerate}
  \tightlist
  \item
        部分関数のためのKleisli圏を構成します (合成と恒等射を定義します)。
  \item
        それがゼロでない場合にその引数の有効な逆数を返す、飾られた関数\code{safe\_reciprocal}を実装します。
  \item
        関数\code{safe\_root}と\code{safe\_reciprocal}を合成して、可能であれば\code{sqrt(1/x)}を計算する\code{safe\_root\_reciprocal}を実装します。
\end{enumerate}