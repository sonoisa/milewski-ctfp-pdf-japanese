% !TEX root = ctfp-print.tex

\input{half-title}

\frontmatter
\tableofcontents

% !TEX root = ../ctfp-print.tex

\ifdefined\OPTCustomLanguage{%
    \chapter*{編集者による注記}
    \addcontentsline{toc}{chapter}{編集者による注記}
    % !TEX root = ../ctfp-print.tex

\ifdefined\OPTCustomLanguage{%
    \chapter*{編集者による注記}
    \addcontentsline{toc}{chapter}{編集者による注記}
    % !TEX root = ../ctfp-print.tex

\ifdefined\OPTCustomLanguage{%
    \chapter*{編集者による注記}
    \addcontentsline{toc}{chapter}{編集者による注記}
    \input{content/\OPTCustomLanguage/editor-note}
  }
\fi
  }
\fi
  }
\fi
\chapter*{序文}
\addcontentsline{toc}{chapter}{序文}
\label{Preface}
\subfile{content/0.0/Preface}

\mainmatter

\part*{第1部}
\addcontentsline{toc}{part}{第1部}

\chapter{圏: 合成の本質}\label{category-the-essence-of-composition}
\subfile{content/1.1/category-the-essence-of-composition}

\chapter{型と関数}\label{types-and-functions}
\subfile{content/1.2/types-and-functions}

\chapter{圏の大小}\label{categories-great-and-small}
\subfile{content/1.3/categories-great-and-small}

\chapter{Kleisli圏}\label{kleisli-categories}
\subfile{content/1.4/kleisli-categories}

\chapter{積と余積}\label{products-and-coproducts}
\subfile{content/1.5/products-and-coproducts}

\chapter{単純代数的データ型}\label{simple-algebraic-data-types}
\subfile{content/1.6/simple-algebraic-data-types}

\chapter{関手}\label{functors}
\subfile{content/1.7/functors}

\chapter{関手性}\label{functoriality}
\subfile{content/1.8/functoriality}

\chapter{関数型}\label{function-types}
\subfile{content/1.9/function-types}

\chapter{自然変換}\label{natural-transformations}
\subfile{content/1.10/natural-transformations}

\part*{第2部}
\addcontentsline{toc}{part}{第2部}

\chapter{宣言的プログラミング}\label{declarative-programming}
\subfile{content/2.1/declarative-programming}

\chapter{極限と余極限}\label{limits-and-colimits}
\subfile{content/2.2/limits-and-colimits}

\chapter{自由モノイド}\label{free-monoids}
\subfile{content/2.3/free-monoids}

\chapter{表現可能関手}\label{representable-functors}
\subfile{content/2.4/representable-functors}

\chapter{米田の補題}\label{the-yoneda-lemma}
\subfile{content/2.5/the-yoneda-lemma}

\chapter{米田埋め込み}\label{yoneda-embedding}
\subfile{content/2.6/yoneda-embedding}

\part*{第3部}
\addcontentsline{toc}{part}{第3部}

\chapter{すべては射に関することです}\label{all-about-morphisms}
\subfile{content/3.1/its-all-about-morphisms}

\chapter{随伴}\label{adjunctions}
\subfile{content/3.2/adjunctions}

\chapter{自由/忘却随伴}\label{free-forgetful-adjunctions}
\subfile{content/3.3/free-forgetful-adjunctions}

\chapter{モナド: プログラマによる定義}\label{monads-programmers-definition}
\subfile{content/3.4/monads-programmers-definition}

\chapter{モナドと効果}\label{monads-and-effects}
\subfile{content/3.5/monads-and-effects}

\chapter{モナドを圏論的に}\label{monads-categorically}
\subfile{content/3.6/monads-categorically}

\chapter{余モナド}\label{comonads}
\subfile{content/3.7/comonads}

\chapter{F-代数}\label{f-algebras}
\subfile{content/3.8/f-algebras}

\chapter{モナドに関する代数}\label{algebras-for-monads}
\subfile{content/3.9/algebras-for-monads}

\chapter{エンドと余エンド}\label{ends-and-coends}
\subfile{content/3.10/ends-and-coends}

\chapter{Kan拡張}\label{kan-extensions}
\subfile{content/3.11/kan-extensions}

\chapter{豊穣圏}\label{enriched-categories}
\subfile{content/3.12/enriched-categories}

\chapter{トポス}\label{topoi}
\subfile{content/3.13/topoi}

\chapter{Lawvere理論}\label{lawvere-theories}
\subfile{content/3.14/lawvere-theories}

\chapter{モナド、モノイド、そして圏}\label{monads-monoids-categories}
\subfile{content/3.15/monads-monoids-and-categories}

\backmatter

\appendix
\addcontentsline{toc}{part}{付録}
\input{index}

\makeatletter\@openrightfalse
\chapter*{謝辞}\label{acknowledgments}
\addcontentsline{toc}{chapter}{謝辞}
\noindent
数学と論理のチェックをしてくれたEdward Kmett氏とGershom Bazerman氏に感謝します。また、多くのボランティアが私の間違いを訂正し、本を改善する手助けをしてくれたことにも感謝しています。

\vspace{1.0em}
\noindent
私のC++のモノイドコンセプトコードを、彼とBjarne Stroustrup氏の最新の提案に基づいて書き直してくれたAndrew Sutton氏に感謝します。

\vspace{1.0em}
\noindent
ドラフトを読んでくれて、C++14の高度な機能を使って型推論を進める\code{compose}の賢い実装を提供してくれたEric Niebler氏に感謝します。これによって、型特性を使って同じことをしていた古風なテンプレートマジックの全セクションを削除できました。さようなら、古い魔法たち!


\chapter*{奥付}\label{colophon}
\addcontentsline{toc}{chapter}{奥付}
OCamlコードは\urlref{https://github.com/ArulselvanMadhavan/ocaml-ctfp}{Arulselvan Madhavan}により翻訳され、\urlref{http://www.mseri.me}{Marcello Seri}と\urlref{https://github.com/XVilka}{Anton Kochkov}により査読されました。

\chapter*{コピーレフトに関する通知}\label{copyleft}
\addcontentsline{toc}{chapter}{コピーレフトに関する通知}
\lettrine[lraise=-0.03,loversize=0.08]{こ}{の本}は\textbf{自由}なライセンスに従っており、
\urlref{https://www.gnu.org/philosophy/free-sw.en.html}{フリーソフトウェア}の哲学に基づいています:
この本を好きなように利用することができ、ソースは公開されています。また、この本を再配布したり、
あなた自身のバージョンを配布することもできます。つまり、印刷したり、コピーしたり、メールで送ったり、
ウェブサイトにアップロードしたり、変更したり、翻訳したり、リミックスしたり、一部を削除したり、
その上に何かを描いたりすることができます。

この本はコピーレフトです: 本を変更して自分のバージョンを配布する場合、受け取る人たちにもこれらの自由を認めなければなりません。
この本はCreative Commons Attribution-ShareAlike 4.0 International License
(\href{http://creativecommons.org/licenses/by-sa/4.0/}{\acronym{CC BY-SA 4.0}})を使用しています。

\@openrighttrue\makeatother
\afterpage{\blankpage}