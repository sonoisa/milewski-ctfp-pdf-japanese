% !TEX root = ctfp-print.tex

% !TEX root = ctfp-print.tex
\input{version}

\thispagestyle{empty}

\vspace*{80pt}

\begin{raggedleft}
  \fontsize{24pt}{24pt}\selectfont
  \textbf{Category Theory \\ for Programmers}\\
  \ifdefined\OPTCustomLanguage{%
    \vspace*{1cm}
    \small\selectfont{
      \textbf{\OPTDisplayLanguageName{} Edition}\\
      \textit{Contains code snippets in Haskell and \OPTDisplayLanguageName}\\
    }
  }
  \fi
  \vspace*{1cm}
  \fontsize{16pt}{18pt}\selectfont \textit{By } \textbf{Bartosz Milewski}\\
  \vspace{1cm}
  \fontsize{12pt}{14pt}\selectfont \textit{compiled and edited by}\\ \textbf{Igal Tabachnik}\\

\end{raggedleft}


\newpage

\vspace*{0.3\textheight}
\thispagestyle{empty}

\begin{small}
  \begin{center}

    \textsc{Category Theory for Programmers}\\

    \vspace{1.0em}
    \noindent
    Bartosz Milewski\\

    \vspace{1.26em}
    \noindent
    Version \texttt{\OPTversion}\\\today


    \vspace{1.6em}
    \noindent
    \includegraphics[width=3mm]{fig/icons/cc.pdf}
    \includegraphics[width=3mm]{fig/icons/by.pdf}
    \includegraphics[width=3mm]{fig/icons/sa.pdf}

    \vspace{0.4em}
    \noindent
    This work is licensed under a Creative Commons\\
    Attribution-ShareAlike 4.0 International License
    (\href{http://creativecommons.org/licenses/by-sa/4.0/}{\acronym{CC BY-SA 4.0}}).

    \vspace{1.26em}
    \noindent
    Converted from a series of blog posts by \href{https://bartoszmilewski.com/2014/10/28/category-theory-for-programmers-the-preface/}{Bartosz Milewski}.\\
    PDF and book compiled by \href{https://hmemcpy.com}{Igal Tabachnik}.\\
    \vspace{1.26em}
    \noindent
    \LaTeX{} source code is available on GitHub: \href{https://github.com/hmemcpy/milewski-ctfp-pdf}{https://github.com/hmemcpy/milewski-ctfp-pdf}
  \end{center}
\end{small}

\frontmatter
\tableofcontents

% !TEX root = ctfp-print.tex

\lettrine[lhang=0.17]{こ}{れは}、\emph{プログラマのための圏論}のReasonML版です。Bartosz Milewski氏のブログ記事シリーズを、美しく
組版された\acronym{PDF}、そしてハードカバーの書籍として利用できるようになり、大変な成功を収めました。多くの貢献があり、
タイプミスや誤りの修正、コードスニペットを他のプログラミング言語に翻訳する作業が行われました。

この本のこの版を紹介できることにわくわくしています。元のHaskellコードに続いて、そのReasonMLでの対応版を含んでいます。
ReasonMLコードスニペットは、ocaml-ctfpの貢献者によって提供されたOCamlスニペットから変換され、この本の形式に合わせてわずかに修正されました。

複数の言語のコードスニペットをサポートするために、外部ファイルからコードスニペットを読み込む\LaTeX{}マクロを使用しています。
これにより、元のテキストをそのままにしつつ、他の言語での本を簡単に拡張できます。そのため、テキスト内で「Haskellで」という言葉を見るたびに、心の中で「そしてReasonMLで」と言葉を足すべきです。

コードは以下のようにレイアウトされています: 元のHaskellコードに続いて、ReasonMLコードがあります。これらを区別するために、
コードスニペットはそれぞれの言語のロゴ \raisebox{-.2mm}{\includegraphics[height=.3cm]{fig/icons/haskell.png}} と \raisebox{-.2mm}{\includegraphics[height=.3cm]{fig/icons/reason.png}} の主要色を使用した縦棒で左側が囲まれています。例えば: 

\srcsnippet{content/1.1/code/haskell/snippet03.hs}{blue}{haskell}
\unskip
\srcsnippet{content/1.1/code/reason/snippet03.re}{RedOrange}{reason}
\NoIndentAfterThis

\chapter*{序文}
\addcontentsline{toc}{chapter}{序文}
\label{Preface}
\subfile{content/0.0/Preface}

\mainmatter

\part*{第1部}
\addcontentsline{toc}{part}{第1部}

\chapter{圏: 合成の本質}\label{category-the-essence-of-composition}
\subfile{content/1.1/category-the-essence-of-composition}

\chapter{型と関数}\label{types-and-functions}
\subfile{content/1.2/types-and-functions}

\chapter{圏の大小}\label{categories-great-and-small}
\subfile{content/1.3/categories-great-and-small}

\chapter{Kleisli圏}\label{kleisli-categories}
\subfile{content/1.4/kleisli-categories}

\chapter{積と余積}\label{products-and-coproducts}
\subfile{content/1.5/products-and-coproducts}

\chapter{単純代数的データ型}\label{simple-algebraic-data-types}
\subfile{content/1.6/simple-algebraic-data-types}

\chapter{関手}\label{functors}
\subfile{content/1.7/functors}

\chapter{関手性}\label{functoriality}
\subfile{content/1.8/functoriality}

\chapter{関数型}\label{function-types}
\subfile{content/1.9/function-types}

\chapter{自然変換}\label{natural-transformations}
\subfile{content/1.10/natural-transformations}

\part*{第2部}
\addcontentsline{toc}{part}{第2部}

\chapter{宣言的プログラミング}\label{declarative-programming}
\subfile{content/2.1/declarative-programming}

\chapter{極限と余極限}\label{limits-and-colimits}
\subfile{content/2.2/limits-and-colimits}

\chapter{自由モノイド}\label{free-monoids}
\subfile{content/2.3/free-monoids}

\chapter{表現可能関手}\label{representable-functors}
\subfile{content/2.4/representable-functors}

\chapter{米田の補題}\label{the-yoneda-lemma}
\subfile{content/2.5/the-yoneda-lemma}

\chapter{米田埋め込み}\label{yoneda-embedding}
\subfile{content/2.6/yoneda-embedding}

\part*{第3部}
\addcontentsline{toc}{part}{第3部}

\chapter{すべては射に関することです}\label{all-about-morphisms}
\subfile{content/3.1/its-all-about-morphisms}

\chapter{随伴}\label{adjunctions}
\subfile{content/3.2/adjunctions}

\chapter{自由/忘却随伴}\label{free-forgetful-adjunctions}
\subfile{content/3.3/free-forgetful-adjunctions}

\chapter{モナド: プログラマによる定義}\label{monads-programmers-definition}
\subfile{content/3.4/monads-programmers-definition}

\chapter{モナドと効果}\label{monads-and-effects}
\subfile{content/3.5/monads-and-effects}

\chapter{モナドを圏論的に}\label{monads-categorically}
\subfile{content/3.6/monads-categorically}

\chapter{余モナド}\label{comonads}
\subfile{content/3.7/comonads}

\chapter{F-代数}\label{f-algebras}
\subfile{content/3.8/f-algebras}

\chapter{モナドに関する代数}\label{algebras-for-monads}
\subfile{content/3.9/algebras-for-monads}

\chapter{エンドと余エンド}\label{ends-and-coends}
\subfile{content/3.10/ends-and-coends}

\chapter{Kan拡張}\label{kan-extensions}
\subfile{content/3.11/kan-extensions}

\chapter{豊穣圏}\label{enriched-categories}
\subfile{content/3.12/enriched-categories}

\chapter{トポス}\label{topoi}
\subfile{content/3.13/topoi}

\chapter{Lawvere理論}\label{lawvere-theories}
\subfile{content/3.14/lawvere-theories}

\chapter{モナド、モノイド、そして圏}\label{monads-monoids-categories}
\subfile{content/3.15/monads-monoids-and-categories}

\backmatter

\appendix
\addcontentsline{toc}{part}{付録}
\setindexprenote{\normalsize
  \begin{quote} Any inaccuracies in this index may be explained by the fact
    that it has been prepared with the help of a computer.

    ---Donald E. Knuth, \textit{Fundamental Algorithms}\\
    (Volume 1 of \textit{The Art of Computer Programming})
  \end{quote}}

\printindex

\makeatletter\@openrightfalse
\chapter*{謝辞}\label{acknowledgments}
\addcontentsline{toc}{chapter}{謝辞}
\noindent
I’d like to thank Edward Kmett and Gershom Bazerman for checking my math
and logic. I'm grateful to many volunteers who corrected my mistakes and improved the book.

\vspace{1.0em}
\noindent
I’d like to thank Andrew Sutton for rewriting my C++ monoid concept
code according to his and Bjarne Stroustrup’s latest proposal.

\vspace{1.0em}
\noindent
I'm grateful to Eric Niebler for reading the draft and providing the
clever implementation of \code{compose} that uses advanced features of
C++14 to drive type inference. I was able to cut the whole section of
old fashioned template magic that did the same thing using type traits.
Good riddance!

\chapter*{奥付}\label{colophon}
\addcontentsline{toc}{chapter}{奥付}
OCaml code translation was done by \urlref{https://github.com/ArulselvanMadhavan/ocaml-ctfp}{Arulselvan Madhavan}
and reviewed by \urlref{http://www.mseri.me}{Marcello Seri} and \urlref{https://github.com/XVilka}{Anton Kochkov}.

\chapter*{コピーレフトに関する通知}\label{copyleft}
\addcontentsline{toc}{chapter}{コピーレフトに関する通知}
\lettrine[lraise=-0.03,loversize=0.08]{こ}{の本}は\textbf{自由}なライセンスに従っており、
\urlref{https://www.gnu.org/philosophy/free-sw.en.html}{フリーソフトウェア}の哲学に基づいています:
この本を好きなように利用することができ、ソースは公開されています。また、この本を再配布したり、
あなた自身のバージョンを配布することもできます。つまり、印刷したり、コピーしたり、メールで送ったり、
ウェブサイトにアップロードしたり、変更したり、翻訳したり、リミックスしたり、一部を削除したり、
その上に何かを描いたりすることができます。

この本はコピーレフトです: 本を変更して自分のバージョンを配布する場合、受け取る人たちにもこれらの自由を認めなければなりません。
この本はCreative Commons Attribution-ShareAlike 4.0 International License
(\href{http://creativecommons.org/licenses/by-sa/4.0/}{\acronym{CC BY-SA 4.0}})を使用しています。

\@openrighttrue\makeatother
\afterpage{\blankpage}