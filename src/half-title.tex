% !TEX root = ctfp-print.tex
\input{version}

\thispagestyle{empty}

\vspace*{80pt}

\begin{raggedleft}
  \fontsize{24pt}{24pt}\selectfont
  % \textbf{Category Theory \\ for Programmers}\\
  \textbf{プログラマのための圏論}\\
  \ifdefined\OPTCustomLanguage{%
    \vspace*{1cm}
    \small\selectfont{
      \textbf{\OPTDisplayLanguageName{} Edition}\\
      \textit{Contains code snippets in Haskell and \OPTDisplayLanguageName}\\
    }
  }
  \fi
  \vspace*{1cm}
  \fontsize{16pt}{18pt}\selectfont \textbf{Bartosz Milewski}著\\
  \vspace{1cm}
  \fontsize{12pt}{14pt}\selectfont \textbf{Igal Tabachnik}編\\
  \fontsize{12pt}{14pt}\selectfont \textbf{Isao Sonobe}訳\\

\end{raggedleft}


\newpage

\vspace*{0.3\textheight}
\thispagestyle{empty}

\begin{small}
  \begin{center}

    \textsc{Category Theory for Programmers}\\

    \vspace{1.0em}
    \noindent
    Bartosz Milewski\\

    \vspace{1.26em}
    \noindent
    Version \texttt{\OPTversion}\\\today


    \vspace{1.6em}
    \noindent
    \includegraphics[width=3mm]{fig/icons/cc.pdf}
    \includegraphics[width=3mm]{fig/icons/by.pdf}
    \includegraphics[width=3mm]{fig/icons/sa.pdf}

    \vspace{0.4em}
    \noindent
    This work is licensed under a Creative Commons\\
    Attribution-ShareAlike 4.0 International License
    (\href{http://creativecommons.org/licenses/by-sa/4.0/}{\acronym{CC BY-SA 4.0}}).

    \vspace{1.26em}
    \noindent
    Converted from a series of blog posts by \href{https://bartoszmilewski.com/2014/10/28/category-theory-for-programmers-the-preface/}{Bartosz Milewski}.\\
    PDF and book compiled by \href{https://hmemcpy.com}{Igal Tabachnik}.\\
    Translated to Japanese by \href{https://github.com/sonoisa}{Isao Sonobe}.\\
    \vspace{1.26em}
    \noindent
    \LaTeX{} source code is available on GitHub: \\
    \href{https://github.com/hmemcpy/milewski-ctfp-pdf}{https://github.com/hmemcpy/milewski-ctfp-pdf}\\
    \href{https://github.com/sonoisa/milewski-ctfp-pdf-japanese}{https://github.com/sonoisa/milewski-ctfp-pdf-japanese}
  \end{center}
\end{small}
